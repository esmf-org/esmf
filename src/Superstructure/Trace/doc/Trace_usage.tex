% $Id$
%
% Earth System Modeling Framework
% Copyright 2002-2017, University Corporation for Atmospheric Research, 
% Massachusetts Institute of Technology, Geophysical Fluid Dynamics 
% Laboratory, University of Michigan, National Centers for Environmental 
% Prediction, Los Alamos National Laboratory, Argonne National Laboratory, 
% NASA Goddard Space Flight Center.
% Licensed under the University of Illinois-NCSA License.

%\subsection{Use and Examples}

ESMF tracing is disabled by default. To enable tracing, set the
{\tt ESMF\_RUNTIME\_TRACE} environment variable to {\tt ON}. You
do not need to recompile your code to enable tracing.

\begin{verbatim}
# csh shell
$ setenv ESMF_RUNTIME_TRACE ON

# bash shell
$ export ESMF_RUNTIME_TRACE=ON
\end{verbatim}

When enabled, the default behavior is to trace all PETs of the
ESMF application. Although the ESMF tracer is designed to write 
events in a compact form, tracing can produce an extremely
large number of events depending on the total number of PETs and
the length of the run. To reduce output, it is possible to restrict
the PETs that produce trace output by setting the {\tt ESMF\_RUNTIME\_TRACE\_PETLIST}
environment variable. For example, this setting:

\begin{verbatim}
$ setenv ESMF_RUNTIME_TRACE_PETLIST "0 101 192-196"
\end{verbatim}

will instruct the tracer to only trace PETs 0, 101, and 192 through 196
(inclusive). The syntax of this environment variable is to list
PET numbers separated by spaces. PET ranges are also supported using
the ``X-Y'' syntax where X < Y. For PET counts greater than 100, it is
recommended to set this environment variable. The one exception is that
PET 0 is always traced, regardless of the {\tt ESMF\_RUNTIME\_TRACE\_PETLIST}
setting.

When tracing is enabled, {\tt phase\_enter} and {\tt phase\_exit} events will
automatically be recorded for all initialize, run, and finalize phases of all
Components in the application. To trace {\em only} user-instrumented regions (via
the {\tt ESMF\_TraceRegionEnter()} and {\tt ESMF\_TraceRegionExit()} calls),
Component-level tracing can be turned off by setting:

\begin{verbatim}
$ setenv ESMF_RUNTIME_TRACE_COMPONENT OFF
\end{verbatim}

After running an ESMF application with tracing enabled, a directory
called {\em traceout} will be created in the run directory and it will
contain a {\em metadata} file and an event stream file {\em esmf\_stream\_XXXX}
for each PET with tracing enabled. Together these files form a valid
CTF trace which may be analyzed with any of the tools listed above.
