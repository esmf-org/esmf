% $Id: AppDriver_desc.tex,v 1.6 2003/11/13 22:36:29 nscollins Exp $
%
% Earth System Modeling Framework
% Copyright 2002-2003, University Corporation for Atmospheric Research, 
% Massachusetts Institute of Technology, Geophysical Fluid Dynamics 
% Laboratory, University of Michigan, National Centers for Environmental 
% Prediction, Los Alamos National Laboratory, Argonne National Laboratory, 
% NASA Goddard Space Flight Center.
% Licensed under the GPL.

%\subsection{Description}

The AppDriver directory contains examples of how to structure 
Applications using the ESMF Superstructure in a variety of styles.
Overall options when deciding how to structure an Application include
choices about:

\begin{description}

\item[Serial vs. Concurrent Execution]

In a Serial Execution model each processor 
executes the same Component until it has produced data needed
by another Component, and then all processors change to
running the next Component.  This has the appeal of simplicity
of data consumption and production: when a Component starts
all required data is available for use, and when a Component finishes
all data produced is ready for consumption by the next Component. 
This approach also has
the possibility of less data movement if the gridding and
data decomposition is done such that each processor's memory contains
the data needed by the next Component.

In a Concurrent Execution model subgroups of processors run
a Component and all Components are active at the same time.  
Data exchange must be coordinated between Components so that
data deadlock does not occur.  This strategy has the advantage
of allowing coupling to other Components at any time during
the computational process, including not having to return to
the calling level of code before making data available to 
other components.

\item[Coupling Strategy]

Coupling components are responsible for taking data from one
Component and putting it into the form expected by another Component.
This might include regridding, change of units, averaging or binning.

Couplers can be written for {\it pairwise} data exchange: the Coupler takes
data from a single Component and transforms it for another single Component.
This simplifies the structure of the Coupler Component code.

Couplers can also be written using a {\it hub and spoke} model where a
single Coupler accepts data from all other Components, can do data
merging or splitting, and formats data for all other Components.

Multiple Couplers, using either of the above two models or some mixture of
these approaches, are also possible.

\item[Implementation Language]

The ESMF framework is implemented with a set of Fortran and C++ interfaces
to all functions.  The main executable program can be written in either
Fortran or C++.

\item[Number of executables]

On a multiple processor machine a cooperating job can be run 
by starting the same executable on all nodes.  All processors run the
same code, but the computation proceeds in parallel by each processor 
working on a different subset of data.  This is a {\tt SPMD} model, 
Single Program Multiple Data.  

The alternative is to start a different executable on different
processors.  This is a {\tt MPMD} model, Multiple Program Multiple Data.
There are complications with many job control systems on multiprocessor
machine in getting the different executables started, and getting
inter-process communcations established.

\end{description}


Examples of these combinations are found in the ESMF source code tree, in
subdirectories under {\tt src/Superstructure/AppDriver}.
A good place to begin is with
{\tt seq\_pairwise\_fdriver\_spmd} which
has sequential component execution, a pairwise coupler, a main program
in Fortran, and all processors launch the same executable.

