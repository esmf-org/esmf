% $Id: GridComp_rest.tex,v 1.3 2004/06/15 23:23:54 cdeluca Exp $
%
% Earth System Modeling Framework
% Copyright 2002-2003, University Corporation for Atmospheric Research, 
% Massachusetts Institute of Technology, Geophysical Fluid Dynamics 
% Laboratory, University of Michigan, National Centers for Environmental 
% Prediction, Los Alamos National Laboratory, Argonne National Laboratory, 
% NASA Goddard Space Flight Center.
% Licensed under the GPL.

%\subsubsection{Restrictions and Future Work}

\begin{enumerate}

\item {\bf No concurrent components.}  While the design of concurrently 
running Components is fairly complete, this release of the framework does 
not support them.  Only sequentially executing Components can be run.

\item {\bf No Transforms.}  Components must exchange data through   
{\tt ESMF\_State} objects.  The input data are available at the time 
the user Component code is called, and data to be returned to another 
Component are available when that code returns.  
{\tt ESMF\_Xform} objects provide a way for
a Component to prepare data to be transformed and 
sent to another Component from within the execution of 
the user Component code.
Transforms are not implemented in this version of the framework.

\item {\bf Data isolation.} 
Gridded Components must only communicate with other
components via data in State objects.  They must 
not make direct references to data in other States.

\item {\bf Namespace isolation.}
If possible, Gridded Components should attempt to make 
all data private, so public names do not interfere with data 
in other components.

\item {\bf Single execution mode.}
It is not expected that a single Gridded Component be able 
to function in both sequential and concurrent modes, although 
Gridded Components of different types can be nested. For example,
a concurrently called Gridded Component can contain several nested 
sequential Gridded Components.   However, only sequentially executing
Components are supported in this release of the framework.

\end{enumerate}
