% $Id$

Debugging failing applications is often a challenging task. Massive
parallelism, issues with compute node access, and large data volumes (just to
name a few typical HPC aspects) add to the difficulties. For coupled
applications, built from many individual components and libraries, additional
complexity is introduced by the many layers of software.

For applications utilizing ESMF, the ESMF library is one of those software
layers. Due to the "framework" nature of ESMF, the situation can be
more subtle than for "simple" libraries. This is because ESMF code
is called from user code (as for "simple" libraries), as well as calling
back into registered user code (the "framework" aspect of ESMF). The
consequences of this fact relating to debugging of applications are discussed
in this section.

One consequence of the "framework" nature is that ESMF code is executing between
major portions of user code. For instance, when one ESMF component calls into
another ESMF component, the control flow goes through the ESMF software layer.
This provides ESMF with a chance to write messages into an application wide
log file. In particular, for user code that has implemented standard return
code handling, ESMF can log an error trace in the event of detecting an
error condition. The ESMF Reference Manual discusses standard "Return Code
Handling" under a section of the same name.

By default, the application wide ESMF log output is written to files that are
named {\tt PET<nnn>.ESMF\_LogFile}, where {\tt <nnn>} is the number of the
persistent execution thread (PET) that is writing. Several characteristics of
the default log can be changed during the call to {\tt ESMF\_Initialize()}. In
order to take advantage of the ESMF log output, it is important
to ensure that the {\tt logkindflag} is set to {\tt ESMF\_LOGKIND\_MULTI}, which
is the default, or {\tt ESMF\_LOGKIND\_MULTI\_ON\_ERROR}. The latter is
recommended for production runs where extra log output is minimized, and the ESMF
log is only activated when an error is encountered.

Assuming that the ESMF log is active for a failing application, and the user
code follows the documented return code handling, the ESMF log files are among
the first files that should be inspected. The log files are written into the
working directory that was active during the application execution. Assuming
default log file naming, we recommend the following grep command to scan for
errors.

\begin{verbatim}
grep ERROR PET*.ESMF_LogFile
\end{verbatim}

A typical error trace looks similar to the following output. Here is an example
error trace for an application using the NUOPC layer.

\begin{verbatim}
20210317 150338.047 ERROR PET0 atm.F90:113       Invalid argument  - \
                                          Passing error in return code
20210317 150338.047 ERROR PET0 ATM:src/addon/NUOPC/src/NUOPC_ModelBase.F90:865
20210317 150338.047 ERROR PET0 esm:src/addon/NUOPC/src/NUOPC_Driver.F90:2570
20210317 150338.047 ERROR PET0 esm:src/addon/NUOPC/src/NUOPC_Driver.F90:1287
20210317 150338.047 ERROR PET0 esm:src/addon/NUOPC/src/NUOPC_Driver.F90:466
20210317 150338.047 ERROR PET0 esmApp.F90:64       Invalid argument  - \
                                           Passing error in return code
\end{verbatim}

The first two columns contain the wall clock information of when each individual
log message was written. The third and forth column indicate the type of the log
message, here {\tt ERROR}, and the PET number, respectively, The sixth column
contains information about the source file and line number. Finally the seventh
column and beyond contain information about the error.

Notice that error traces are logged in backward order. The first {\tt ERROR}
entry corresponds to the lowest level, where the error condition was first
detected. Here the error was first detected in file {\tt atm.F90}, at line 113.
The error is then propagated back up to the highest application level,
here ending up in file {\tt esmApp.F90}, line 64. Due to the framework nature
of ESMF discussed earlier, it is very common to see several layers of ESMF
library code in an error trace, as is the case in this example. Notice however,
that the error was first caught in "user code" {\tt atm.F90}.

Even if the lowest level indicated (i.e. the start of an error trace) is inside
the ESMF library, it does not immediately indicate an issue with ESMF code. In
such cases it is good to follow the error trace to the first user code entry,
and investigate what ESMF call is made just before that location.
Then consider looking at the specific information passed into the ESMF method
and ensure correctness.

There are situations where an application experiences a hard crash, either
triggered by the runtime library, or the operating system itself. In these cases
the ESMF log files are typically not as helpful, and might even be missing. A
hard crash that produces a code dump, a backtrace to stderr, or is caught under
a debugger, can still be a good source of information to track down the 
problematic issue.

An example backtrace for a hard crash is shown below.

\begin{verbatim}
Program received signal SIGFPE: Floating-point exception - erroneous arithmetic operation.

Backtrace for this error:
#0  0x7f9e45bed49f in ???
#1  0x40430e in realize
        at /tmp/AtmOcnProto/ocn.F90:149
#2  0x7f9e49568f8b in _ZNK5ESMCI13MethodElement7executeEPvPi
        at /tmp/esmf/src/Superstructure/Component/src/ESMCI_MethodTable.C:333
#3  0x7f9e49569e74 in _ZN5ESMCI11MethodTable7executeENSt7__cxx1112basic_stringIcSt11char_traitsIcESaIcEEEPvPiPb
        at /tmp/esmf/src/Superstructure/Component/src/ESMCI_MethodTable.C:519
#4  0x7f9e49568dea in c_esmc_methodtableexecuteef_
        at /tmp/esmf/src/Superstructure/Component/src/ESMCI_MethodTable.C:303
#5  0x7f9e4982d8da in __esmf_attachmethodsmod_MOD_esmf_methodgridcompexecute
        at /tmp/esmf/src/Superstructure/AttachMethods/src/ESMF_AttachMethods.F90:1278
#6  0x7f9e4a69fcd0 in initializeipdvxp04
        at /tmp/esmf/src/addon/NUOPC/src/NUOPC_ModelBase.F90:1263
#7  0x7f9e492b8d1a in _ZN5ESMCI6FTable12callVFuncPtrEPKcPNS_2VMEPi
        at /tmp/esmf/src/Superstructure/Component/src/ESMCI_FTable.C:2036
#8  0x7f9e492b605a in ESMCI_FTableCallEntryPointVMHop
        at /tmp/esmf/src/Superstructure/Component/src/ESMCI_FTable.C:765
#9  0x7f9e49654bc3 in _ZN5ESMCI3VMK5enterEPNS_7VMKPlanEPvS3_
        at /tmp/esmf/src/Infrastructure/VM/src/ESMCI_VMKernel.C:2195
#10  0x7f9e49668136 in _ZN5ESMCI2VM5enterEPNS_6VMPlanEPvS3_
        at /tmp/esmf/src/Infrastructure/VM/src/ESMCI_VM.C:1211
#11  0x7f9e492b64f0 in c_esmc_ftablecallentrypointvm_
        at /tmp/esmf/src/Superstructure/Component/src/ESMCI_FTable.C:922
#12  0x7f9e499f8678 in __esmf_compmod_MOD_esmf_compexecute
        at /tmp/esmf/src/Superstructure/Component/src/ESMF_Comp.F90:1216
#13  0x7f9e49e5e19b in __esmf_gridcompmod_MOD_esmf_gridcompinitialize
        at /tmp/esmf/src/Superstructure/Component/src/ESMF_GridComp.F90:1408
#14  0x7f9e4a63740e in loopmodelcompss
        at /tmp/esmf/src/addon/NUOPC/src/NUOPC_Driver.F90:2534
#15  0x7f9e4a641e92 in initializeipdv02p3
        at /tmp/esmf/src/addon/NUOPC/src/NUOPC_Driver.F90:1833
#16  0x7f9e4a6650c9 in initializep1
        at /tmp/esmf/src/addon/NUOPC/src/NUOPC_Driver.F90:467
#17  0x7f9e492b8d1a in _ZN5ESMCI6FTable12callVFuncPtrEPKcPNS_2VMEPi
        at /tmp/esmf/src/Superstructure/Component/src/ESMCI_FTable.C:2036
#18  0x7f9e492b605a in ESMCI_FTableCallEntryPointVMHop
        at /tmp/esmf/src/Superstructure/Component/src/ESMCI_FTable.C:765
#19  0x7f9e49654bc3 in _ZN5ESMCI3VMK5enterEPNS_7VMKPlanEPvS3_
        at /tmp/esmf/src/Infrastructure/VM/src/ESMCI_VMKernel.C:2195
#20  0x7f9e49668136 in _ZN5ESMCI2VM5enterEPNS_6VMPlanEPvS3_
        at /tmp/esmf/src/Infrastructure/VM/src/ESMCI_VM.C:1211
#21  0x7f9e492b64f0 in c_esmc_ftablecallentrypointvm_
        at /tmp/esmf/src/Superstructure/Component/src/ESMCI_FTable.C:922
#22  0x7f9e499f8678 in __esmf_compmod_MOD_esmf_compexecute
        at /tmp/esmf/src/Superstructure/Component/src/ESMF_Comp.F90:1216
#23  0x7f9e49e5e19b in __esmf_gridcompmod_MOD_esmf_gridcompinitialize
        at /tmp/esmf/src/Superstructure/Component/src/ESMF_GridComp.F90:1408
#24  0x401aaf in esmapp
        at /tmp/AtmOcnProto/esmApp.F90:58
#25  0x401ee4 in main
        at /tmp/AtmOcnProto/esmApp.F90:17
Floating exception (core dumped)
\end{verbatim}

Just as for the ESMF log, the backtrace is produced in reverse order, starting
at the lowest level where the problem was encountered, tracing all the way up
the call stack to {\tt main}. As expected, the backtrace contains many
layers of ESMF library code. Again this does not immediately indicate that there
is a problem in the library code. In this example user code is visible at the
very top of the stack {\tt esmApp.F90}, and at the very bottom {\tt ocn.F90}.
In fact the location of the division by zero is correctly identified by the
runtime library at line 149 in file {\tt ocn.F90}.

It is possible to force a hard crash within the ESMF library while logging
ERRORS or WARNINGS to the {\tt PET<nnn>.ESMF\_LogFile}. Doing so can be
advantageous because it may produce a code dump and/or backtrace at the
initial point of error without ESMF return code handling. There are two
settings controlling log message error handling. The first setting,
{\tt ESMF\_RUNTIME\_ABORT\_LOGMSG\_TYPES}, configures the ESMF library to abort
during specified log message types, such as {\tt ESMF\_LOGMSG\_ERROR}. When
setting {\tt ESMF\_RUNTIME\_ABORT\_LOGMSG\_TYPES}, multiple log message types
can be listed, such as {\tt ESMF\_LOGMSG\_ERROR,ESMF\_LOGMSG\_WARNING}. The
second setting, {\tt ESMF\_RUNTIME\_ABORT\_ACTION}, configures the ESMF library
abort action. By default, the ESMF library abort handler will call
{\tt MPI\_Abort}. This can be changed to {\tt SIGABRT} or {\tt SIGQUIT}, which
will raise their respective exceptions. These two settings can be configured in
the user environment before initializing ESMF or set in the configuration file
passed to {\tt ESMF\_Initialize}.

An example configuration file configuring ESMF to raise {\tt SIGABRT} during
{\tt ESMF\_LOGMSG\_ERROR} or {\tt ESMF\_LOGMSG\_WARNING}.

\begin{verbatim}
ESMF_RUNTIME_ABORT_ACTION: "SIGABRT"
ESMF_RUNTIME_ABORT_LOGMSG_TYPES: "ESMF_LOGMSG_ERROR,ESMF_LOGMSG_WARNING"
\end{verbatim}
