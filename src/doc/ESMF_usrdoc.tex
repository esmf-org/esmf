% $Id: ESMF_usrdoc.tex,v 1.1 2001/11/13 18:35:47 dneckels Exp $

\documentclass[]{article}

\usepackage{epsf}
\usepackage{html}
\usepackage[T1]{fontenc}

\textwidth 6.5in
\textheight 8.5in
\addtolength{\oddsidemargin}{-.75in}

\begin{document}

\bodytext{BGCOLOR=white LINK=#083194 VLINK=#21004A}

\begin{titlepage}

\begin{center}
{\Large Earth System Modeling Framework } \\
\vspace{.25in}
{\Large {\bf Earth System Modeling Framework User's Guide}} \\
\vspace{.25in}
{\large {\it David Neckels, Cecelia DeLuca, others}}
\vspace{.5in}
\end{center}

\begin{latexonly}
\vspace{5.5in}
\begin{tabular}{p{5in}p{.9in}}
\hrulefill \\
\noindent {\bf NASA High Performance Computing and Communications Program} \\
\noindent Earth and Space Sciences Project \\
\noindent CAN 00-OES-01 \\
\noindent http://www.esmf.ucar.edu \\
\end{tabular}
\end{latexonly}

\end{titlepage}

\tableofcontents

\newpage

\section{Introduction}

This Guide to the Earth System Modeling Framework (ESMF) is intended 
for a new user.  It outlines the scope, structure, and function of the 
ESMF software in order to help the user understand what benefits the ESMF 
can provide.  It also describes how to install the ESMF software and how 
to incorporate it into applications.

The ESMF Software Developer's Guide contains detailed information on how 
to extend the ESMF system.  Reference manuals are available for individual 
libraries within ESMF; see these for additional information on interfaces 
and usage.

%\section{Description}
\input{esmf_desc}

%\section{Installation}
\input{esmf_install}

%\section{Basic Usage and Conventions}
\input{esmf_usage}

\end{document}











