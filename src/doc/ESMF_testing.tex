% $Id: ESMF_testing.tex,v 1.17 2004/06/22 14:17:44 nscollins Exp $

\subsection{Running ESMF Self-Tests}
\label{testing}

Robustness and portability are primary goals of the ESMF development
effort.  To ensure that these goals are met, the ESMF includes a
comprehesive suite of tests.  They allow testing and validation of
everything from individual functions to complete system tests.  These
test suites are used by the ESMF development team as part of their
regular development process.  ESMF users can run the testing suites to
verify that the framework software was built and installed properly,
and is running correctly on a particular platform.

Test targets will compile the ESMF library if it has not already been built.

\subsubsection{Running ESMF Unit Tests}

\label{UnitTestDescription}
The unit tests provided with the ESMF library evaluate the following:
\begin{itemize}
\item correctness of individual functions
\item behavior of individual modules or classes
\item appropriate error handling
\end{itemize}

Unit tests can be run in either an exhaustive or a non-exhaustive (sanity check)
mode.  The exhaustive mode includes the sanity check tests.  Typically, sanity
checks for each ESMF capability include creating and destroying an object and 
testing its basic function using a valid argument set.  In the exhaustive mode,
a wide range of valid and non-valid arguments are evaluated for correct behavior.

\label{RunUnitTests}

The following commands are used to build and run the unit tests provided with 
the ESMF:
\begin{verbatim}
        gmake [ESMF_EXHAUSTIVE=<ON,OFF>] tests
        gmake [ESMF_EXHAUSTIVE=<ON,OFF>] tests_uni
\end{verbatim}

The {\tt tests\_uni} target runs the tests on a single processor. 
The {\tt tests} target runs the test on multiple processors.

The non-exhaustive set of unit tests should all pass.  At this point in 
development, the exhaustive tests do not all pass.  Current problems with 
unit tests are being tracked and corrected by the ESMF development team.

The results of running the unit tests can be found in the following location:
\begin{verbatim}
${ESMF_DIR}/test/test${ESMF_BOPT}/${ESMF_ARCH}.${ESMF_COMPILER}.${ESMF_PREC}.${ESMF_SITE}
\end{verbatim}

For example, if your esmf source files have been placed in: 
\begin{verbatim}
       /usr/local/esmf
\end{verbatim}

If your platform is a Linux uni-processor that has an installed Lahey
Fortran compiler and ESMF\_COMPILER has been set to lahey, then the build
system configuration file will be:

\begin{verbatim}
      build_config/Linux.lahey.default/build_rules.mk
\end{verbatim}

If you want to run a debug version of non-exhaustive unit tests,
then you use these commands from /usr/local/esmf:

\begin{verbatim}
       setenv ESMF_DIR /usr/local/esmf
       gmake ESMF_BOPT=g ESMF_SITE=lahey ESMF_EXHAUSTIVE=OFF tests_uni
\end{verbatim}


If you are using ksh, then replace the setenv command with:
\begin{verbatim}
       export ESMF_DIR=/usr/local/esmf
\end{verbatim}

The results of the unit tests will be in:
\begin{verbatim}
       /usr/local/esmf/test/testg/Linux.lahey.32.default/
\end{verbatim}

At the end of unit test execution a script runs to analyze the results.

The script output indicates whether there are any unit test failures.
The following is a sample from the script output:

\begin{verbatim}

Unit Tests stdout files found: 
-rw-r--r--    1 svasquez scd           576 Apr 25 10:31 ESMF_ArrayBasicUTest.stdout
-rw-r--r--    1 svasquez scd           960 Apr 25 10:31 ESMF_ArrayF90PtrUTest.stdout
-rw-r--r--    1 svasquez scd          7791 Apr 25 10:31 ESMF_ArrayUTest.stdout
-rw-r--r--    1 svasquez scd            99 Apr 25 10:31 ESMF_BaseUTest.stdout
-rw-r--r--    1 svasquez scd          1690 Apr 25 10:31 ESMF_BundleUTest.stdout
-rw-r--r--    1 svasquez scd         73209 Apr 25 10:31 ESMF_ClockUTest.stdout
-rw-r--r--    1 svasquez scd          2585 Apr 25 10:31 ESMF_FieldUTest.stdout
-rw-r--r--    1 svasquez scd           399 Apr 25 10:31 ESMF_GridUTest.stdout
-rw-r--r--    1 svasquez scd          6484 Apr 25 10:32 ESMF_StateUTest.stdout

Unit test stdout files of zero length indicate that the unit test
did not run because it failed to compile or it failed to execute. 

112  Unit Tests passed 

No Unit Tests Failed.

\end{verbatim}

The following is an example of the output generated when a unit test fails:
\begin{verbatim}
ESMF_FieldUTest.stdout: FAIL  Unique default Field names Test, FLD1.5.1 & 1.7.1,
                        ESMF_FieldUTest.F90, line 204  Field names not unique
\end{verbatim}

\subsubsection{Running ESMF System Tests}
\label{SystemTestDescription}

The system tests provided with the ESMF library evaluate:
\begin{itemize}
\item interface agreement between parts of the system
\item behavior of the system as a whole
\end{itemize}

The current system test suite includes tests that perform layout
reduction operations, redistribution-transpose, halo operations,
component creation and intra-grid communication.  Some of the system
tests are no longer compatible with the current API, but are included
in the release for completeness.  A complete description of each
available system test and its current compatibility status can be
found at the ESMF website,
\htmladdnormallink{http://www.esmf.ucar.edu}{http://www.esmf.ucar.edu}.  
The testing
and validation page is accessible from the {\bf Development} 
link on the navigation bar.

The following commands are used to build and run the system tests:

\begin{verbatim}
        gmake [SYSTEM_TEST=xxx] system_tests
        gmake [SYSTEM_TEST=xxx] system_tests_uni
\end{verbatim}

The {\tt system\_tests\_uni} target runs the tests on a single processor. 
The {\tt system\_tests} target runs the test on multiple processors.

If a particular SYSTEM\_TEST is not specified, then all available system tests 
are built and run.

The results of the test can be found in the following location:
\begin{verbatim}
${ESMF_DIR}/test/test${ESMF_BOPT}/${ESMF_ARCH}.${ESMF_COMPILER}.${ESMF_PREC}.${ESMF_SITE}
\end{verbatim}

For example, if your ESMF source files have been placed in your home directory:
\begin{verbatim}
       ~/esmf
\end{verbatim}

and your platform and compiler configuration is:
\begin{verbatim}
       Alpha multi-processor using the native compiler
\end{verbatim}

and you want to run an optimized version of system test SimpleCoupling,
then you use these commands from the directory {\tt \~/esmf}. 
\begin{verbatim}
       setenv ESMF_PROJECT <project_name>
       gmake ESMF_DIR=`pwd` SYSTEM_TEST=ESMF_SimpleCoupling system_tests
\end{verbatim}

If you are using ksh then replace the setenv command with
this:

\begin{verbatim}
       export ESMF_PROJECT=<project_name>
\end{verbatim}

The results will be in:
\begin{verbatim}
~/esmf/test/testO/OSF1.default.64.default/ESMF_SimpleCouplingSTest.stdout
\end{verbatim}

\subsubsection{Running the ESMF VAlidation (EVA) Suite}
\label{EVATestDescription}

The ESMF VAlidation(EVA) Suite is a collection of seven codes
representative of those used in climate, weather, and data
assimulation. These codes are currently used for ESMF prototyping.
They will eventually provide the basis for ESMF tutorial examples.

The {\it EVA Suite User's Guide} and source code can be downloaded from the {\bf
User Links and Downloads} link on the ESMF website, 
\htmladdnormallink{http://www.esmf.ucar.edu}{http://www.esmf.ucar.edu}.

