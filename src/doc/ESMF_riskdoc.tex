% $ Id: $

\documentclass[english]{article}
\usepackage{babel}
\usepackage{supertabular}
\usepackage{html}
\usepackage{times}
\usepackage{graphicx}
\usepackage[T1]{fontenc}

\newcommand{\docmttype}{Risk Report}
\newcommand{\req}[1]{\section{\hspace{.2in}#1}}
\newcommand{\sreq}[1]{\subsection{\hspace{.2in}#1}}
\newcommand{\ssreq}[1]{\subsubsection{\hspace{.2in}#1}}
\newcommand{\mytitle}{DRAFT \longname \docmttype ~~}

\newenvironment
{reqlist}
{\begin{list} {} {} \rm \item[]}
{\end{list}}

%===============================================================================
% User-defined commands
%-------------------------------------------------------------------------------
\newcommand{\longname}{ESMF }
\newcommand{\funcname}{ESMF }
\newcommand{\shortname}{RSK}
\newcommand{\myauthors}{ESMF Technical Leads:  V. Balaji, Cecelia DeLuca, Chris Hill}
%===============================================================================
\setlength{\textwidth}{6.5truein}
\setlength{\textheight}{8.5truein}
\setlength{\oddsidemargin}{0in}
\setlength{\unitlength}{1truecm}

\begin{document}

\bodytext{BGCOLOR=white LINK=#083194 VLINK=#21004A}

% Title page

% $Id: title_alldoc.tex,v 1.1 2002/01/25 20:47:59 cdeluca Exp $

\begin{titlepage}

\begin{center}
{\Large Earth System Modeling Framework } \\
\vspace{.25in}
{\Large {\bf \mytitle}} \\
\vspace{.25in}
{\large {\it \myauthors}}
\vspace{.5in}
\end{center}

\begin{latexonly}
\vspace{5in}
\begin{tabular}{p{5in}p{.9in}}
\hrulefill \\
\noindent {\bf NASA Earth Science Technology Office} \\
\noindent Computational Technologies Project \\
\noindent CAN 00-OES-01 \\
\noindent http://www.esmf.ucar.edu \\
\end{tabular}
\end{latexonly}

\end{titlepage}








\newpage
\tableofcontents

\newpage
\section{Synopsis}

This document lists a series of risks that the Earth System Modeling Framework project faces
in delivering a system that meets the requirements of 

\begin{enumerate}
\item adequate performance
\item general present and future applicability to a range of applications and a broad mix of hardware platforms
\item reasonable ease of adoption
\item ongoing extensibility
\end{enumerate}

Risks have been categorized as {\bf Very High}, {\bf High}, {\bf Medium}, {\bf Fairly Low} and {\bf Low}.
This classification is used for both measuring impact on the ESMF project ( the {\it Level of Impact On Project})
category and for quantifying the currently perceived exposure to the risk within
the project ( the {\it Problem Likelihood} attribute), so that the
status of the risk is a pair of values, such as ({\bf Low,Low}), that expresses
both the risk seriousness (the first element) and the risk impact (the second element) on
ESMF.  This document is reviewed on a monthly basis by the technical lead team and adjustments
made. Preventitive measures are to be taken for any risk that has a status greater than {\bf Medium, Medium}.
Fix actions are taken for risks with a ({\bf High, Medium}) status or higher.

\req{Community Adoption Issues}
\sreq{Inadequate community dialog}
\begin{reqlist}
{\bf Level of Impact On Project:}  High \\
{\bf Problem Likelihood:} Fairly Low\\
{\bf Preventive Measures Being Taken:} Community meetings. Active participation by project
members in related community efforts. Deployment of wide number of community codes
onto the framework. Attendance by project participants at related project
meetings.\\
{\bf Fix:} Increase community pariticpation in project. Improve features to make
adoption more compeling.\\
{\bf Notes:} \\
\end{reqlist}

\sreq{Unable to respond to all commuity needs}
\begin{reqlist}
{\bf Level of Impact On Project:}  Medium \\
{\bf Problem Likelihood:} Medium \\
{\bf Preventitive Measures Being Taken:} Ongoing dialog to identify an prioritize
community needs. Critical requirements will be adopted to the extent
that contractual obligactions admit.\\
{\bf Fix:} Seek resources for requirements that cannot be accomodated within current
contract obligations. Discuss contract modifications with NASA if truly critical
need (as determined by advisory board members) can not be addressed. \\
{\bf Notes:} \\
\end{reqlist}

\sreq{Overcommitment to support community}
\begin{reqlist}
{\bf Level of Impact On Project:} High \\
{\bf Problem Likelihood:} Medium \\
{\bf Preventitive Measures Being Taken:} Clear emphasis of project milestones. Implementation
team only accountable to technical leads.\\
{\bf Fix:} \\
{\bf Notes:} \\
\end{reqlist}

\sreq{Unable to achieve community adoption}
\begin{reqlist}
{\bf Level of Impact On Project:}  High \\
{\bf Problem Likelihood:} Fairly low \\
{\bf Preventitive Measures Being Taken:} Broad range of dialog and visibility 
forums being maintained. Example codes and workshops are being planned.
Close monitoring of experiences of collaborator groups.
\\
{\bf Fix:} Solicit non-adopter input on what is missing or why framework is not working for them.\\
{\bf Notes:} \\
\end{reqlist}

\req{Superstructure Level Issues}
\sreq{Arbitrary SPMD and MPMD support}
The {\it Control} layer in ESMF is required to manage arbitrary collections
of SPMD and MPMD programs running across a potentially heterogeneous
mix of platforms.
\begin{reqlist}
{\bf Level of Impact On Project:} High \\
{\bf Problem Likelihood:} Low \\
{\bf Preventitive Measures Being Taken:} Detailed prototype studies to demostrate all required
combinations.\\
{\bf Fix:} Reduce scope of support to a tractable level.\\
{\bf Notes:}
\end{reqlist}

\req{System Issues}
\sreq{Interoperation between languages}
ESMF will use a mix of Fortran, C and C++ languages. Several interlanguage binary binding
issues need to be addressed covering both normal execution and exceptional
behavior.
\begin{reqlist}
{\bf Level of Impact On Project:} High \\
{\bf Problem Likelihood:} Low\\
{\bf Preventitive Measures Being Taken:} Significant prototyping in being undertaken and
the language boundaries are being very clearly delineated.\\
{\bf Fix:} Entire framework would need to be in C/C++.\\
{\bf Notes:} Reimplementation in C/C++ would be time-consuming but
otherwise straightforward.
\end{reqlist}

\sreq{F90 Continued Availability}
Fortran 90/95 remain {\it fringe} languages, even though most of the Earth system
community uses them. Two consequences of this are that ESMF Fortran 90 code will
rely on vendor supplied compilers that have an ever decreasing user base and
that there is no true open-source compiler for Fortran 90/95.
This impacts ESMF directly because some of its code will use F90 and
indirectly because many of its components exist in F90 only.
\begin{reqlist}
{\bf Level of Impact On Project:} High \\
{\bf Problem Likelihood:} Low\\
{\bf Preventitive Measures Being Taken:} Some framework development is being pursued in C/C++.
Availability and quality of Fortran 90/95 compilers
is being tracked.\\
{\bf Fix:} Entire framework would need to be in C/C++.\\
{\bf Notes:} Reimplementation in C/C++ would be time-consuming but
otherwise straightforward. For components the impact would be far more
serious, but this is outside the scope of ESMF.
Components in Fortran 77, C and C++ would not be affected by lack of 
compiler availability as compilers exist almost all hardware for these languages.
\end{reqlist}

\sreq{Vector Platform Availability of ESMF}
\begin{reqlist}
{\bf Level of Impact On Project:} Medium \\
{\bf Problem Likelihood:} Fairly Low\\
{\bf Preventitive Measures Being Taken:} ESMF being tested and built on a range of *NIX systems \\
{\bf Fix:} Test and modify to build on apropriate system\\
{\bf Notes:} This could require resources but otherwise would not be a major challenge.
\end{reqlist}

\sreq{MS Windows Availability of ESMF}
\begin{reqlist}
{\bf Level of Impact On Project:} Fairly Low \\
{\bf Problem Likelihood:} High \\
{\bf Preventitive Measures Being Taken:}  ESMF being tested and built on a range of *NIX systems.
No specific tests of build actions are envisaged to ensure Windows compatability.\\
{\bf Fix:} Test and modify to build on a Windows system\\
{\bf Notes:} This could require resources but otherwise would not be a major challenge.
\end{reqlist}

\sreq{Other OS Validation}
\begin{reqlist}
{\bf Level of Impact On Project:} Low \\
{\bf Problem Likelihood:} High \\
{\bf Preventitive Measures Being Taken:} ESMF being tested and built on a range of *NIX systems\\
{\bf Fix:}  Test and modify to build on a Windows system\\
{\bf Notes:} This could require resources but otherwise would not be a major challenge.
\end{reqlist}

\req{Infrastructure - Fields and Grid Level Issues}
\sreq{Too many complex features}
Many different grids will be supported under ESMF. A manageable and scalable approach
to development and testing is required to enable robust support for different grids.
\begin{reqlist}
{\bf Level of Impact On Project:} High \\
{\bf Problem Likelihood:} Medium \\
{\bf Preventitive Measures Being Taken:} A layered interface has been defined 
which highlights generic grid properties and expresses them through common abstractions.
Many groups will attempt to deploy the system in their own in-house
applications and their experiences will be fed-back into the system design.\\
{\bf Fix:} Interface redesign \\
{\bf Notes:} 
\end{reqlist}
\sreq{Inadequate performance for certain grids}
Grid operations, like halo-update and transpose, can have significant impact on parallel scaling.
Regriddings between different grids can also greatly impact scaling.
A general purpose system that can scale and perform as well as specialized, application and
grid specific forms, could be hared to devise. Definciencies in this areana will
impact adoption.
\begin{reqlist} 
{\bf Level of Impact On Project:} High \\ 
{\bf Problem Likelihood:} Medium \\
{\bf Preventitive Measures Being Taken:} Quantitative measures of grid primitives
performance are being devised and will be tracked.\\
{\bf Fix:} Develop specialized forms and increase level of resources
to developing these parts of the system.\\
{\bf Notes:} 
\end{reqlist}
\sreq{Inadequate performance for certain platforms}
Grid operations can be spcialized to take advatage of platform specific
architectural features. In a general purpose system it can be hard to compete with these
specialized forms.
\begin{reqlist}
{\bf Level of Impact On Project:} Medium \\
{\bf Problem Likelihood:} Medium \\
{\bf Preventitive Measures Being Taken:} Quantitative measures of grid primitives
performance are being devised and will be tracked.\\
{\bf Fix:} Develop specialized forms and increase level of resources
to developing these parts of the system.\\
{\bf Notes:} 
\end{reqlist}

\req{Infrastructure - Utilities Level Issues}
Inadequate reqources for developing highly portable and versatile
signaling, prefromance monitoring, I/O, communication, memory sharing and namespace sharing tools.
\begin{reqlist}
{\bf Level of Impact On Project:} High \\
{\bf Problem Likelihood:} Medium \\
{\bf Preventitive Measures Being Taken:} Tools developed under other projects (for example PetSC, ALICE, TAU)
will be adopted wherever possible.
\\
{\bf Fix:} Devote more resources to this area \\
{\bf Notes:} 
\end{reqlist}


\end{document}
