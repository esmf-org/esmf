% $Id$
%
% Earth System Modeling Framework
% Copyright (c) 2002-2025, University Corporation for Atmospheric Research, 
% Massachusetts Institute of Technology, Geophysical Fluid Dynamics 
% Laboratory, University of Michigan, National Centers for Environmental 
% Prediction, Los Alamos National Laboratory, Argonne National Laboratory, 
% NASA Goddard Space Flight Center.
% Licensed under the University of Illinois-NCSA License.

\section{Overview of Infrastructure Utility Classes}

The ESMF utilities are a set of tools for quickly assembling modeling applications.

The ESMF Info class enables models to be self-describing via metadata, which are instances of JSON-compatible key-value pairs.

The Time Management Library provides utilities for time and time interval representation and calculation, and higher-level utilities that control model time stepping, via clocks, as well as alarming.

The ESMF Config class provides configuration management based on NASA DAO's Inpak package, a collection of methods for accessing files containing input parameters stored in an ASCII format.

The ESMF LogErr class consists of a variety of methods for writing error, warning, and informational messages to log files. A default Log is created during ESMF initialization. Other Logs can be created later in the code by the user.

The DELayout class provides a layer of abstraction on top of the Virtual Machine (VM) layer. DELayout does this by introducing DEs (Decomposition Elements) as logical resource units. The DELayout object keeps track of the relationship between its DEs and the resources of the associated VM object. A DELayout can be shaped by the user at creation time to best match the computational problem or other design criteria.

The ESMF VM (Virtual Machine) class is a generic representation of hardware and system software resources. There is exactly one VM object per ESMF Component, providing the execution environment for the Component code. The VM class handles all resource management tasks for the Component class and provides a description of the underlying configuration of the compute resources used by a Component.  In addition to resource description and management, the VM class offers the lowest level of ESMF communication methods.

The ESMF Fortran I/O utilities provide portable methods to access capabilities which are often implemented in different ways amongst different environments. Currently, two utility methods are implemented: one to find an unopened unit number, and one to flush an I/O buffer.
