%===============================================================================
% CVS $Id: testing.tex,v 1.4 2006/08/04 21:03:06 cdeluca Exp $
% CVS $Source: /mnt/twixshare/Storage/Archive-SF-Repos/ESMF_CVS_Repo/esmf/src/doc/dev_guide/testing.tex,v $
% CVS $Name:  $
%===============================================================================

\section{Release Schedule}
\label{sec:build}

The ESMF Release Schedule is generated by the CRB during its
quarterly meetings.  It's posted on the ESMF website home
page under {\bf Quick Links}.

\section{Core Team Task List}

The Core Team maintains a Task List to archive past, current,
and future development tasks.  This tool is
what the CRB uses during their meetings to build up a new
release schedule.  It's updated by the Core Team Manager 
quarterly, prior to CRB meetings.  Like other SourceForge
tools it is browsable by anyone.  The Task List is at:  
\htmladdnormallink{http://www.sourceforge.net/projects/esmf} 
{http://www.sourceforge.net/projects/esmf}, under the 
{\it Tasks} link. 

\section{Testing and Validation}
\label{sec:testing}

ESMF software is tested using the following strategies:
\begin{enumerate}
\item unit test - simple per-class tests
\item system test - generally involve inter-component functions
\item use test case (UTC) - tests at realistic problem sizes (e.g., large data sets, processor counts, grids)
\item examples - range from simple to complex
\end{enumerate}
Unit tests, system tests, and examples are distributed with the
ESMF software.  Use test cases, because of their size, are 
stored and distributed separately.  Tests are run nightly,
following a weekly schedule, on a wide variety of platforms.  

The ESMF team keeps track of test coverage on a per-method basis.
This information is on the {\bf Metrics} page under the {Development}
link on the navigation bar.

Testing information is stored on a {\bf Test and Validation} webpage,
under the {\bf Development} link on the ESMF 
website.  This webpage includes:
\begin{itemize}
\item separate webpages for each system test and use test case;
\item separate webpages for the results of each system test;
\item links to the {\it Developer's Guide}, SourceForge Tracker, Requirements 
Spreadsheet, and any other pertinent information; and
\item separate webpage for automated regression test information and results.
\end{itemize}

\subsection{Unit Tests}

Each class in the framework is associated with a suite of unit tests.
Typically the unit tests are stored in one file per class, and are
located near the corresponding source code in a test directory.  The 
framework {\tt make} system will have an option to build and run unit tests.
The user has the option of building either a "sanity check" type test
or an exhaustive suite. The exhaustive tests include tests of many 
functionalities and a variety of valid and invalid input values. The sanity 
tests are a minumum set of tests to indicate whether, for example, the 
software has been installed correctly. It is the responsiblity of the 
software developer to write and execute the unit tests. Unit tests 
are distributed with the framework software.

\subsubsection{Unit Test Goals and Strategies}

To achieve adequate unit testing, developers shall attempt to meet the following goals. 

\begin{itemize}
\item Individual procedures will be evaluated with at least one test function.  However,
as many test functions as necessary will be implemented to assure that 
each procedure works properly.  
\item The result of each test will be a {\tt PASS/FAIL}.  
In some cases for floating point comparisons, an epsilon value will be used.
\item Unit tests will be implemented for each language interface that is 
supported.
\item Developers will unit test their code to the degree possible  
before it is checked into the repository.  It is assumed that 
developers will use stubs as necessary.
\item Unit testing will be executed on all supported platforms.
\item Unit testing will include a variety of paradigms (e.g. pure shared memory,
pure distributed memory, and a mix of both).
\item Unit testing will be executed on a range of configurations (e.g. uni-processor,
multi-processor).
\item Variables will be tested for acceptable range and precision.
\item Variables will be tested for a range of valid values, including boundary
values.
\item Unit tests will verify that error handling works correctly.
\end{itemize}

\subsection{System Tests}

System tests are written to test functionality that spans several 
classes.  The following areas are given special
consideration during our system testing.

\begin{itemize}
\item Design omissions (e.g. incomplete or incorrect behaviors).
\item Associations between objects (e.g. fields, grids, bundles).
\item Control and Infrastructure. (e.g. couplers, time management, error handling).
\item Feature interactions or side effects when multiple features are used
simultaneously.
\item Compatibility between previously working software releases and new releases.
\item Different behaviors among different platforms.
\item Language interfaces.
\end{itemize}

The system tester shall issue a test log after each software release is tested,
which shall be recorded on the {\bf Test and Validation} webpage. The test 
log shall
include: a test ID number, a software release ID number, testing environment 
descriptions, a list of test cases executed, results, and any unexpected 
events. Bugs will be documented in the \htmladdnormallink{{\it SourceForge Bug 
Tracker}}{https://sourceforge.net/tracker/?group\_id=38089&atid=421185} and 
any bug fixes shall be validated.

\subsubsection{System Test Goals and Strategies}

The ESMF is designed to run on several target platforms, in different 
configurations, and is required to interoperate with many combinations 
of application software. To increase our confidence in the quality in 
our software releases, the following system test goals shall be met.

\begin{itemize}
\item System tests shall be executed on all target platforms. Note: Successful
execution on all platforms may not be required for some software deliveries.
\item System tests shall be executed on a variety of programming paradigms
(e.g pure shared memory, pure distributed memory and a mix of both) as
applicable.
\item System testing shall be executed on as many configurations (e.g. uni-processor,
multi-processor) as applicable.
\end{itemize}

\subsection{Automated Regression Tests}

The purpose of regression testing is to reveal faults caused by new
or modified software (e.g. side effects, incompatibility between 
releases, and bad bug fixes).  Automated regression testing tools shall 
be available to ESMF developers.  
The regression tests will regularly exercise all interfaces of the code on 
all target platforms.  The gatekeeper (see Section~\ref{sec:cm}) runs 
automated nightly builds on a wide variety of machines, listed on the
{\bf Test and Validation} website. 
Log files of the build results are stored at {\tt /fs/projects/css/esmf/esmf\_test/daily\_builds} on the NCAR fileserver. 
The log files are kept on a monthly basis in 
separate directories. For example, the December 2002 directory is named {\tt 0212\_test}. The 
directory contains an {\tt ESMFdailyLog} file which has a one line entry for each platform on 
which the build was attempted indicating whether the build was successful or not. 

The \htmladdnormallink{{\tt ESMFdailyLog}}
{https://www.esmf.ucar.edu/test/daily\_test/ESMFdailyLog} file for the current month is available on the {\bf Test and Validation} website.

If the build is not successful, then a file containing the build output is stored in 
the {\tt /fs/projects/css/esmf/esmf\_test/daily\_builds} directory on the NCAR fileserver. The file naming convention for the build output file is 
{\tt build\_BOPT\_(day)(platform)ESMF\_ARCH}. For example, if the build fails on longs on the 19th of the month with {\tt BOPT=g}, the file would be called {\tt build\_g\_19longslinux\_lf95}. 

Email is sent daily to the ESMF core team listing all the platforms on which the build was attempted with a {\tt PASS/FAIL} indication. 

\subsubsection{Regression system tests with non-linear codes}
 For regression tests of full codes a two-stage
numerical checksum procedure will be used.

\begin{itemize}
\item A primary set of tests that
use bit-wise reproducible options shall be
maintained. These tests may not always use fully optimized
forms of user code or framework communication code.
A simple {\tt PASS/FAIL} metric based on bit-wise
check sums will be used in these tests.

\item A secondary set of tests that can validate
optimized framework code, in particular communication primitives, but
that may not yield exact bit reproducibility
with a non-linear user code will also
be maintained. For these tests a {\tt PASS/FAIL}
status will be established based on the number of
matching digits for an appropriately chosen measure.
This scheme will allow us to track any time
digits change between framework code changes, system upgrades,
system changes. The level of {\tt PASS/FAIL} will be determined for
each test configuration.

\end{itemize}

Together these sets of tests will allow fully optimized code
to be validated using very short (a few minutes or less of compute time)
test runs. These tests can be run automatically on a daily basis
to enable fine-scale monitoring of the system.

\subsection{Internal Releases}

ESMF software is released in a beta form, as an Internal Release,
three months before it is publicly released.  This gives users
a chance to test the software and report back any problems to 
support.

\subsection{Bug Tracking}

All software bugs found during testing are recorded
in a tracker on the ESMF SourceForge site called Bugs.

When a bug is identified, the tester opens a bug report on SourceForge, giving as much detail
about the bug as possible to help the developer reproduce the problem. When the developer
fixes the bug, he/she logs into SourceForge, adds a note to the bug report, and changes the status
to Pending. This triggers SourceForge to send an email to the originator of the bug report. It is
the responsibility of the originator to verify that the bug has been fixed and close the bug report.

CVS allows projects to be developed on several branches simultaneously. The ongoing development
is done on the main trunk, and as software releases are identified, branches are created and tagged. Once this
occurs, software development may continue on separate versions of the code. 

When a bug is found, it may exist on the trunk, a branch or both. The originator of the bug report
must state where the bug was found in the bug report.  A bug that exists on both the trunk and a 
branch, will generally be fixed only on the trunk. If the bug is fixed on both the trunk and branch, the 
developer must state this explicitly in the bug report.  Customer queries about known problems for a 
release will be directed to the developers.

\subsection{Software Release Test Procedures}

We provide two types of tar files, the ESMF source and the shared libraries of 
the supported platforms. Consequently, there are two test procedures followed before placing the 
tar files on the ESMF download website. 

\subsubsection{Source Code Test Procedure}

This procedure is followed on all the supported platforms for the particular release.

\begin{enumerate}
\item Verify that the source code builds in both {\tt BOPT=g} and {\tt BOPT=O}.
\item Verify that  the {\tt ESMF\_COUPLED\_FLOW} demonstration executes successfully.
\item Verify that the unit tests run successfully, and that there are no {\tt NON-EXHAUSTIVE} unit tests  failures.
\item Verify that all system tests run successfully. 
\end{enumerate}


\subsubsection{Shared Libraries Test Procedure}

\begin{enumerate}
\item Change to the {\tt CoupledFlowEx} directory and execute {\tt gmake}. Verify that the demo runs successfully.
\item Change to the {\tt CoupledFlowSrc} directory and execute {\tt gmake} then {\tt gmake run}. Verify that the demo runs successfully.
\item Change to the {\tt examples} directory and execute {\tt gmake} and {\tt gmake run}. Verify that the example runs successfully.
\end{enumerate}














