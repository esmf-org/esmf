%===============================================================================
% CVS $Id: testing.tex,v 1.13.2.1 2010/02/03 22:57:20 svasquez Exp $
% CVS $Source: /mnt/twixshare/Storage/Archive-SF-Repos/ESMF_CVS_Repo/esmf/src/doc/dev_guide/testing.tex,v $
% CVS $Name:  $
%===============================================================================

\subsection{Testing and Validation}
\label{sec:testing}

ESMF software is subject to the following tests:
\begin{enumerate}
\item Unit tests, which are simple per-class tests.
\item System tests, which generally involve inter-component interactions.
\item Use test cases (UTCs), which are tests at realistic problem
sizes (e.g., large data sets, processor counts, grids).
\item Examples that range from simple to complex.
\item Beta testing through preliminary releases.
\end{enumerate}
Unit tests, system tests, and examples are distributed with the
ESMF software.  UTCs, because of their size, are 
stored and distributed separately.  Tests are run nightly,
following a weekly schedule, on a wide variety of platforms.  
Beta testing of ESMF software is done by providing an Internal Release
to customers three months before public release.  

The ESMF team keeps track of test coverage on a per-method basis.
This information is on the {\bf Metrics} page under the {\bf Development}
link on the navigation bar.

Testing information is stored on a {\bf Test and Validation} web page,
under the {\bf Development} link on the ESMF 
web site.  This web page includes:
\begin{itemize}
\item separate web pages for each system test and UTC;
\item links to the {\it Developer's Guide}, SourceForge Tracker, Requirements 
Spreadsheet, and any other pertinent information; and
\item separate web page for automated regression test information and results.
\end{itemize}

The ESMF is designed to run on several target platforms, in different 
configurations, and is required to interoperate with many combinations 
of application software. Thus our test strategy includes the following.

\begin{itemize}
\item Tests are executed on as many target platforms as possible. 
\item Tests are executed on a variety of programming paradigms
(e.g pure shared memory, pure distributed memory and a mix of both).
\item Tests are executed in multiple configurations (e.g. uni-processor,
multi-processor).
\item The result of each test is a {\tt PASS/FAIL}.  
\item In some cases, for floating point comparisons, an epsilon value
will be used.
\item Tests are implemented for each language interface that is 
supported.
\end{itemize}

\subsubsection{Unit Tests}

Each class in the framework is associated with a suite of unit tests.
Typically the unit tests are stored in one file per class, and are
located near the corresponding source code in a test directory.  The 
framework {\tt make} system will have an option to build and run unit tests.
The user has the option of building either a "sanity check" type test
or an exhaustive suite. The exhaustive tests include tests of many 
functionalities and a variety of valid and invalid input values. The sanity 
check tests are a minimum set of tests to indicate whether, for example, the 
software has been installed correctly. It is the responsibility of the 
software developer to write and execute the unit tests. Unit tests 
are distributed with the framework software.

To achieve adequate unit testing, developers shall attempt to meet the following goals. 

\begin{itemize}
\item Individual procedures will be evaluated with at least one unit
test function.  However, as many test functions as necessary will be
implemented to assure that each procedure works properly.  
\item Developers should unit test their code to the degree possible  
before it is checked into the repository.  It is assumed that 
developers will use stubs as necessary.
\item Variables are tested for acceptable range and precision.
\item Variables are tested for a range of valid values, including boundary
values.
\item Unit tests should verify that error handling works correctly.
\end{itemize}

\paragraph{Writing Unit Tests}

Unit tests usually test a single argument of a method to make it easier to
identify the bug when a unit test fails.
There are several steps to writing a unit test.
First, each unit test must be labled with one of the following tags:
\begin{itemize}
\item NEX\_UTest - This tag signifies a non-exhaustive test. These tests are always run and
are considered to be sanity tests they usually consist of creating and destroying a specific class.
\item EX\_UTest - This tag signifies an exhaustive unit test. These tests are more rigorous and
are run when the ESMF\_EXHAUSTIVE environmental variable is set to ON. These unit test must be between the \#ifdef ESMF\_EXHAUSTIVE
and \#endif definitions in the unit test file.
\item NEX\_UTest\_Multi\_Proc\_Only - These are non-exhaustive multi-processor unit tests that will not be
run when the run\_unit\_tests\_uni or unit\_test\_uni targets are specified.
\item EX\_UTest\_Multi\_Proc\_Only - These are exhaustive multi-proccesor unit tests that will not be
run when the run\_unit\_tests\_uni or unit\_tests\_uni targets are specified.
\end{itemize}
Note that when the NEX\_UTest\_Multi\_Proc\_Only or EX\_UTest\_Multi\_Proc\_Only tags are used, all the unit tests in
the file must be labeled as such. You may not mix these tags with the other tags. In addition, verify that the makefile
does not allow the unit tests with these tags to be run uni.

Second, a string is specified describing the test, for example:
\begin{verbatim}

	write(name, *) "Grid Destroy Test"

\end{verbatim}
Third, a string to be printed when the test fails is specified, for example:
\begin{verbatim}

	write(failMsg, *) "Did not return ESMF_SUCCESS"

\end{verbatim}
Fourth, the ESMF\_Test subroutine is called to determine the test results, for example:
\begin{verbatim}

	call ESMF_Test((rc.eq.ESMF_SUCCESS), name, failMsg, result, ESMF_SRCLINE)

\end{verbatim}
The following two tests are good examples of how unit tests should be written.
The first test verify that getting the attribute count from a Field returns ESMF\_SUCCESS, while
the second verifies the attribute count is correct. These two tests could be combined into one
with a logical AND statement when calling ESMF\_Test, but breaking the tests up allows you
to identify the source of the bug immediately.
\begin{verbatim}

      !------------------------------------------------------------------------
      !EX_UTest
      ! Getting Attrubute count from a Field
      call ESMF_FieldGetAttributeCount(f1, count, rc=rc)
      write(failMsg, *) "Did not return ESMF_SUCCESS"
      write(name, *) "Getting Attribute count from a Field "
      call ESMF_Test((rc.eq.ESMF_SUCCESS), name, failMsg, result, ESMF_SRCLINE)

      !------------------------------------------------------------------------
      !EX_UTest
      ! Verify Attribute Count Test
      write(failMsg, *) "Incorrect count"
      write(name, *) "Verify Attribute count from a Field "
      call ESMF_Test((count.eq.0), name, failMsg, result, ESMF_SRCLINE)

      !------------------------------------------------------------------------

\end{verbatim}


\paragraph{Analyzing unit test results }
When unit test are run, a Perl script prints out the test results as shown in
Section "Running ESMF Unit Tests" in the ESMF User's Guide. To print out the test results,
the Perl script must determine the number of unit tests in each test file and the number of
processors executing the unit test. It determines the number of tests by counting the
EX\_UTest, NEX\_UTest, EX\_UTest\_Multi\_Proc\_Only, or NEX\_UTest\_Multi\_Proc\_Only 
tags in the test source file whichever is appropriate for the test being run. 
To determine the number of processors, it counts the number of "NUMBER\_OF\_PROCESSORS" strings in the
unit test output Log file. The script then counts the number of PASS and FAIL strings in the
test Log file.
The Perl script first divides the number of PASS strings by the number of processors. If the
quotient is not a whole  number then the script concludes that the test crashed. If the quotient
is a whole number, the script then divides the number of FAIL strings by the number of processors.
The sum of the two quotients must equal the total number of tests, if not the test is marked
as crashed.

Sometimes a unit test is written expecting a subset of the processors to fail the test. To
handle this case, the unit test must verify results from each processor as in the unit test below:
\begin{verbatim}


    !------------------------------------------------------------------------
    !EX_UTest
    ! Verify that the rc is correct on all pets.
    write(failMsg, *) "Did not return FAILURE  on PET 1, SUCCESS otherwise"
    write(name, *) "Verify rc of a Gridded Component Test"
    if (localPet==1) then
      call ESMF_Test((rc.eq.ESMF_FAILURE), name, failMsg, result, ESMF_SRCLINE)
    else
      call ESMF_Test((rc.eq.ESMF_SUCCESS), name, failMsg, result, ESMF_SRCLINE)
    endif

    !------------------------------------------------------------------------

\end{verbatim}

\subsubsection{System Tests}

System tests are written to test functionality that spans several 
classes.  The following areas should be addressed in system testing.

\begin{itemize}
\item Design omissions (e.g. incomplete or incorrect behaviors).
\item Associations between objects (e.g. fields, grids, bundles).
\item Control and infrastructure. (e.g. couplers, time management, error handling).
\item Feature interactions or side effects when multiple features are used
simultaneously.
\end{itemize}

The system tester should issue a test log after each software release is tested,
which is recorded on the {\bf Test and Validation} web page. The test 
log shall
include: a test ID number, a software release ID number, testing environment 
descriptions, a list of test cases executed, results, and any unexpected 
events. Bugs should be documented in the \htmladdnormallink{{\it SourceForge Bug 
Tracker}}{https://sourceforge.net/tracker/?group\_id=38089&atid=421185} and 
any bug fixes shall be validated.

\paragraph{Writing System Tests}

System tests should contain the following secctions:
\begin{itemize}
\item Create - Create Components, Couplers, Clock, Grids, States etc.
\item Register - Register Components and the initialize, run and finalize subroutines.
\item Initialize - Initialize as needed.
\item Run - Run the test.
\item Finalize - Verify results.
\item Destroy - Destory all classes.
\end{itemize}

At the end of the system it is recommended that the ESMF\_TestGlobal subroutine be used to gather
test results from all processors and print out a single PASS/FAIL message instead
of individual PASS/FAIL messages from all the processors.
After the test is written it must be documented on the ESMF Test \& Validaton
web page:

\begin{verbatim}

http://www.earthsystemmodeling.org/developers/test/system/

\end{verbatim}


\subsubsection{Use Test Cases (UTCs)}

Use Test Cases are problems of realistic size created to test the ESMF
software.  They were initiated when the ESMF team and its users saw that
often ESMF capabilities could pass simple system tests but would fail
out in the field, for real customer problems.  UTCs have realistic
processor counts, data set sizes, and grid and data array sizes.  UTCs are
listed on the {\bf Test \& Validation} page of the ESMF website.  They
are not distributed with the ESMF software; instead they are stored in
a separate module in the main repository called {\tt use\_test\_cases}.

\subsubsection{Beta Testing}

ESMF software is released in a beta form, as an Internal Release,
three months before it is publicly released.  This gives users
a chance to test the software and report back any problems to 
support.

\subsubsection{Automated Regression Tests}

The purpose of regression testing is to reveal faults caused by new
or modified software (e.g. side effects, incompatibility between 
releases, and bad bug fixes).  
Regression tests regularly exercise all interfaces of the code on 
all target platforms.  The Integrator runs 
automated nightly builds on a wide variety of machines, listed on the
{\bf Test and Validation} website. This site stores log files of the
build results.
The log files are kept on a monthly basis in 
separate directories. For example, the December 2002 directory is named {\tt 0212\_test}. The 
directory contains an {\tt ESMFdailyLog} file which has a one line entry for each platform on 
which the build was attempted indicating whether the build was successful or not. 

The {\tt ESMFdailyLog} file for the current month is available on the {\bf Test and Validation} website.

If the build is not successful, then a file containing the build output is stored in 
the:\\
{\tt /fs/projects/css/esmf/esmf\_test/daily\_builds}\\
directory on the NCAR fileserver. The file naming convention for the build output file is:\\
{\tt build\_BOPT\_(day)(platform)ESMF\_ARCH}.\\
For example, if the build fails on longs on the 19th of the month with {\tt BOPT=g},
the file would be called:\\
{\tt build\_g\_19longslinux\_lf95}. 

Email is sent daily to the ESMF core team listing all the platforms on which the build was attempted with a {\tt PASS/FAIL} indication. 

\subsubsection{Testing for Releases}

We provide two types of tar files, the ESMF source and the shared
libraries of the supported platforms. Consequently, there are two test
procedures followed before placing the tar files on the ESMF download website. 

The {\bf Source Code Test Procedure} is followed on all the supported
platforms for the particular release.

\begin{enumerate}
\item Verify that the source code builds in both {\tt BOPT=g} and {\tt BOPT=O}.
\item Verify that  the {\tt ESMF\_COUPLED\_FLOW} demonstration executes successfully.
\item Verify that the unit tests run successfully, and that there are no {\tt NON-EXHAUSTIVE} unit tests  failures.
\item Verify that all system tests run successfully. 
\end{enumerate}

The {\bf Shared Libraries Test Procedure} is also followed on all supported
platforms for a release.

\begin{enumerate}
\item Change to the {\tt CoupledFlowEx} directory and execute {\tt gmake}. Verify that the demo runs successfully.
\item Change to the {\tt CoupledFlowSrc} directory and execute {\tt gmake} then {\tt gmake run}. Verify that the demo runs successfully.
\item Change to the {\tt examples} directory and execute {\tt gmake} and {\tt gmake run}. Verify that the example runs successfully.
\end{enumerate}














