%===============================================================================
% CVS $Id: cm.tex,v 1.1 2005/05/05 22:55:21 nscollins Exp $
% CVS $Source: /mnt/twixshare/Storage/Archive-SF-Repos/ESMF_CVS_Repo/esmf/src/doc/dev_guide/Attic/cm.tex,v $
% CVS $Name:  $
%===============================================================================

\section{Configuration Management}
\label{sec:cm}

A central ESMF CVS repository is maintained on the 
SourceForge open source, web-based environment.  The CVS repository is 
web-accessible at: \newline
\htmladdnormallink{{\bf http://sourceforge.net/cvs/?group\_id=38089}}
{http://sourceforge.net/cvs/?group_id=38089} \\

The SourceForge site has instructions for checking out code.  Other CVS
documentation is available at: \newline
\htmladdnormallink{{\bf http://www.gnu.org/manual/cvs/html\_node/cvs\_toc.html}}{http://www.gnu.org/manual/cvs/html_node/cvs_toc.html} \\

All documents, source code, test and other scripts, and ESMF website
files are stored in the central ESMF CVS repository.  For code releases 
test scripts will be available for download from the ESMF website along 
with the framework source code.  Some data sets may be too large to store 
in the this 
repository and will be stored on the ESMF testbed system.  The ESMF
website will include a link to download these datasets once they are
available.

\subsection{Procedures for the ESMF SourceForge Repository}

\begin{description}

\item [Tagging of new releases]  A gatekeeper will be established for the 
ESMF Core Team at NCAR.  The gatekeeper will assess the 
status of ESMF code at regular intervals and will tag new 
system versions with coherent changes.  Prior to release all ESMF software 
will be regression-tested on all platforms and interfaces.  The gatekeeper 
will be responsible for regression testing, identifying problems and 
notifying the appropriate developers, and sending out release notes. 

\item [Check-in] ESMF developers will check their changes into the 
SourceForge repository as they complete them.  

We expect that developers will be able to resolve most conflicts 
at check-in themselves. If developers cannot agree or need more information,
the gatekeeper or one or more of the technical leads will make a decision
in consultation with appropriate team members.

\end{description}

\subsection{Source Code Naming and Tagging Conventions}
\label{sec:tagging}

We provide two types of releases, public and internal, with similiar tagging conventions.

\subsubsection{Public Releases}
Public releases are each given a branch created from the main CVS repository ESMF trunk. The tagging convention
for public releases is 
{\tt ESMF\_*\_*\_*r[p\#]}, e.g., {\tt ESMF\_0\_2\_1r}, where the first digit represents a major release, the 
second digit an incremental release, the third digit a routine update, and an official release {\tt r}.
Subsequent patches to the release are identified with the letter {\tt p} followed by the patch number,
e.g., {\tt ESMF\_1\_0\_0rp2}. Patches are tagged on the release branch.
After the ESMF community has had an opportunity to download and use the release, a quality assessment
of {\tt good}, {\tt fair}, or {\tt poor} will be assigned by the ESMF Core group.  The grade will 
be posted on the ESMF web site.

\subsubsection{Internal Releases}
ESMF software internal releases are identified as tags on the CVS main trunk on a monthly basis.
The tagging convention for internal releases is
{\tt ESMF\_*\_*\_*}, e.g., {\tt ESMF\_1\_0\_1}, where the first digit represents a 
major release, the second digit an incremental release, and the third digit a routine update.
Similiarly as the case with public releases, internal releases will be assigned a grade
after an evaluation period, which will be posted on the ESMF web site.

Application components are tagged following the conventions of their
home institution.

\subsection{Document Naming and Tagging Conventions}

The following labeling convention is used for ESMF document tagging:

{\tt <NAME>\_<DOC TYPE>\_<STATUS>\_<TARGET MILESTONE>\_<SUFFIX>}

\noindent Valid {\tt <NAME>} values are the abbreviations for each functionality 
class - e.g., {\tt TMG} for Time Manager.  The value {\tt ESMF} is used 
for general documents.

\noindent Valid <DOC TYPE> values are:
\begin{description}
  \item {\tt REQ} - requirements 
  \item {\tt ARCH} - architecture
  \item {\tt DES} - design
  \item {\tt REF} - reference manual
  \item {\tt USR} - user's manual
  \item {\tt POL} - policies 
\end{description}

\noindent Valid {\tt <STATUS>} values are:
\begin{description}
  \item {\tt DRAFT} - draft, prior to any review 
  \item {\tt REVIEW} - version subject to team review
  \item {\tt SIGNOFF} - stable, signoff version
  \item {\tt RELEASE} - version released for a milestone
\end{description}

\noindent {\tt <SUFFIX>}es are as follows:
\begin{description}
  \item For {\tt DRAFT} tags: {\tt <SUFFIX> = <REVNUMBER>}
  \item For {\tt REVIEW} tags: {\tt <SUFFIX> = <MAJOR REVNUMBER>\_<MINOR REVNUMBER>}
  \item For {\tt SIGNOFF} tags: {\tt <SUFFIX> = <MAJOR REVNUMBER>\_<MINOR REVNUMBER>}
  \item For {\tt RELEASE} tags: {\tt <SUFFIX> = <REVNUMBER>}

  {\tt MAJOR REVNUMBER}s begin at 1 and {\tt MINOR REVNUMBERS} begin at 0.
\end{description}

\subsection{Backups}

The ESMF SourceForge repository is backed up nightly.  The tarball
from this backup is downloaded on a weekly basis and stored on a local
machine, currently an IBM at NCAR and eventually the NASA testbed 
system.  Datasets that are not included in the CVS repository will be 
stored and backed up on the ESMF testbed and mirrored at a second site, 
likely NCAR.  We anticipate that the systems staff at NASA will perform
frequent and regular backups of data on the testbed, perhaps on a nightly 
or weekly basis.  If this is not the case other arrangements for regular
backups will be made. 

\subsection{Contact Information}

If there are any questions or concerns about the SourceForge CVS 
repository, mail should be sent to \htmladdnormallink{esmf@ucar.edu}
{mailto:esmf@ucar.edu}.







