% $Id: tracking.tex,v 1.1 2005/05/05 22:55:21 nscollins Exp $

\section{Tracking and Software Task Sequencing}
\label{sec:tracking}

Software tasks are prioritized in the {\it Build Plan} 
(see Section~\ref{sec:build}), as
a result of discussion during core team developer meetngs and
core team developer status reports, and distributed team telecons and
meetings.  The person primarily responsible for prioritizing software
tasks is the ESMF Software Engineering Manager at NCAR.  Developer
skill and integrity is the primary mechanism for assuring that
tasks are correct and complete.  Unit tests and system tests 
performed by the integrator confirm that code works as expected.

Eventually we envision the formation of a Change Review Board,
but this may not occur until after the period of NASA funding is over.

\subsection{Tools for Software Tracking}
\label{sec:tracking_tools}

Developers will use the software tracker available on the ESMF 
SourceForge site to monitor and prioritize software tasks such as 
defect fixes and requests for enhancement.  
(See \htmladdnormallink{http://www.sourceforge.net/projects/esmf}
{http://www.sourceforge.net/projects/esmf}, under the ``Bugs or Support'' links.
This tracking system is used successfully by major software 
efforts such as the
\htmladdnormallink{Python language} {http://www.sourceforge.net/projects/python}
and \htmladdnormallink{TCL}{http://www.sourceforge.net/projects/tcl}.
Both developers and deployment team members can submit tracking items.  

Each task in the tracker is associated with a category, assignee, 
group, priority, resolution and status.  These are explained 
in more detail below. 
\begin{description}
\item[Category] The functionality class associated with the task; for
example, time management.
\item[Assignee] The person who is assigned implementation or resolution
of the task.  Default assignees can be specified for each category.
\item[Group] Currently ESMF groups include Bug, Documentation,
Optimization, Port, Request for Enhancement, and Standardization, among 
others.  These may change somewhat as the ESMF project evolves. 
\item[Priority] The priority listing is from 1-9, and is visually indicated
by color intensity.  Generally, 9 indicates a problem that is causing 
system failure, and 1 indicates a cosmetic or inessential modification.
A defect that results in a priority 1 requirement or milestone not being
met shall receive a high priority, while defects that reult in less
critical requirements nor being satisfied shall receive a lower priority.  
Specific values are determined by common sense.  More reductive
associations (e.g., 8 means a priority 1 requirement is not being met) 
would be overly restrictive and unlikely to be used in practice.  
\item[Resolution] Possible resolutions include Fixed, Postponed, 
Works for me, and others - see the tracker itself, URL shown 
above, for the complete list.
\item[Status] Status options include Open, Closed, Deleted, and Pending.
\end{description}

Tasks can conveniently be sorted and viewed based on priority, status, 
category, group, or assignee.

\subsection{Metrics}

The NCAR core team integrator is responsible for tracking the
various aspects of the ESMF project to measure the implementation and
testing progress.
\subsubsection{Unit and System Tests Coverage}
Throughout the ESMF software development process, the software is constantly
being unit and system tested. It is essential to know the percentage of the
code that has been fully or partially tested. A script has been written that 
lists all the public interface subprograms (functions and subroutines) and 
determines which of these have been unit and/or system tested. From this script
output, the tested percentage can be calculated.
\subsubsection{ESMF Requirements Coverage}
An Excel spreadsheet is maintained listing the requirements as described in the
ESMF Requirements Document. A hardcopy of the spreadsheet has been posted in a
convenient place for the core team. Using color coding the core team indicates
which requirements have been implemented and of these which have been tested.
\subsubsection{Source Lines of Code, SLOC}
The ESMF SLOC information is provided in an internal ESMF webpage. The
SLOC data is presented in graph form with the following columns:
\begin{description}
\item[Fortran] 
\item[C++] 
\item[c] 
\item[Makefiles] 
\item[SLOC Total] 
\item[Lines of text] 
\end{description}

The graphs are a monthly breakdown from January 1, 2002 to the present.
























