% $Id: tracking.tex,v 1.2 2006/08/01 22:02:59 cdeluca Exp $

\section{Trackers}
\label{sec:tracking}

\subsection{Support Requests}
\label{sec:tracking_tools}

Developers use the Support Request tracker on the ESMF 
SourceForge site to monitor and prioritize user requests.
Customers write the esmf\_support@ucar.edu list and their
requests are entered into the tracker by a Core Team
member.  (See \htmladdnormallink{http://www.sourceforge.net/projects/esmf}
{http://www.sourceforge.net/projects/esmf}, under {\bf Support Requests}
on the menu bar.

The Support Request tracker categorizes requests by Assignee, 
Status (Open, Closed, etc.), Category (Array, Regrid, etc.), and
Group (Bug, Feature Request, etc.)  Requests can be sorted and
viewed based on these keys.

\subsection{Bugs}

The Bugs tracker on the main ESMF SourceForge site is used
to record bugs reports.  It works in much the same way as
the Support Request tracker.
(See \htmladdnormallink{http://www.sourceforge.net/projects/esmf}
{http://www.sourceforge.net/projects/esmf}, under {\bf Bugs}
on the menu bar.

\subsection{Metrics}

The NCAR core team integrator is responsible for tracking the
various aspects of the ESMF project to measure the implementation and
testing progress.
\subsubsection{Unit and System Tests Coverage}
Throughout the ESMF software development process, the software is constantly
being unit and system tested. It is essential to know the percentage of the
code that has been fully or partially tested. A script has been written that 
lists all the public interface subprograms (functions and subroutines) and 
determines which of these have been unit and/or system tested. From this script
output, the tested percentage can be calculated.
\subsubsection{ESMF Requirements Coverage}
An Excel spreadsheet is maintained listing the requirements as described in the
ESMF Requirements Document. A hardcopy of the spreadsheet has been posted in a
convenient place for the core team. Using color coding the core team indicates
which requirements have been implemented and of these which have been tested.
\subsubsection{Source Lines of Code, SLOC}
The ESMF SLOC information is provided in an internal ESMF webpage. The
SLOC data is presented in graph form with the following columns:
\begin{description}
\item[Fortran] 
\item[C++] 
\item[c] 
\item[Makefiles] 
\item[SLOC Total] 
\item[Lines of text] 
\end{description}

The graphs are a monthly breakdown from January 1, 2002 to the present.
























