
This part of the {\it Guide} describes a set of conventions and
procedures recommended for the Deployment Team when incorporating
ESMF into application codes and preparing interoperability 
demonstrations.

We note that the processes defined are minimal.  The Deployment Team 
consists of many large projects, funded primarily by other sources, who have 
established software conventions and procedures prior to the ESMF project.
The creation of these conventions and procedures, and their continuing 
evolution, is determined by groups of scientists and software engineers, 
in some cases numbering dozens and even hundreds, whose affiliation with
and obligation to the ESMF software effort is for the most part limited 
or nonexistent.
Software process development in such large modeling groups typically 
occurs slowly and through consensus.  It is highly unlikely that the 
ESMF can effect significant software process changes within these groups
over a short time period, and the friction generated by such an attempt
is not likely to have a positive impact on the ESMF project overall.

An equally unsatisfactory alternative is to have an application group
set ESMF adoption on an isolated development 
path, whose software process can be carefully and independently controlled.  
The obvious danger with this approach is that the modifications 
necessary for ESMF compliance never 
find their way into the real application.  We may meet our milestones, but
the effort has in reality failed.  We are determined to avoid the creation 
of ESMF-compliant 
science components which, because of foreign software conventions, an 
incompatible development cycle, or outdated science content are not 
actively used by operational or research
applications.  This can be avoided by allowing ESMF development to proceed
within the context of accepted institutional software processes, accepting
that such processes may vary widely and may on occasion not reflect
standard practices.  

{\bf It cannot be overstated how fundamental it is to the 
acceptance and success of the ESMF effort that incorporation of the 
framework into Deployment Team codes occurs within the normal development 
cycles of application groups.}  

The above said, we are also determined to deliver high-quality, 
exhaustively tested components whose build procedures and component 
interfaces are fully documented.  We are committed to providing insight 
into and descriptions of the software processes followed by our application 
groups when possible and useful.  On the following pages we describe our
strategy.





