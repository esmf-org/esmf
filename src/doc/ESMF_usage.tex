% $Id: ESMF_usage.tex,v 1.1 2001/11/13 18:35:47 dneckels Exp $

\section{Basic Usage and Conventions}

\subsection{Bindings}

The library includes both C/C++ and F90 bindings.  The C/C++ prefix for
procedures and parameters is {\tt ESMC\_} and the F90 prefix is {\tt ESMF\_}.

\subsection{Constraints}

The library design imposes some constraints on the user:

\begin{itemize}
\item {\it No direct access of Fortran derived types.}  Attributes
of Fortran derived types are private.

\item{\it All types should be initialized.} In order to provide consistent
argument checking and to increase the overall robustness of the library,
a user should call one of the {\tt ``Init''} routines before using the
library's Fortran derived types, or a {\tt ``Construct''} routine before 
using its C/C++ classes.

\end{itemize}

\subsection{Error Handling}

All C/C++ procedures return an integer error code.  All F90 procedures have 
an optional integer return code argument.  Return codes are translated 
into error descriptions using the methods: 

\begin{verbatim}
    void ESMC_ErrPrint(int rc)

    subroutine ESMF_ErrPrint(rc)  
    integer, intent(in) :: rc
\end{verbatim}

A return code of {\tt ESM[F/C]\_SUCCESS} indicates that an 
operation executed without errors.

The user can currently choose from two different error handlers.
{\tt ESM[F/C]\_ERR\_RETURN} will simply return from a routine in which an error 
is identified, without printing an error description.
{\tt ESM[F/C]\_ERR\_EXIT} will print a detailed error description including
file, function name, and line number, and will then terminate execution
(this is a simple exit, not an {\tt MPI\_ABORT}).  These handlers are set 
using the methods: 
\begin{verbatim}
    void _ErrHandlerSetType(ESM[F/C]_ErrHandlerType type)

    subroutine ESM[F/C]_ErrHandlerSetType(type)
    integer, intent(in) :: type
\end{verbatim}

The intent is to provide an error handling system in which
users can choose from a variety of handlers or supply their own.






