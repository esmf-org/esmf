% $Id: ESMF_usage.tex,v 1.9 2002/09/10 18:11:40 cdeluca Exp $

\subsection{Bindings}

The library includes both C/C++ and F90 bindings.  The C/C++ prefix
for procedures and parameters is {\tt ESMC\_} and the F90 prefix is
{\tt ESMF\_}.

\subsection{Object Function Conventions}

To enhance ease of use in the ESMF, naming conventions are
used for common methods such as creation and initialization.

\begin{itemize}
\item{\tt ESM[F/C]\_<Class>New} Allocates a {\it deep} object.  This
  function will allocate space from the heap and may create other
  resources which must be free'd or deallocated.  All items created
  with {\tt New} must be deleted with {\tt Delete} below.  {\tt e.g.
    ESMF\_LogNew}
  
\item{\tt ESM[F/C]\_<Class>Delete} De-Allocates an object created with
  New.  Any transient resources that were created will be cleaned up.
  {\tt e.g. ESMF\_LogDelete}
  
\item{\tt ESM[F/C]\_<Class>Init} Initializes a {\it shallow} object.
  This class of objects does not need to be destructed and is
  guaranteed not to allocate any resources that must be cleaned up.
  {\tt e.g. ESMF\_TimeInit}
  
\item{\tt ESM[F/C]\_<Class>Set<Value>} Sets a given value with the
  class.  The {\tt Value} parameter is decided by the class.  {\tt
    e.g. ESMF\_LogSetState}
  
\item{\tt ESM[F/C]\_<Class>Get<Value>} Gets a given value with the
  class.  The {\tt Value} parameter is decided by the class.  {\tt
    e.g. ESMF\_LogGetState}
  
\item{\tt ESM[F/C]\_<Class>SetConfig} This function takes a list of
  resorces as defined in the resource section the class.  Some class
  may not have resources and this function has no meaning.  The
  function allows a user to set multiple resources with one function
  call.  {\tt e.g. ESMF\_LogSetConfig}
  
\item{\tt ESM[F/C]\_<Class>GetConfig} This function takes a list of
  resorces as defined in the resource section the class.  Some class
  may not have resources and this function has no meaning.  The
  function allows a user to get multiple resources with one function
  call.  {\tt ESMF\_LogGetConfig}
  
\item{\tt ESM[C]\_<Class>Construct} This function fills initializes an
  object with valid data.  This function is called by both the {\tt
    Init} and {\tt New} functions.  Depending on the type of object
  this function may or may not allocate resources that need to be
  freed.
  
\item{\tt ESM[C]\_<Class>Destruct} This function cleans up any
  resources that were created in the {\tt Construct} method.

\end{itemize}

\subsection{Constraints}

The library design imposes some constraints on the user:

\begin{itemize}
\item {\it No direct access of Fortran derived types.} Elements of
  Fortran derived types are private.
  
\item{\it All types should be initialized.} In order to provide
  consistent argument checking and to increase the overall robustness
  of the library, a user should call one of the {\tt Init}
  routines before using the library's Fortran derived types, or the
  {\tt Construct} routine before using its C/C++ classes.  A {\tt
    New} routine is provided for C/C++ if dynamic memory
  allocation is desired.

\end{itemize}

\subsection{Error Handling}

All C/C++ procedures return an integer error code.  All F90 procedures have 
an optional integer return code argument (with the exception of a select few
functions that use {\tt stdargs}).  Return codes are translated 
into error descriptions using the methods: 

\begin{verbatim}
    void ESMC_ErrPrint(int rc)

    subroutine ESMF_ErrPrint(rc)  
    integer, intent(in), optional :: rc
\end{verbatim}

A return code of {\tt ESM[F/C]\_SUCCESS} indicates that an 
operation executed without errors.

The user can currently choose from two different error handlers.
{\tt ESM[F/C]\_ERR\_RETURN} will simply return from a routine in which an error 
is identified, without printing an error description.
{\tt ESM[F/C]\_ERR\_EXIT} will print a detailed error description including
file, function name, and line number, and will then terminate execution
(this is a simple exit, not an {\tt MPI\_ABORT}).  These handlers are set 
using the methods: 
\begin{verbatim}
    void _ErrHandlerSetType(ESM[F/C]_ErrHandlerType type)

    subroutine ESM[F/C]_ErrHandlerSetType(type)
    integer, intent(in) :: type
\end{verbatim}

The intent is to provide an error handling system in which
users can choose from a variety of handlers or supply their own.








