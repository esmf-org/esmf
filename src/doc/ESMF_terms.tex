
\label{sec:glos}

This glossary defines terms used in Earth system modeling to describe 
parallel computer architectures, grids and grid decompositions, and 
numerical and computational methods.  While some of the concepts in 
the glossary may eventually appear as computational objects, many 
will not.  The goal here is not to define a framework design or an 
object model but simply to achieve a common language.

\begin{description}

\item[360-day calendar] \label{glos:360DayCal} A calendar in which 
  every one of twelve months has thirty days.  See also \htmlref{Calendar}
  {glos:Calendar}, \htmlref{no-leap calendar}{glos:NoLeap}.

\item[Accumulator] \label{glos:Accumulator} A facility for collecting 
  and averaging data values.  Generally accumulators are associated with 
  temporal averaging, although they might be associated with 
  other weighted averaging operations.    
  
\item[Address space (ASP)] \label{glos:ASP} A term that refers to 
  the memory that a computer program can write to directly using
  simple language primitives. 

\item[Alarm] \label{glos:Alarm} An Alarm is an ESMF class that represents
  an event that occurs at a particular 
  time (or set of times).  It is like an alarm on a real alarm clock 
  except that in order to determine whether it is "ringing", an ESMF Alarm 
  is "read" by an explicit application action.
  See also \htmlref{Clock}{glos:Clock}.

\item[Application] \label{glos:Application} A coherent computational 
  entity run as a single executable or set of communicating executables.  
  It typically consists of a set of interacting components.  
  See also \htmlref{component}{glos:Component}.

\item[Array] \label{glos:Array} An ESMF class that represents a multi-dimensioned
  computational array.  Unlike a native Fortran or C++ array, an ESMF Array can store information about its halo.  See also \htmlref{halo}{glos:Halo}.

\item[Background grid] \label{glos:BackGrid} 
  A background grid associates each point in an observational data stream 
  (Location Stream) with a location on a grid. A single grid cell may contain 
  zero or more Location Stream points. See also \htmlref{Location Stream}{glos:LocStream}, \htmlref{cell}{glos:Cell}. 

\item[Bundle] \label{glos:Bundle} A Bundle is an ESMF class that represents 
  a set of fields that 
  are associated with the same physical grid and distributed in a similar 
  fashion across the same physical axes.  Fields within a Bundle may be
  staggered differently and may have different dimensions. 
  See also \htmlref{Field}{glos:Field}. 

\item[Calendar] \label{glos:Calendar} A Calendar is an ESMF class that 
  stores a representation of a particular calendar type, such as Gregorian.
  In this glossary we list calendars specific to modeling, 
  such as \htmlref{360-day}
  {glos:360DayCal} and \htmlref{no-leap}{glos:NoLeap}.

\item[Cell] \label{glos:Cell} A physical location that is specified by both 
  its extent (vertices) and nominal central location, and is associated with 
  a single integer index value or a set of integer index values ( e.g.
  (i) for 1-d, (i,j) for 2-d, (i,j,k) for 3d ). 
  See also \htmlref{index}{glos:Index}.

\item[Clock] \label{glos:Clock} A Clock is an ESMF class that tracks the 
  passage of time and 
  reports the current time instant, like a real clock.  However, an ESMF Clock 
  is different than a real clock in that it is generally stepped forward 
  by an ESMF component, in increments of an explicitly coded time step.
  See also \htmlref{Time}{glos:TimeInstant}, \htmlref{Time Interval}{glos:TimeInterval}.

\item[Component] \label{glos:Component} A large-scale computational entity 
  associated with a particular physical process or computational function, 
  such as a land model.  A component is also an ESMF class that represents
  a large-scale computational entity; currently ESMF supports \htmlref{Gridded 
  Component}{glos:GridComp} and \htmlref{Coupler Component}{glos:Coupler} classes.
  Components may be \htmlref{generic}{glos:GenericComp} or 
  \htmlref{user-supplied}{glos:UserComp}.  

\item[Computational domain] \label{glos:CompDomain} For a given DE, the 
  data points that have unique global indices and are updated by the local DE.  
  See also \htmlref{exclusive domain}{glos:ExcDomain}, \htmlref{total domain}
  {glos:TotDomain}, \htmlref{halo}{glos:halo}.

\item[Compute resource] \label{glos:CompResource} Something that appears as a
  physical or virtual computer resource. Example of compute resources
  are a CPU, a network connection, a communication API, a protocol, a 
  particular network fabric or a piece of computer memory. 

\item[Concurrent execution] \label{glos:ConcurrentExecution} 
  Concurrent execution of model components occurs when two components,
  whether in the same or different executables, run simultaneously.
  Components executing concurrently may have coincident or 
  non-overlapping distributions.  See also \htmlref{sequential execution}
  {glos:SequentialExecution}.

\item[Coupler Component] \label{glos:Coupler}
  An ESMF component that includes all data and actions needed to enable 
  communication between two or more Gridded Components. 
  See also \htmlref{component}{glos:Component}, \htmlref{Gridded Component}
  {glos:GridComp}.

\item[Data dependency] \label{glos:DataDep} The property of a computational
  operator that defines the data indices required to perform
  the computation at a point.  

\item[Data parallel] \label{glos:DataParallel} In a data parallel operation,
  roughly the same calculation is performed by all processors at the same 
  time on the same data set, which is partitioned among multiple memory 
  locations.  Operations within many components are essentially data 
  parallel.  See also \htmlref{task parallel}{glos:TaskParallel}, \htmlref{SPMD}{glos:SPMD}, 
  \htmlref{MPMD}{glos:MPMD}. 

\item[Data transpose] \label{glos:DataTranspose} Rearrangement of data 
  arrays 
  between two Distributed Grids sharing the same global domain.
  See also \htmlref{Distributed Grid}{glos:DistGrid}.

\item[Day of year] \label{glos:DayOfYear} The day number in the calendar year. 
  January 1 is day 1 of the year. Day of year expressed in a floating point 
  format is used to express the day number plus the time of day. 
  For example, assuming a Gregorian calendar:

\begin{tabular}{ll}
  {\bf date}              & {\bf day of year} \\
  \hline 
  10 January 2000, 6Z     & 10.25 \\
  31 December 2000, 18Z   & 366.75 
\end{tabular}

\item[DE] \label{glos:DE} 
  Short for \htmlref{Decomposition Element}{glos:Decomp_Element}.

\item[DELayout] \label{glos:DELayout} A DELayout is an ESMF class that
  defines the topology of a set of Decomposition Elements and specifies 
  how DEs are assigned to PETs in an ESMF  
  \htmlref{Virtual Machine}{glos:VM}. 

\item[Decomposition Element (DE)] \label{glos:Decomp_Element}
  A Decomposition Element is an ESMF class that represents a 
  virtual portion of a computational task.
  DELayouts assign a topology to Decomposition Elements.
  See also \htmlref{DELayout}{glos:DELayout}.

\item[Deep object] \label{glos:DeepObjects} In an environment
  in which the calling and implementation language of a library are
  different, deep objects are defined as those whose memory is 
  allocated by the implementation language. 
  See also \htmlref{shallow object}{glos:ShallowObjects}. 

\item[Distributed grid] \label{glos:DistGrid}
  A Distributed Grid is a private ESMF class that defines the 
  decomposition of a grid's global index space across a DELayout. 

\item[Distribution] \label{glos:Distribution} The function that expresses
  the relationship between the indices in a distributed grid and the elements 
  in a DELayout. See also \htmlref{distributed grid}{glos:DistGrid}, 
  \htmlref{layout}{glos:DELayout}. 

\item[Domain decomposition] \label{glos:DomainDecomp} The act of grid 
  distribution: creating a DELayout; and associating grid points with 
  the DELayout.  The dimensionality of the domain decomposition is the 
  same as the dimensionality of the associated DELayout.

\item [Exact] \label{glos:Exact} The word exact is used
  to denote entities, such as time instants and time intervals, for 
  which truncation-free arithmetic is required. 

\item[Exchange grid] \label{glos:ExchangeGrid} A grid whose vertices are
  formed by the intersection of the vertices of two overlying grids.  Each 
  cell in the exchange grid overlies exactly one cell in each grid of the 
  exhange. See also \htmlref{grid}{glos:Grid}, \htmlref{cell}{glos:Cell}.

\item[Exchange Packets] \label{glos:EP} Exchange Packets are a private
  ESMF class that contains data in an optimal form for data transfers.

\item[Exclusive domain] \label{glos:ExcDomain} For a given DE, the 
  set of data points that are not replicated on any other DE.  See also 
  \htmlref{total domain}{glos:TotDomain},
  \htmlref{computational domain}{glos:CompDomain}, \htmlref{halo}{glos:Halo}.

\item[Executable] \label{glos:Exec} 
  A parallel program that is under independent control by the operating 
  system.
%NOTE:should this be exclusively parallel programs?

\item[Export State] \label{glos:ExportState} The data and metadata that 
  a component can make available for exchange with other components. 
  This may be data at a physical boundary (e.g land-atmosphere interface) 
  or in other cases, it might be the entire model state.  
  See also \htmlref{State}{glos:State}, \htmlref{import State}{glos:ImportState}.

\item[Field] \label{glos:Field} A Field is an ESMF class that represents
  a physical quantity
  defined within a region of space.  A Field includes a Grid 
  and any metadata necessary for a full description of the field data.
  See also \htmlref{Grid}{glos:Grid}.

\item[Framework] \label{glos:Framework} We use the term framework to 
  refer to a structured collection of software building blocks that can be used 
  and customized to develop components, assemble them into an application, and 
  run the application.

\item[Generic component] \label{glos:GenericComp} A generic component
  is one supplied by the framework.  The user is not expected to 
  customize or otherwise modify it.  ESMF does not currently contain any
  generic components.  See also \htmlref{user component}{glos:UserComp}, 
  \htmlref{component}{glos:Component}. 

\item[Generic transform] \label{glos:GenericTrans} A generic transform 
  is a operation supplied by the framework, for example, a method 
  that converts gridded data from one supported Physical Grid and/or 
  decomposition to another using a specified technique.  See also \htmlref{user 
  transform}{glos:UserTrans}.

\item[Global] \label{glos:GlobDomain}
  Global generally refers to the entire extent of a DELayout or Grid.

\item[Global reduction] \label{glos:GlobReduction} 
  Reduction operations (sum, max, min, etc.) that condense data distributed
  over a DELayout.
  See also \htmlref{global broadcast}{gloss:GlobBroadcast}.

\item[Global broadcast] \label{glos:GlobBroadcast}
  Scatter operations on data distributed over a DELayout.
  See also \htmlref{global reduction}{glos:GlobReduction}.

\item[Grid] \label{glos:Grid} The discrete division of space associated with
  a particular coordinate system.  A grid contains all Physical Grid and memory 
  organization information (via Distributed Grid and DELayout) required to manipulate 
  Fields, as well as to create and execute grid transforms. 
  See also \htmlref{physical grid}{term:PhysGrid}. 

\item[Grid metrics] \label{glos:GridMetrics} Terms relating measurements 
  in index space to physical grid quantities like distances and areas.

\item[Grid staggering] \label{glos:GridStagger} 
  A descriptor of relative locations
  of scalar and vector data on a structured grid. On different
  staggered grids, vector data may lie at cell faces or vertices,
  while scalar data may lie in the interior. The staggered locations
  are often written in a notation like $(i+\frac12,j+\frac12)$ to
  describe the offset of a corner with respect to the cell $(i,j)$.

\item[Grid topology] \label{glos:GridTopo} Description of data 
  connectivities in index space.

\item[Grid union] \label{glos:GridUnion} The formation of a new grid
  by taking the union of the vertices of two input grids.
  See also \htmlref{grid}{glos:Grid}. 

\item[Gridded Component] \label{glos:GridComp}
  An ESMF class that represents a component that is associated with one 
  or more grids.  No requirements 
  may be placed on the physical content of a Gridded Component's data or 
  on the nature of its computations. See also \htmlref{component}{glos:Component},
  \htmlref{Coupler Component}{glos:Coupler}. 

\item[Halo] \label{glos:Halo} 
  The points in the data domain outside the compuational domain. 
  See also \htmlref{local domain}{glos:CompDomain}. 

\item[Halo update] \label{glos:HaloUpdate}
  Halo points are associated with other DEs'
  compuational domains, and the halo update operation involves
  synchronization of some or all halo points with other DEs. 

\item[Import State] \label{glos:ImportState} The data and metadata 
  that a component requires from other components in order to run.  
  See also \htmlref{State}{glos:State}, \htmlref{export State}{glos:ExportState}.

\item[Index] \label{glos:Index} An integer value associated with a set
  of coordinates that describe a cell or location in physical space.

\item[Index space] \label{glos:IndexSpace} The space implied 
  by a set of indices.  An index space has a defined dimensionality and 
  connectivity.

\item[Index space location] \label{glos:IndexSpaceloc} 
  A location within index space.  A index space location may be fractional.
  See also \htmlref{physical location}{glos:PhysLoc}.

\item[Instantiate] \label{glos:Instantiate}
  To create an actual instance of a software class.  For example, each 
  variable of derived type Field in an ESMF Fortran application is an 
  instance of the Field class.

\item[Interface] \label{glos:Interface}
  Used generally to refer to a set of operations that characterize 
  the behavior of a class or a component.  Also used to refer to the
  name and argument list of a particular method.

\item[JMC] \label{glos:JMC} 
  Joint Milestone Codeset.  This is the set of climate, weather and
  data assimilation applications that will be used as ESMF testbeds 
  during the initial NASA-funded phase of framework development.

\item[Location Stream] \label{glos:LocStream} An ESMF class that represents
  a list of locations with no assumed relationship between these locations.  The
  elements of a location stream are assumed to share the same data
  items and metadata, though some elements may have blank entries for
  particular data or attributes. Location Streams are not yet implemented.
  See also \htmlref{background grid}{glos:BackGrid}.

\item[Logically rectangular grid] \label{glos:RecGrid} A grid in 
  which sequential indices are physically adjacent, and in which the 
  extent of each index is independent of the other indices.
  See also \htmlref{grid}{glos:Grid}.

\item[Loose Bundle] \label{glos:LooseBundle} A loose Bundle 
  is an ESMF Bundle object that contains fields whose data is 
  not contiguous in memory.  See also \htmlref{Bundle}{glos:Bundle},
  \htmlref{packed Bundle}.

\item[Machine model] A generic representation of the computing 
  platform architecture.

\item[Mask] \label{glos:Mask} A field marking a span within a larger grid.

\item[Memory domain] \label{glos:MemDomain} The portion of memory 
  associated with a local domain.  The memory domain is always at least 
  as large as the local domain.

\item[MPMD] \label{glos:MPMD} Multiple Program Multiple Datastream.
  Multiple executables, any of which could itself be an SPMD
  executable, executing independently within an application. 
  See also \htmlref{SPMD}{glos:SPMD}

\item[Node] \label{glos:Node} A node is a set of computational resources
  that is typically located in close proximity on a computing platform
  and that is associated with a single shared memory buffer.

\item [No-leap calendar] \label{glos:NoLeap} Every year uses the same months 
  and days per month as in a non-leap year of a Gregorian calendar.  See
  also \htmlref{Calendar}{glos:Calendar}, \htmlref{360-day calendar}{glos:360DayCal}.

\item[Packed Bundle] \label{glos:PackedBundle} A packed Bundle is an 
  ESMF Bundle object that contains
  a data buffer with field data arranged contiguously in memory. See 
  also \htmlref{Bundle}{glos:Bundle}, \htmlref{loose Bundle}{glos:LooseBundle}.

\item[PE] \label{glos:PE} Short for \htmlref{Processing Element}{glos:Process_Element}.

\item[PET] \label{glos:PET} Short for 
  \htmlref{Persistent Execution Thread}{glos:PermET}.

\item[Persistent execution thread (PET)] \label{glos:PermET} Provides a
  path for executing an instruction sequence. A PET has a lifetime at least 
  as long as the associated data objects. The PET is a key abstraction 
  used in the ESMF Virtual Machine. See also  
  \htmlref{VM}{glos:VM}.

\item[Physical grid] \label{term:PhysGrid} 
  A physical grid contains a variety of information on the location 
  in physical space and physical metrics (area, grid lengths, etc.) 
  of various grid points. See also \htmlref{Distributed Grid}{glos:DistGrid}.  

\item[Physical location] \label{glos:PhysLoc} The point in physical space 
  to which data pertain. 

\item[Platform] \label{glos:Platform} 
  The processor hardware, operating system, compiler and
  parallel library that together form a unique compilation target.

\item[Processing element (PE)] \label{glos:Processing_Element}
  A Processing Element (PE) is the smallest physical processing unit available
  on a particular hardware platform.

\item[Scheduler] \label{glos:Scheduler} An operating system component 
  that assigns system resources (processors, memory, CPU time, 
  I/O channels, etc.) to executables.

\item[Sequential execution] \label{glos:SequentialExecution}
  Sequential execution of model components describes the case in which 
  one component waits for the other to finish before it begins
  to run.  Components executing sequentially may be in the same or 
  different executables and may have conincident or non-overlapping 
  memory distributions.  See \htmlref{concurrent execution}
  {glos:ConcurrentExecution}.

\item[Shallow object] \label{glos:ShallowObjects} In an environment
  in which the calling and implementation language of a library are
  different, shallow objects are defined as those whose memory is 
  allocated by the calling language. 
  See also \htmlref{deep object}{glos:DeepObjects}.

\item[Span] \label{glos:Span} The physical extent associated with a grid.

\item[SPMD] \label{glos:SPMD} Single Program Multiple Datastream. 
  A single executable, possibly with many components (representing 
  for example the atmosphere, the ocean, land surface) executing 
  serially or concurrently. See also \htmlref{MPMD}{glos:MPMD}. 

\item [State] \label{glos:State} A State is an ESMF class that may 
  contain Arrays, Bundles, Fields, or other States.  It is used to 
  tranfers data between components.  See also \htmlref{import State}
  {glos:ImportState}, \htmlref{export State}{glos:ExportState}.

\item [System time] \label{glos:SysTime} Time spent doing system tasks 
  such as I/O or in system calls.  May also include time spent running 
  other processes on a multiprocessor system. See also \htmlref{user 
  time}{glos:UserTime}, \htmlref{wall clock time}{glos:WallClockTime}.

\item[Task parallel] \label{glos:TaskParallel}  In a task parallel operation,
  different calculations are performed by different processors at the same time
  on what are usually different data sets.  Operations on different model 
  components running within either a SPMD or MPMD application may be task 
  parallel. See also \htmlref{data parallel}{glos:DataParallel}, 
  \htmlref{SPMD}{glos:SPMD}, \htmlref{MPMD}{glos:MPMD}. 

\item [Time] \label{glos:TimeInstant}
  A Time is an ESMF class that is made up of a time and date and an 
  associated calendar. It may include a time zone.
  \emph{Jan 3rd 1999, 03:30:24.56s, UTC} is one example of a Time.
  See also \htmlref{Calendar}{glos:Calendar}.

\item [Time Interval] \label{glos:TimeInterval} A Time Interval is an
  ESMF class that represents the
  period between any two time instants, measured in units, such as days, 
  seconds, and fractions of a second.  Time intervals may be 
  negative.  The periods \emph{2 days and 10 seconds}, 
  \emph{86400 and 1/3 seconds} and \emph{31104000.75 seconds} are all 
  examples of time intervals.  
  Mathematical operations such as addition, multiplication and subdivision 
  can be applied to time intervals. See also \htmlref{time instant}{glos:TimeInstant}

\item[Total domain] \label{glos:TotDomain} For a given DE, the entirety 
  of the data points allocated, included replicated points from neighboring
  DEs.  See also \htmlref{computational domain}{glos:CompDomain}, 
  \htmlref{exclusive domain}{glos:ExcDomain}, \htmlref{halo}{glos:Halo}

\item[User component] \label{glos:UserComp} A component that is customized or
  written by the user.  All ESMF components are currently user components.
  See also \htmlref{generic component}{glos:GenericComp}.

\item[User time] \label{glos:UserTime} Processor time actually spent executing 
  a PET's code. See also \htmlref{system time}{glos:SysTime}, 
  \htmlref{wall clock time}{glos:WallClockTime}.

\item[User transform] \label{glos:UserTrans} A user-supplied 
  method that is used to extend framework capabilities beyond generic 
  transforms. See also \htmlref{generic transform}{glos:GenericTrans}. 

\item[VM] \label{glos:VM} Short for 
  \htmlref{Virtual Machine}{glos:VMachine}.

\item[Virtual Machine (VM)] \label{glos:VMachine} An ESMF class that 
  abstracts hardware and 
  operation system details. The VM's responsibilities are resource management
  and topological description of the underlying compute resources in terms of 
  \htmlref{PETs}{glos:PET}. In addition the VM provides a transparent, low level
  communication API. 

\item [Wall clock time] \label{WallClockTime} Elapsed real-world time 
  (i.e. difference between start time minus stop time).
  See also \htmlref{system time}{glos:SysTime}, \htmlref{user time}{glos:UserTime}.

\end{description}








































