% $Id: ESMF_superoverview.tex,v 1.15 2004/04/23 13:18:15 cdeluca Exp $
%
% Earth System Modeling Framework
% Copyright 2002-2003, University Corporation for Atmospheric Research, 
% Massachusetts Institute of Technology, Geophysical Fluid Dynamics 
% Laboratory, University of Michigan, National Centers for Environmental 
% Prediction, Los Alamos National Laboratory, Argonne National Laboratory, 
% NASA Goddard Space Flight Center.
% Licensed under the GPL.

\section{Overview of Superstructure}

ESMF superstructure classes define an architecture for assembling
Earth system applications from modeling {\bf components}.  A component
may be defined in terms of the physical domain that it represents,
such as an atmosphere or sea ice model.  It may also be defined in terms
of a computational function, such as a data assimilation system.
Earth system research often requires that such components be {\bf coupled} 
together to create an application.  By coupling we mean the data 
transformations and, on parallel computing systems, data transfers, 
that are necessary to allow data from one component to be utilized by 
another.  ESMF offers regridding methods and other tools to simplify 
the organization and execution of inter-component data exchanges.  

In addition to components defined at the level of major physical 
domains and computational functions, components may be defined that 
represent smaller computational functions within larger components, 
such as the transformation of data between the physics and dynamics 
in a spectral atmosphere model, 
or the creation of nested higher resolution regions 
within a coarser grid.  The objective is to couple components at varying 
scales both flexibly and efficiently.  ESMF encourages a hierachical
application structure, in which large components branch into 
smaller sub-components (see Figure \ref{fig:tree}).  ESMF also makes 
it easier for the same component to be used in multiple contexts 
without changes to its source code.

\begin{center}  
\begin{tabular}{|p{6in}|}
\hline
\vspace{.01in}
{\bf Key Features} \\[.01in]
Modular, component-based architecture. \\
Hierarchical assembly of components into applications.\\
Use of components in multiple contexts without modification.\\
Sequential or concurrent component execution.\\
Single program, multiple datastream (SPMD) applications for 
maximum portability and reconfigurability.\\[.03in] \hline
\end{tabular}
\end{center}

\begin{center}
\begin{figure}
\caption{ESMF enables applications such as a seasonal forecast
model to be structured hierarchically, and reconfigured and 
extended easily.}
\label{fig:tree}
\scalebox{1.0}{\includegraphics{ESMF_tree.eps}}
\end{figure}
\end{center}

\subsection{Superstructure Classes}

There are a small number of classes in the ESMF superstructure:

\begin{itemize}
\item {\bf Component}  An ESMF component has two parts, one that is 
supplied by the ESMF and one that is supplied by the user.  The
part that is supplied by the framework is an ESMF derived type that
is either a Gridded Component ({\bf GridComp}) or a Coupler 
Component ({\bf CplComp}).  A Gridded Component typically represents
a physical domain in which data is associated with one or more 
grids - for example, a sea ice model.  A Coupler Component 
arranges and executes data transformations and transfers between
one or more Gridded Components. Gridded Components and Coupler 
Components have standard methods, which include initialize, run,
and finalize.  These methods can be multi-phase.

The second part of an ESMF Component is user code, such as a
model or data assimilation system.  Users set entry points 
within their code so that it is callable by the framework.  
In practice, setting entry points means that within user code 
there are calls to ESMF methods that associate the name of a 
Fortran subroutine with a corresponding standard ESMF operation.  
For example, a user-written initialization routine called 
{\tt popOceanInit} might be associated with the standard 
Initialize routine of an ESMF Gridded Component named ``POP'' 
that represents an ocean model.

\item {\bf State}  ESMF components exchange information with other 
components only through States.  A State is an ESMF derived
type that can contain Fields, Bundles, Arrays, and other
States.  A Gridded Component  is associated with two States, an 
{\bf Import State} and an {\bf Export State}.  Its Import State 
holds the data that it receives from other Gridded Components.  
Its Export State contains data that it can make available to 
other Gridded Components. 

\item {\bf Application Driver} The Application Driver ({\bf AppDriver}) 
is a small, generic driver program that contains the ``main'' 
routine for an ESMF application.

\end{itemize}

An ESMF coupled application typically involves an AppDriver, a parent 
Gridded Component, two or more child Gridded Components that require 
an inter-component data exchange, and one or more Coupler 
Components. 

The parent Gridded Component is responsible for creating the child 
Gridded Components that are exchanging data and creating the Coupler, 
for creating the necessary Import and Export States, and for 
setting up the desired sequencing.  The AppDriver ``main'' routine
calls the parent Gridded Component's Initialize, Run, and Finalize 
methods in order to execute the application.  For each of these
standard methods, the parent Gridded Component in turn calls the 
corresponding methods in the child Gridded Components and the 
Coupler Component.  For example, consider a simple coupled 
ocean/atmosphere simulation.  When the initialize method of the 
parent Gridded Component is called by the AppDriver, it in turn 
calls the initialize methods of its child atmosphere and ocean 
Gridded Components, and the initialize method of an 
ocean-to-atmosphere Coupler Component.  Figure \ref{fig:appunit}
shows this schematically for a coupled hurricane model with ocean
and atmosphere components.

\begin{center}
\begin{figure}
\caption{A call to a standard ESMF initialize (run, finalize) method
by a parent component triggers calls to initialize (run, finalize) 
all of its child components.}
\label{fig:appunit}
\scalebox{1.0}{\includegraphics{ESMF_appunit.eps}}
\end{figure}
\end{center}

\subsection{Distribution and Scoping of Components}
\label{sec:scoping}

Components are distributed across Distribution Elements, or {\bf DEs}.
A DE represents a piece of a decomposition.  A DELayout is a collection
of DEs with some associated connectivity that describes a specific 
distribution.  For example, the distribution of a grid divided 
into four segments in the x-dimension would be expressed in ESMF as
a DELayout with four DEs lying along an x-axis. On parallel computing
systems, a DE is often associated with a distinct computational resource, 
such as a processor or a Posix thread.  

All data transfers within an ESMF application occur {\it within} a 
component.  For example, a Gridded Component may contain halo updates.
Another example is that a Coupler Component may contain a regridding 
and data redistribution between two Gridded Components.  As a result, 
the architecture of ESMF does not depend on any particular data 
communication mechanism, and new communication schemes can be 
introduced without affecting the overall structure of the application.

Since all data communication happens within a component, a Coupler 
Component must be created on the union of the DEs of all
the Gridded Components that it couples.  

A Gridded Component may exist across all the DEs in an application.  
A Gridded Component may also reside on a subset of DEs in an 
application.  These DEs may wholly coincide with, be wholly contained 
within, or wholly contain another component.  

When a set of Gridded  Components and a Coupler runs in sequence 
on the same set of DEs the application is executing in a {\bf sequential} 
mode. When Gridded Components are created and run on mutually exclusive
sets of DEs, and are coupled by a Coupler Component that extends over
the union of these sets, the mode of execution is {\bf concurrent}.

It is possible for ESMF applications to contain some component sets
that are executing sequentially and others that are executing concurrently.
We might have, for example, atmosphere and land components created 
on the same subset of DEs, ocean and sea ice components created on 
the remainder of DEs, and a Coupler created across all the DEs in
the application.

\subsection{Integrating ESMF into Applications}

Depending on the requirements of the application, the user may 
want to begin integrating ESMF in either a top-down or bottom-up 
manner.  In the top-down approach, tools at the superstructure 
level are used to help reorganize and structure the interactions
among large-scale components in the application.  It is appropriate
when interoperability is a primary concern; for example, when 
several different versions or implementations of components are going 
to be swapped in, or a particular component is going to be used 
in multiple contexts.  Another reason for deciding on a top-down 
approach is that the application contains legacy code that for 
some reason (e.g., very large, difficult to work with, 
highly performance-tuned, resource limitations) there is little 
motivation to fully restructure.  The superstructure can be 
incorporated into such applications in a way that is non-intrusive.

In the bottom-up approach, the user selects desired utilities 
(data communications, calendar management, performance profiling,
logging and error handling, etc.) from the ESMF infrastructure 
and either writes new code using them, introduces them into 
existing code, or replaces the functionality in existing code 
with them.  This makes sense when there is a specific need for 
some functionality, like robust data acommunications, or 
when the component writer is starting from scratch.

\subsubsection{Using the ESMF Superstructure}

The following is a typical set of steps involved in adopting
the ESMF superstructure.  The first two tasks, which occur 
before an ESMF call is ever made, have the potential to be 
the most difficult and time-consuming.  They are the work 
of splitting an application into components and ensuring that
each component has well-defined stages of execution.\footnote{ESMF
aside, this sort of code structure helps to promote application
clarity and maintainability, and the effort put into it is likely
to be a good investment in any case.}

\begin{enumerate}

\item Decide how to organize the application as discrete Gridded 
and Coupler Components.  The developer might need to reorganize code
so that individual components are cleanly separated and their 
interactions consist of a minimal number of data exchanges.

\item Divide the code for each component into initialize, run, and
finalize methods.  These methods can be multi-phase, e.g., 
{\tt init\_1, init\_2}.

\item Pack any data that will be transferred between components
into ESMF Import and Export State data structures.  The user must 
describe the distribution of grids over resources on a parallel
computer via the VM and DELayout.

\item Pack time information into ESMF time management data 
structures.

\item Using code templates provided in the ESMF distribution, create
ESMF Gridded and Coupler Components to represent each component
in the user code.

\item Write a {\tt SetServices} routine that sets ESMF entry 
points for each user component's initialize, run, and finalize 
methods.

\item Run the application using an ESMF Application Driver.

\end{enumerate} 

\subsubsection{Using the ESMF Infrastructure}

Adoption of infrastructure utilities and data structures can
follow many different paths.  The calendar management utility 
is a popular place to start, since there is enough functionality
in the ESMF time manager to merit the effort required to 
integrate it into codes and bundle it with an application.


\subsection{Performance}
\label{sec:performance}

The ESMF design enables the user to configure ESMF
applications so that data is transferred directly from one component 
to another, without requiring that it be copied or sent to a different data
buffer as an interim step.  This is likely to be the most efficient way 
of performing inter-component coupling.  However, if desired, an 
application can also be configured so that data from a source component 
is sent to a distinct set of Coupler Component DEs for processing 
before being sent to its destination.

The ability to overlap computation with communication is essential for
performance.  When running with ESMF the user can initiate data 
sends during Gridded Component execution, as soon as the data is ready.
Computations can then proceed simultaneously with the data transfer.

\newpage
\subsection{Object Model}

The following is a simplified UML diagram showing the relationships among
ESMF superstructure classes.  See Appendix A, {\it A Brief Introduction 
to UML}, for a translation table that lists the symbols in the diagram 
and their meaning.

\begin{center}
\includegraphics{Comp_obj.eps}   
\end{center}



