\section{Introduction}
%test
The ESMF is a complex and ambitious 
project.   The scope of Earth science modeling is so broad, and the 
term ``framework'' so ambiguous, that the ESMF name alone provides 
little clarification as to what the ESMF actually {\it is}.  In 
this first part of the document, we set about addressing the first wave 
of questions that typically accompany an introduction to the ESMF project.  
What will the 
ESMF do for the modeling community?  What kinds of functionality will the ESMF 
include?  What sorts of components will it couple?  Is it associated with 
a standardization effort and if so which interfaces will be 
standardized?  What happens when the initial funding period is over?  
It so happens that these questions can be largely addressed by
describing three types of ``general requirements'': an {\it overall vision}
for the project; a {\it statement of the scope} of the project; and a {\it formal listing of requirements} that apply to the whole body of ESMF sofware.

The first type of requirement, itemized in the statement of project
vision, is sometimes called a ``business requirement.'' \cite{wiegers}
It describes those objectives, however hazy and unquantifiable, whose
attainment will ultimately lead the Earth System Community to assess
the ESMF as a success or failure.  The second type of requirement, 
discussed in the statement of project scope, describes the
infrastructure requirements that the ESMF will fulfil.  Here we
describe the functionality of the ESMF, and specify what capabilities
the ESMF will {\it not} 
\footnote{ Items that are {\it not}
included may be added in follow-up development efforts.}
include.
Finally, we list a set of functional
requirements -- some quite specific -- that apply to all ESMF
software.  These will be referenced in specific requirements 
sections. 





