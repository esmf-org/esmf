\section{ESMF\_RegridWeightGen}
\label{sec:ESMF_RegridWeightGen}

\subsection{Description}

The ESMF\_RegridWeightGen application is used to generate regridding weights in an offline mode.
This program combines the two offline regrid applications in the previous release 
\ref{sec:regrid:offline} into one application.  It takes a source grid file and a destination grid file
in either the SCRIP format or the ESMF Unstructured grid format, and generate a regridding weight 
file in the SCRIP format.  It supports bilinear, patch and conservative regridding methods.  
The source and the destination grids can be either a regularly rectangular grid or an unstructured grid
including cubed sphere grid.  This application is a parallel application that can be run on any number of PETs.

The command line arguments are all keyword based.  Both the long keyward prefixed with {\tt '--'} or the 
one character short keyword prefixed with {\tt '-'} are supported.  The format to run the application is 
as follows:

\begin{verbatim}
ESMF_RegridWeightGen  [--source|-s] src_grid_filename 
                      [--destination|-d] dst_grid_filename 
                      [--weight|-w] out_weight_file 
                      [--method|-m] [bilinear|patch|conservative] 
                      [--pole|-p] [all|none|1|2|..] 
                      --src_type [SCRIP|ESMF] 
                      --dst_type [SCRIP|ESMF]
                      -t [SCRIP|ESMF]

where
  --source or -s -- a required argument specifying the source grid file name

  --destination or -d - a required argument specifying the destination grid file name

  --weight or -w - a required argument specifying the output regridding weight file name

  --method or -m - an optional argument specifying which interpolation method is used.  The value
                   can be one of the following:

                   bilinear     - for bilinear interpolation, also the default method if not specified.
                   patch        - for patch recovery interpolation
                   conservative - for first order conservative interpolation

   --pole or -p - an optional argument indicating what to do with the pole.  The value can be one of
                  the following:

                  none - No pole, the source grid ends at the top (and bottom) row of 
                        nodes specified in <source grid>.
                  all  - Construct an artificial pole placed in the center of the 
                        top (or bottom) row of nodes, but projected onto the sphere 
                        formed by the rest of the grid. The value at this pole is the 
                        average of all the pole values. This is the default option.
                  <N>  - Construct an artificial pole placed in the center of the 
                        top (or bottom) row of nodes, but projected onto the sphere 
                        formed by the rest of the grid. The value at this pole is the 
                        average of the N source nodes next to the pole and surrounding
                        the destination point (i.e. the value may differ for each
                        destination point. Here N ranges from 1 to the number of nodes 
                        around the pole. 

    --src_type - an optional argument specifying the source grid file type.  The value could be either
                 SCRIP or ESMF.  Currently, the ESMF file type is only available for the unstructured 
                 grid. The default option is SCRIP.

    --dst_type - an optional argument specifying the destination grid file type.  The value could be 
                 either SCRIP or ESMF.  Currently, the ESMF file type is only available for the
                 unstructured grid. The default option is SCRIP.

    -t         - an optional argument specifying the file types for both the source and the destination
                 grid files.  The default option is SCRIP.  If both -t and --src_type or --dst_type
                 are given at the same time and they disagree with each other, an error message will
                 be generated.
\end{verbatim}

In this application, we use {\tt ESMF\_GridCreate}~\ref{example:2DLogRecFromScrip} to create a 
ESMF_Grid object if the source or destination grid is a logically rectangular grid.  We use 
{\tt ESMF\_MeshCreate}~\ref{sec:example:UnstructFromFile} to create a ESMF_Mesh object if the 
source or destination grid is a cubed sphere grid or an unstructured grid.  We then use
{\tt ESMF_FieldRegridStore} to generate the regridding weight table and indicies table.   
