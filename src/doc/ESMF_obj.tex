\section{Objectives}


For any software system, there are multiple levels of requirements.
The great bulk of our requirements will consist of the detailed
specification of functional capabilities.  There is, however,
another essential type of requirement to which we must attend.  
These project objectives describe the desired impact of the
ESMF on the Earth system modeling community.  Although the objectives 
are non-quantititive and difficult to verify, they still must be 
satisfied to consider the ESMF project truly a success.  It
is essential to allow them to influence project execution, since
it will be possible to satisfy detailed functional
requirements but fail to achieve the underlying objectives of 
our effort.

\newpage
These are the objectives of the ESMF project:

\begin{itemize}

\item Facilitate the exchange of scientific codes
(interoperability) so that researchers may more easily take advantage
of the wealth of resources that are available in the US in
smaller-scale, process modeling and to more easily share experience
among diverse large-scale modeling and data assimilation efforts. 

\item Promote the reuse of standard, non-scientific software, the 
development of which now accounts for a substantial fraction of the
software development budgets of large groups. 
Any center developing or maintaining a large system for NWP, climate or
seasonal prediction, data assimilation, or basic research will have to
solve very similar software engineering and routine computational
problems.  

\item Focus community resources to deal with architectural changes
and the lack of quality commodity middleware. The non-scientific parts
of the codes that would be dealt with in a common framework are also the
most sensitive to architectural changes and ``middleware'' quality. 

\item Present the computer industry with a unified, well defined
and well documented task for them to address in their software design.
The scientific community's influence with the industry
is much diminished, but it will be even smaller if it is exercised
separately by five or six centers. 

\item Share the overhead costs of the housekeeping aspects of
software development: documentation and configuration management.  These
are the efforts that are most easily neglected when corners have to be
cut. 

\item Provide institutional continuity to model and data assimilation
 development efforts.  Most US modeling and data assimilation efforts
 are necessarily tied to only a few individuals, and centers are
 hard-pressed to maintain continuity that transcends them.  The
 competitive job market we are in will result in shorter tenure for
 programmers. At the same time, the increasing complexity of both our
 systems and the technology will produce more reliance on software
 specialists --- and less on scientists --- to maintain these aspects
 of the systems. Both of these factors will contribute to a more
 unstable workforce and much greater difficulty in maintaining
 ``institutional continuity.'' A framework can help us do this by
 having a much larger institution to support it --- the whole
 community.  Also by having codes depend on a common, well-known
 framework, it will be easier to find and train new people to continue
 a line of development.

\end{itemize}






