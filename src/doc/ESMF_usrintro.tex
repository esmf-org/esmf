\section{Introduction to the Earth System Modeling Framework}

The Earth System Modeling Framework (ESMF) is a structured collection of software 
building blocks that can be used or customized to develop model components, assemble 
them into an application, and run the application.  The functional split between developing 
and combining components is the foremost feature of the ESMF architecture.  The 
simplest view of the ESMF is that it consists of a layered {\it infrastructure} of 
utilities and data structures for building model components and a {\it superstructure} 
for coupling them.  

This {\it ESMF User Guide} will eventually serve as an introduction for the new ESMF 
user and as a reference for the experienced user.  Since ESMF Release 1.0 is the first 
public ESMF release, and is a prototype rather than a production-ready package, 
we will assume you are a 
potential user interested in learning more about the ESMF software.  This edition
of the {\it User Guide} is designed to guide you through that process.  It contains 
a {\it Quick Start} section that explains how to install the ESMF software and 
run examples, and an architectural overview that describes the framework's basic 
goals and features.  To help you become familiar with ESMF terminology, the 
{\it User Guide} also includes a glossary.

The purpose of ESMF Release 1.0 is to provide a first look at the ESMF
Application Programming Interface (API), and to demonstrate the viability 
of the ESMF architecture and implementation.  We encourage you to run a 
demonstration program, {\tt ESMF\_COUPLED\_WAVE}, that illustrates both ESMF 
utilities and coupling services.  This program is described in section \ref{sec:demo}.  

Those curious about specific interfaces should refer to the {\it ESMF Reference 
Manual for Fortran90}, which contains a detailed listing and description of 
the ESMF API.

While we are delighted to have potential users experiment with ESMF, the
ESMF team can offer only limited support to those who are trying to incorporate 
the framework into applications at this early stage.  The ESMF is still
too young to be considered viable infrastructure for active research and 
operational codes.  For ESMF Release 1.0, we have focused on implementing 
major architectural features and basic functionality.  The current performance 
characteristics and memory requirements of the software are unlikely to resemble 
those in later releases.

Our next major release will occur in April 2004.  At that time, the {\it ESMF 
User Guide} will be expanded to include a comprehensive section on how to adapt 
application codes for the framework, and support staff will be available to 
assist users with ESMF adoption.  In the interim, we are relying on Earth
system modelers to provide us with the design feedback essential for creating a 
shared, easy to use, high performance software framework.  










