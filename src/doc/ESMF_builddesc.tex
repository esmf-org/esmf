% $Id: ESMF_builddesc.tex,v 1.3 2003/05/07 14:14:38 cdeluca Exp $

\subsection{Make System}
For most users the description of the build above should be sufficient.  Some
users, however, may wish to have a more detailed knowledge of the make system
that is used by the library either for configuring different build options or
other reasons.
\subsubsection{General Structure}
The main components of the make system are:
\begin{itemize}
\item{Build directories}

The {\tt build} directory contains some generic makefiles that are included by
the Makefiles in the source tree.  In addition, for each supported
configuration there is a directory which contains the makefiles defining
compilers, compiler flags, and the various other defintions that are necessary
to make each configuration work.  Examples include {\tt build/IRIX64,
build/rs6000\_sp, ...}

\item{Top level Makefile}

All of the make targets described installation directions above originate in
this Makefile.  This makefile includes the common makefiles from the {\tt
build} directory.  Many of the commands in this Makefile spawn a recursive make
through the directory structure.

\item{Source tree Makefiles}

Each directory contains a Makefile which includes the {\tt build} common
makefiles.  These local Makefiles include defintions that allow the local files
and documents to be built.
\end{itemize}

\subsubsection{Configuration}

Each configuration is defined in the {\tt build/ESMF\_ARCH} directory.  There
are several files in each configuration that may be hand edited to modify the
build:

\begin{itemize}

\item{{\tt base.site}}

This file contains variables which define the location of includes and
libraries such as MPI, openMP, BLAS, PCL, etc...

\item{{\tt base\_variables}}

Contains the basic definitions for compilers and compiler flags.

\item{{\tt conf.h}}

Lists compiler defines that are used to enable/disable certain options at
compiler time.  Examples include pthreads, omp, fortran\_underscore...

\end{itemize}

\subsection{Build Options}
\label{BuildOptions}

There is an install target which will copy the library and *.mod files to
specified directories.  To invoke this target use:
\begin{verbatim}
  gmake BOPT=[O,g] ESMF_LIB_INSTALL=<path for library> ESMF_MOD_INSTALL=<path for *.mod files> install 
\end{verbatim}

Some users may wish for the library to be built in a directory different from 
where the source code resides.  To do this, build using:
\begin{verbatim}
   gmake ESMF_BUILD=<path> BOPT=[O,g]
\end{verbatim}

The {\tt ESMF\_BUILD} variable gives an alternate path in which to place the libraries,
*.mod files and object files.  This variable defaults to {\tt ESMF\_DIR}.  If it is 
assigned another value, the {\tt ESMF\_BUILD} variable will need to be passed as
an additional argument to the the above make commands.  (Alternatively the variable
{\tt ESMF\_BUILD} can be set in the environment (using setenv or export) and then it 
need not be passed to any make calls).








