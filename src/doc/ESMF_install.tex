% $Id: ESMF_install.tex,v 1.1 2001/11/13 18:35:47 dneckels Exp $

\section{Installation}

Currently the following environment variables need to be set:
\begin{verbatim}
  ESMF_DIR      top-level ESMF directory
  ESMF_ARCH     platform and compiler configuration
\end{verbatim}

\noindent The following configurations are supported:

\begin{tabular}{lll}
{\tt ESMF\_ARCH}  & {\tt alpha}      &  OSF1, Native compilers. \\
                  & {\tt IRIX64}     &  IRIX64, MIPSpro/mpt 64 bit compilers. \\
                  & {\tt IRIX}       &  IRIX64, MIPSpro/mpt n32 abi compilers. \\
                  & {\tt rs6000\_sp}  &  AIX, mpxlf90\_r, mpcc\_r, and mpCC\_r.  \\
                  & {\tt linux\_gnupgf90} & Linux, pgf90, gcc and g++.  \\
                  & {\tt linux\_pgi}  &  Linux, pgf90, pgcc, pgCC. \\
                  & {\tt linux\_lf95} &  Linux, lf95, gcc, g++. \\
                  & {\tt sun}        &  SunOs, SUNWhpc compilers. \\
\end{tabular}

\smallskip

The library requires {\tt gmake} to build.  Simultaneous multiple architecture builds are supported, with
one restriction; the test cases may only be run on one platform at a time. 

\smallskip

\noindent Build the library with the command:
\begin{verbatim}
  gmake BOPT=g  
\end{verbatim}
  for a debug version or
\begin{verbatim}
  gmake BOPT=O  
\end{verbatim}
  for an optimized version.

\noindent To build and run C tests:
\begin{verbatim}
  gmake BOPT=g test_c
\end{verbatim}

\noindent To build and run F90 tests:
\begin{verbatim}
  gmake BOPT=g test_f90
\end{verbatim}

Output files from the test examples will be directed to files 
in {\tt */examples/output} directories. \\

\noindent To build documentation:
\begin{verbatim}
  gmake doc
\end{verbatim}

Please note that in order to run the tests the standard
environment variables for the platform must be set by the user; 
e.g., {\tt MP\_NODES, MP\_TASKS\_PER\_NODE} for an IBM SP.  It's suggested
that the user run the tests as a single task, since output from
multiple tasks running the same test suite currently spills into a single 
file.

To use the library from C/C++, link with the library executable and include
the {\tt ``ESMC.h''} file.

To use the library from Fortran, link with the library executable and
create links to the library modules {\tt esmf\_timemgmtmod.mod}
and {\tt esmf\_timemgmttypesmod.mod} in your build directory.  These are
in the top level {\tt mod} directory under the appropriate architecture.
To include the library in application modules, {\tt USE} the
module {\tt ESMF\_TimeMgmtMod}.  






