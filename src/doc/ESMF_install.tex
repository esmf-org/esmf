% $Id: ESMF_install.tex,v 1.27 2003/05/30 21:18:37 flanigan Exp $

\subsection{ESMF Download Options}

Major releases of the ESMF software can be downloaded by following
the instructions on the 
the {\bf Downloads \& Documentation} page on the ESMF 
website, \htmladdnormallink{http://www.esmf.ucar.edu}{http://www.esmf.ucar.edu}.  There are two options for using the ESMF:

\begin{itemize}
\item Download a pre-built ESMF shared object library and
test applications for a particular platform.  If you choose
this approach, you can skip ahead to Section~\ref{UsingLibrary},
Using the ESMF.  
\item Download the full ESMF source code, and compile and link
the framework to any necessary system libraries.  This will
result in a shared object file (with a *.so extension)
that can be linked in with the user's code, or with the demo
{ESMF\_COUPLED\_FLOW} executable.  In this case you will need
to follow all the instructions in subsequent sections of the 
{\it Quick Start} guide, beginning with Section~\ref{InstallProcedures},
Installation.
\end{itemize}

You may find it necessary to build the ESMF yourself
if we do not offer a shared object library for the current
version of your compiler.  The compiler versions that we offer
shared objects for are noted on the download web page.

\subsection{Installation}
\label{InstallProcedures}

% $Id: ESMF_systemreq.tex,v 1.2 2004/06/22 14:17:44 nscollins Exp $

\subsubsection{System Requirements}
\label{sec:systemreq}

The following compilers and utilities are required for compiling and 
linking the ESMF software:
\begin{itemize}
\item a Fortran90 compiler and libraries;
\item a C++ compiler;
\item a MPI implementation compatible with these compilers (but see below);
\item the \htmladdnormallink{GNU make}{http://www.gnu.org/software/make/make.html} utility; 
\item the tar utility, for unpacking data files;
\item the \htmladdnormallink{GNU zip}{http://www.gnu.org/software/gzip/gzip.html} utility, for unpacking data files.
\end{itemize} 

An alternative to the MPI library is provided with the ESMF,
a single-process MPI-bypass library.  It allows applications which
use MPI to be linked but only run single process.

In order to build html and pdf version of the ESMF documentation, 
\LaTeX, the latex2html conversion utility, and the dvipdf 
utility must be installed.









\subsubsection{ESMF Environment Variables}

The following environment variables must be set:
\begin{verbatim}
  ESMF_DIR      top-level ESMF directory
  ESMF_ARCH     platform and compiler configuration
\end{verbatim}

On Alpha machines an additional environment variable needs
to be set:

\begin{verbatim}
  ESMF_PROJECT  Load Sharing Facility (LSF) project name 
\end{verbatim}

On an Alpha machine, test and demo applications are run using 
the bsub command.  The value of ESMF\_PROJECT is used as the 
argument for bsub's -P option. The -P option assigns a job to 
a specific project.  

Environment variables must be set in the user's shell and not
inside an ESMF makefile or build system file.  Here is an example 
of setting an environment variable in tcsh and csh shells:
\begin{verbatim}
  setenv ESMF_ARCH rs6000_sp
\end{verbatim}
In ksh shell environment variables are set this way:
\begin{verbatim}
  export ESMF_ARCH=rs6000_sp
\end{verbatim}


\subsubsection{Supported Platforms}
% $Id$

% List of architectures supported.  This file is 
% meant to be included in a user doc.

The following two tables list various combinations of environment 
variable settings used by the ESMF build system. A {\tt default}
value in the compiler column indicates the vendor compiler. A {\tt mpi}
value in the comm column indicates the vendor MPI implementation.

The first table lists the exact combinations which are tested regularly and are
fully supported. The second table lists all possible combinations which are 
included in the build system.

\vspace{1ex}
{\bf Fully tested combinations}: (See \htmladdnormallink{https://www.earthsystemcog.org/projects/esmf/platforms\_8\_0\_0}{https://www.earthsystemcog.org/projects/esmf/platforms\_8\_0\_0} for the most up-to-date table of supported combinations.)
\vspace{1ex}

\begin{longtable}{lllllll}
  &{\bfseries\footnotesize ESMF\_OS} &{\bfseries\footnotesize ESMF\_COMPILER} & {\bfseries\footnotesize ESMF\_COMM} & {\bfseries\footnotesize ESMF\_ABI} &
  {\bfseries\footnotesize F90 compiler} & {\bfseries\footnotesize C++ compiler} \\

%Hera 
Cray Compute   &\tt Linux  &\tt gfortran     &\tt mpiuni,            &\tt 64 & gfortran \footnotesize 4.8.4        & g++   \footnotesize 4.8.5         \\
Cluster        &           &                 &\tt intelmpi \footnotesize (2018.0.4)&       &                                     &                                   \\
Cray Compute   &\tt Linux  &\tt intel        &\tt intelmpi \footnotesize (2018.0.4)&\tt 64 & ifort    \footnotesize 18.0.5.274   & icpc  \footnotesize 18.0.5.274    \\
Cluster        &           &                 &                       &       &                                     &                                   \\
Cray Compute   &\tt Linux  &\tt pgi          &\tt mpiuni             &\tt 64 & pgf90    \footnotesize 18.10-1 	   & pgc++ \footnotesize 18.10-1       \\
Cluster        &           &                 &\tt intelmpi \footnotesize (2018.0.4)&       &                                     &                                   \\
%Cori
Cray XC30      &\tt Unicos &\tt intel        &\tt mpi \footnotesize (cray-mpich/7.7.6) &\tt 64     & ftn/ifort \footnotesize 19.0.3.199  & CC/icpc \footnotesize 19.0.3.199  \\
%Gaea
Cray XE6       &\tt Unicos &\tt gfortran     &\tt mpi \footnotesize (cray-mpich/7.7.3) &\tt 64     & ftn/gfortran \footnotesize 5.3.0    & CC/g++  \footnotesize 5.3.0       \\
Cray XE6       &\tt Unicos &\tt intel        &\tt mpi \footnotesize (cray-mpich/7.7.3) &\tt 64     & ftn/ifort \footnotesize 16.0.3.210  & CC/icpc \footnotesize 16.0.3.210  \\
Cray XE6       &\tt Unicos &\tt pgi          &\tt mpi \footnotesize (cray-mpich/7.7.3) &\tt 64     & ftn/pgf90 \footnotesize 16.5-0      & CC/pgc++\footnotesize 16.5-0      \\
%Electra
HPE/SGI ICE X  &\tt Linux  &\tt gfortran     &\tt mpiuni           &\tt 64           & gfortran \footnotesize 6.2.0        & g++ \footnotesize 6.2.0          \\
               &           &                 &\tt mpi \footnotesize (mpt/2.14r19)&                 &                                     &                                  \\
HPE/SGI ICE X  &\tt Linux  &\tt intel        &\tt mpiuni           &\tt 64           & ifort \footnotesize 15.0.3.187      & icpc \footnotesize 15.0.3.187    \\
               &           &                 &\tt mpi \footnotesize (mpt/2.12r26)&                 &                                     &                                  \\
HPE/SGI ICE X  &\tt Linux  &\tt pgi          &\tt mpiuni           &\tt 64           & pgf90 \footnotesize 17.1-0          & pgc++ \footnotesize 17.1-0       \\
%Pleiades
HPE/SGI ICE X  &\tt Linux  &\tt gfortran     &\tt mpiuni           &\tt 64           & gfortran \footnotesize 6.2.0        & g++ \footnotesize 6.2.0          \\
               &           &                 &\tt mpi \footnotesize (mpt/2.14r19)&                 &                                     &                                  \\
HPE/SGI ICE X  &\tt Linux  &\tt intel        &\tt mpiuni           &\tt 64           & ifort \footnotesize 18.0.3.222      & icpc \footnotesize 18.0.3.222    \\
               &           &                 &\tt mpi \footnotesize (mpt/2.15r20)&                 &                                     &                                  \\
HPE/SGI ICE X  &\tt Linux  &\tt pgi          &\tt mpiuni           &\tt 64           & pgf90 \footnotesize 17.1-0          & pgc++ \footnotesize 17.1-0       \\
               &           &                 &\tt mpi \footnotesize (mpt/2.17r13)&                 &                                     &                                  \\
%Cheyenne
HPE/SGI ICE XA &\tt Linux  &\tt gfortran     &\tt mpich3 \footnotesize (3.2)     &\tt 64           & gfortran \footnotesize 6.3.0        & g++ \footnotesize 6.3.0          \\
Cluster        &           &                 &                     &                 &                                     &                                  \\
HPE/SGI ICE XA &\tt Linux  &\tt gfortran     &\tt mpich3 \footnotesize (3.2)     &\tt 64           & gfortran \footnotesize 7.2.0        & g++ \footnotesize 7.2.0          \\
Cluster        &           &                 &                     &                 &                                     &                                  \\
HPE/SGI ICE XA &\tt Linux  &\tt gfortran     &\tt openmpi \footnotesize (3.1.0)  &\tt 64           & gfortran \footnotesize 8.1.0        & g++ \footnotesize 8.1.0          \\
Cluster        &           &                 &                     &                 &                                     &                                  \\
HPE/SGI ICE XA &\tt Linux  &\tt gfortran     &\tt mpt \footnotesize (2.19)       &\tt 64           & gfortran \footnotesize 9.1.0        & g++ \footnotesize 9.1.0          \\
Cluster        &           &                 &                     &                 &                                     &                                  \\
HPE/SGI ICE XA &\tt Linux  &\tt intel        &\tt mpt \footnotesize (2.19),      &\tt 64           & ifort \footnotesize 18.0.5.274      & g++ \footnotesize 18.0.5.274     \\
Cluster        &           &                 &\tt openmpi \footnotesize (3.1.4)  &                 &                                     &                                  \\
               &           &                 &\tt intelmpi \footnotesize (2018.4.274)  &           &                                     &                                  \\
HPE/SGI ICE XA &\tt Linux  &\tt intel        &\tt mpt \footnotesize (2.19)       &\tt 64           & ifort \footnotesize 19.0.2.187      & g++ \footnotesize 19.0.2.187     \\
Cluster        &           &                 &                     &                 &                                     &                                  \\
%Summitdev
IBM Power      &\tt Linux  &\tt gfortran     &\tt mpiuni           &\tt 64           & gfortran \footnotesize 4.8.5        & g++ \footnotesize 4.8.5 \\
IBM Power      &\tt Linux  &\tt pgi          &\tt mpiuni           &\tt 64           & pgf90 \footnotesize 19.7-0          & g++ \footnotesize 19.7-0 \\
%Eris
Mac Xeon       &\tt Darwin &\tt gfortran     &\tt mpiuni           &\tt 64           & gfortran \footnotesize 6.1.0        & g++ \footnotesize 6.1.0 \\
Mac Xeon       &\tt Darwin &\tt gfortran     &\tt openmpi \footnotesize (1.8)    &\tt 64           & gfortran \footnotesize 4.9.2        & g++ \footnotesize 4.9.2           \\
Mac Xeon       &\tt Darwin &\tt \footnotesize gfortranclang&\tt mpiuni           &\tt 64           & gfortran \footnotesize 6.1.0        & clang \footnotesize 1000.10.44.4  \\
%Catania
Mac Xeon       &\tt Darwin &\tt gfortran     &\tt mpiuni           &\tt 64           & gfortran \footnotesize 9.2.0        & g++ \footnotesize 9.2.0 \\
%Rutgers
Mac Xeon       &\tt Darwin &\tt gfortran     &\tt mpiuni,          &\tt 64           & gfortran \footnotesize 7.3.0        & g++ \footnotesize 7.3.0 \\
               &           &                 &\tt openmpi \footnotesize (2.1.5),&    &                                     &                                  \\
               &           &                 &\tt openmpi \footnotesize (3.1.3)&     &                                     &                                  \\
Mac Xeon       &\tt Darwin &\tt \footnotesize gfortranclang&\tt mpiuni           &\tt 64   & gfortran \footnotesize 7.3.0  & clang \footnotesize 902.0.39.2   \\
Mac Xeon       &\tt Darwin &\tt intel        &\tt mpiuni,          &\tt 64           & ifort \footnotesize 18.0.2.164      & ifort \footnotesize 18.0.2.164   \\
               &           &                 &\tt openmpi \footnotesize (2.1.5)&     &                                     &                                  \\
%Linux-regtest2
PC Xeon        &\tt Linux  &\tt gfortran     &\tt mpiuni,          &\tt 64           & gfortran \footnotesize 4.8.5        & g++ \footnotesize 4.8.5           \\
               &           &                 &\tt mpich3 \footnotesize (3.2.1) &     &                                     &                                   \\
PC Xeon        &\tt Linux  &\tt gfortran     &\tt mpiuni,          &\tt 64           & gfortran \footnotesize 7.3.0        & g++ \footnotesize 7.3.0           \\
               &           &                 &\tt mpich3 \footnotesize (3.2.1) &     &                                     &                                   \\
%Marktest
PC Xeon        &\tt Linux  &\tt gfortran     &\tt mpich3 \footnotesize (3.2.1)&\tt 64& gfortran \footnotesize 4.8.5        & g++ \footnotesize 4.8.5           \\
PC Xeon        &\tt Linux  &\tt gfortran     &\tt openmpi \footnotesize (3.1.1),&\tt 64& gfortran \footnotesize 8.1.0      & g++ \footnotesize 8.1.0           \\
               &           &                 &\tt mpich3 \footnotesize (3.2.1) &     &                                     &                                   \\
%Bebop
PC Xeon        &\tt Linux  &\tt gfortran     &\tt mvapich2 \footnotesize (2.3a), &\tt 64 & gfortran \footnotesize 7.1.0    & g++  \footnotesize 7.1.0          \\
Cluster        &           &                 &\tt mpich3 \footnotesize (3.2),    &                 &                                     &                     \\
               &           &                 &\tt openmpi \footnotesize (2.1.1), &                 &                                     &                     \\
               &           &                 &\tt intelmpi \footnotesize (2018.4.274)  &           &                                     &                     \\
PC Xeon        &\tt Linux  &\tt intel        &\tt mvapich2 \footnotesize (2.3) , &\tt 64 & ifort \footnotesize 18.0.5.274   & icpc  \footnotesize 18.0.5.274   \\
Cluster        &           &                 &\tt openmpi \footnotesize (3.1.3), &                 &                                     &                     \\
               &           &                 &\tt intelmpi \footnotesize (2018.4.274)  &           &                                     &                     \\
%Discover
PC Xeon        &\tt Linux  &\tt gfortran     &\tt mpiuni,   &\tt 64             & gfortran \footnotesize 4.8.1        & g++ \footnotesize 4.8.1                \\
Cluster        &           &                 &\tt mvapich2 \footnotesize (1.9),  &                 &                                     &                     \\
               &           &                 &\tt openmpi \footnotesize(1.7.2)  &                 &                                     &                      \\
PC Xeon        &\tt Linux  &\tt gfortran     &\tt mpiuni,   &\tt 64             & gfortran \footnotesize 4.9.2        & g++ \footnotesize 4.9.2                \\
Cluster        &           &                 &\tt mvapich2 \footnotesize (2.1),  &                 &                                     &                     \\
PC Xeon        &\tt Linux  &\tt intel        &\tt intelmpi \footnotesize (5.0.3.048) &\tt 64 & ifort \footnotesize 15.0.2.164 & icpc \footnotesize 15.0.2.164  \\
Cluster        &           &                 &                     &                 &                                     &                                   \\
PC Xeon        &\tt Linux  &\tt intel        &\tt mpiuni,  &\tt 64           & ifort \footnotesize 17.0.4.196      & icpc \footnotesize 17.0.4.196             \\
Cluster        &           &                 &\tt mvapich2 \footnotesize (2.3b)  &                 &                                     &                     \\
PC Xeon        &\tt Linux  &\tt intel        &\tt mvapich2 \footnotesize (2.3b)  &\tt 64 & ifort \footnotesize 17.0.4.196  & icpc \footnotesize 17.0.4.196     \\
Cluster        &           &                 &                     &                 &                                     &                                   \\
PC Xeon        &\tt Linux  &\tt intel        &\tt intelmpi \footnotesize (5.1.2.150) &\tt 64 & ifort \footnotesize 18.0.1.163  & icpc \footnotesize 18.0.1.163 \\
Cluster        &           &                 &                     &                 &                                     &                                   \\
PC Xeon        &\tt Linux  &\tt intel        &\tt openmpi \footnotesize (3.1.1)  &\tt 64 & ifort \footnotesize 18.0.3.222  & icpc \footnotesize 18.0.3.222     \\
Cluster        &           &                 &                     &                 &                                     &                                   \\
PC Xeon        &\tt Linux  &\tt intel        &\tt mpiuni,           &\tt 64           & ifort \footnotesize 18.0.5.274     & icpc \footnotesize 18.0.5.274     \\
Cluster        &           &                 &\tt intelmpi \footnotesize (18.0.5.274) &                 &                  &                                   \\
PC Xeon        &\tt Linux  &\tt nag          &\tt mpiuni           &\tt 64           & nagfor \footnotesize 6.2            & g++  \footnotesize 4.8.1          \\
Cluster        &           &                 &                     &                 &                                     &                                   \\
PC Xeon        &\tt Linux  &\tt pgi          &\tt mvapich2 \footnotesize (2.0b)  &\tt 64 & pgf90 \footnotesize 14.1-0      & pgc++ \footnotesize 14.1-0        \\
Cluster        &           &                 &                     &                 &                                     &                                   \\
PC Xeon        &\tt Linux  &\tt pgi          &\tt mpiuni,          &\tt 64           & pgf90 \footnotesize 17.5-0          & pgc++ \footnotesize 17.5-0        \\
Cluster        &           &                 &\tt openmpi \footnotesize (2.1.1)&     &                                     &                                   \\
PC Xeon        &\tt Linux  &\tt pgi          &\tt openmpi \footnotesize (2.1.1)&\tt 64& pgf90 \footnotesize 17.7-0         & pgc++ \footnotesize 17.7-0        \\
Cluster        &           &                 &                     &                 &                                     &                                   \\
PC Xeon        &\tt Linux  &\tt pgi          &\tt openmpi \footnotesize (3.1.1)&\tt 64& pgf90 \footnotesize 18.5-0         & pgc++ \footnotesize 18.5-0        \\
Cluster        &           &                 &                     &                 &                                     &                                   \\
%Jet
PC Xeon        &\tt Linux  &\tt intel        &\tt intelmpi \footnotesize (2018.4.274)&\tt 64& ifort \footnotesize 18.0.5.274 & icpc \footnotesize 18.0.5.274   \\
Cluster        &           &                 &                     &                 &                                     &                                   \\
PC Xeon        &\tt Linux  &\tt pgi          &\tt mpiuni           &\tt 64           & pgf90 \footnotesize 18.10-1         & pgc++ \footnotesize 18.10-1       \\
Cluster        &           &                 &\tt mvapich2 \footnotesize (2.3) &     &                                     &                                   
\end{longtable}

\vspace{1ex}

{\bf All possible options}. Where multiple options exist, and the default is independent
of {\tt ESMF\_MACHINE}, the default value is in {\bf bold}:

\vspace{1ex}


\begin{longtable}{lllll}
  {\bfseries\footnotesize ESMF\_OS} &{\bfseries\footnotesize ESMF\_COMPILER} & {\bfseries\footnotesize ESMF\_COMM} & {\bfseries\footnotesize ESMF\_ABI} \\

AIX     &\tt default     &\footnotesize \tt mpiuni,{\bf mpi},user      &\tt 32, {\bf 64} \\
Cygwin  &\tt g95         &\footnotesize \tt {\bf mpiuni},mpich,mpich2,mpich3,lam,openmpi,user &\tt 32, 64 \\
Cygwin  &\tt gfortran    &\footnotesize \tt {\bf mpiuni},mpich,mpich2,mpich3,lam,msmpi,openmpi,user &\tt 32, 64 \\
Darwin  &\tt absoft      &\footnotesize \tt {\bf mpiuni},mpich,mpich2,mpich3,mvapich2,lam,openmpi,user &\tt 32, 64 \\
Darwin  &\tt g95         &\footnotesize \tt {\bf mpiuni},mpich,mpich2,mpich3,mvapich2,lam,openmpi,user &\tt 32, 64 \\
Darwin  &\tt gfortran    &\footnotesize \tt {\bf mpiuni},mpich,mpich2,mpich3,mvapich2,lam,openmpi,user &\tt 32, 64 \\
Darwin  &\tt gfortranclang &\footnotesize \tt {\bf mpiuni},mpich,mpich2,mpich3,mvapich2,lam,openmpi,user &\tt 32, 64 \\
Darwin  &\tt intel       &\footnotesize \tt {\bf mpiuni},mpich,mpich2,mpich3,mvapich2,intelmpi,lam,openmpi,user &\tt 32, 64 \\
Darwin  &\tt intelclang  &\footnotesize \tt {\bf mpiuni},mpich,mpich2,mpich3,intelmpi,lam,openmpi,user &\tt 32, 64 \\
Darwin  &\tt intelgcc    &\footnotesize \tt {\bf mpiuni},mpich,mpich2,mpich3,intelmpi,lam,openmpi,user &\tt 32, 64 \\
Darwin  &\tt nag         &\footnotesize \tt {\bf mpiuni},mpich,mpich2,mpich3,mvapich2,lam,openmpi,user &\tt 32, 64 \\
Darwin  &\tt pgi         &\footnotesize \tt {\bf mpiuni},mpich,mpich2,mpich3,mvapich,mvapich2,lam,openmpi,user &\tt 32, 64 \\
Darwin  &\tt xlf         &\footnotesize \tt mpiuni,{\bf mpi},mpich,mpich2,mpich3,lam,openmpi,user &\tt 32 \\
Darwin  &\tt xlfgcc      &\footnotesize \tt mpiuni,{\bf mpi},mpich,mpich2,mpich3,lam,openmpi,user &\tt 32 \\
IRIX64  &\tt default     &\footnotesize \tt mpiuni,{\bf mpi},user     &\tt 32, {\bf 64} \\
Linux   &\tt absoft      &\footnotesize \tt {\bf mpiuni},mpich,mpich2,mpich3,mvapich2,lam,openmpi,user &\tt 32, 64 \\
Linux   &\tt absoftintel &\footnotesize \tt {\bf mpiuni},mpich,mpich2,mpich3,lam,openmpi,user &\tt 32, 64  \\
Linux   &\tt g95         &\footnotesize \tt {\bf mpiuni},mpich,mpich2,mpich3,mvapich2,lam,openmpi,user &\tt 32, 64, \\
        &                &                              &\tt ia64\_64, \\
        &                &                              &\tt x86\_64\_32, \\
        &                &                              &\tt x86\_64\_small, \\
        &                &                              &\tt x86\_64\_medium \\
Linux   &\tt gfortran    &\footnotesize \tt {\bf mpiuni},mpi,mpt,mpich,mpich2,mpich3,mvapich2, &\tt 32, 64, \\
        &                &\footnotesize \tt intelmpi,lam,openmpi,user                          &\tt ia64\_64, \\
        &                &                              &\tt x86\_64\_32, \\
        &                &                              &\tt x86\_64\_small, \\
        &                &                              &\tt x86\_64\_medium \\
Linux   &\tt gfortranclang &\footnotesize \tt {\bf mpiuni},mpi,mpt,mpich,mpich2,mpich3,mvapich2, &\tt 32, 64, \\
        &                & \footnotesize \tt lam,openmpi,user                                    &\tt ia64\_64, \\
        &                &                              &\tt x86\_64\_32, \\
        &                &                              &\tt x86\_64\_small, \\
        &                &                              &\tt x86\_64\_medium \\
Linux   &\tt intel       &\footnotesize \tt {\bf mpiuni},mpi,mpt,mpich,mpich2,mpich3,mvapich2, &\tt 32, 64, \\
        &                &\footnotesize \tt intelmpi,scalimpi,lam,openmpi,user                 &\tt ia64\_64, \\
        &                &                              &\tt x86\_64\_32, \\
        &                &                              &\tt x86\_64\_small, \\
        &                &                              &\tt x86\_64\_medium, \\
        &                &                              &\tt mic \\
Linux   &\tt intelgcc    &\footnotesize \tt {\bf mpiuni},mpi,mpt,mpich,mpich2,mpich3,mvapich2, &\tt 32, 64, \\
        &                &\footnotesize \tt intelmpi,lam,openmpi,user                          &\tt ia64\_64, \\
        &                &                              &\tt x86\_64\_32, \\
        &                &                              &\tt x86\_64\_small, \\
        &                &                              &\tt x86\_64\_medium \\
Linux   &\tt lahey       &\footnotesize \tt {\bf mpiuni},mpich,mpich2,mpich3,mvapich2,lam,openmpi,user &\tt 32, 64 \\
Linux   &\tt nag         &\footnotesize \tt {\bf mpiuni},mpich,mpich2,mpich3,mvapich2,lam,openmpi,user &\tt 32, 64 \\
Linux   &\tt nagintel    &\footnotesize \tt {\bf mpiuni},mpich,mpich2,mpich3,lam,openmpi,user &\tt 32, 64 \\
Linux   &\tt pathscale   &\footnotesize \tt {\bf mpiuni},mpich,mpich2,mpich3,lam,openmpi,user &\tt 32, 64, \\
        &                &                              &\tt x86\_64\_32, \\
        &                &                              &\tt x86\_64\_small, \\
        &                &                              &\tt x86\_64\_medium \\
Linux   &\tt pgi         &\footnotesize \tt {\bf mpiuni},mpi,mpt,mpich,mpich2,mpich3,mvapich,mvapich2 &\tt 32, 64, \\
        &                &\footnotesize \tt intelmpi,scalimpi,lam,openmpi,user &\tt x86\_64\_32, \\
        &                &                              &\tt x86\_64\_small, \\
        &                &                              &\tt x86\_64\_medium \\
Linux   &\tt pgigcc      &\footnotesize \tt {\bf mpiuni},mpich,mpich2,mpich3,lam,openmpi,user &\tt 32 \\
Linux   &\tt sxcross     &\footnotesize \tt mpiuni,{\bf mpi},user      &\tt 32  \\
Linux   &\tt xlf         &\footnotesize \tt mpiuni,{\bf mpi},user      &\tt 32  \\
MinGW   &\tt gfortran    &\footnotesize \tt {\bf mpiuni},msmpi,user    &\tt 32, 64 \\
MinGW   &\tt intel       &\footnotesize \tt {\bf mpiuni},msmpi,user    &\tt 32, 64 \\
MinGW   &\tt intelcl     &\footnotesize \tt {\bf mpiuni},msmpi,user    &\tt 32, 64 \\
OSF1    &\tt default     &\footnotesize \tt mpiuni,{\bf mpi},user      &\tt 64  \\
SunOS   &\tt default     &\footnotesize \tt mpiuni,{\bf mpi},user      &\tt 32, {\bf 64} \\
Unicos  &\tt default     &\footnotesize \tt mpiuni,{\bf mpi},user      &\tt 64  \\
Unicos  &\tt cce         &\footnotesize \tt mpiuni,{\bf mpi},user      &\tt 64  \\
Unicos  &\tt gfortran    &\footnotesize \tt mpiuni,{\bf mpi},user      &\tt 64  \\
Unicos  &\tt intel       &\footnotesize \tt mpiuni,{\bf mpi},user      &\tt 64  \\
Unicos  &\tt pathscale   &\footnotesize \tt mpiuni,{\bf mpi},user      &\tt 64  \\
Unicos  &\tt pgi         &\footnotesize \tt mpiuni,{\bf mpi},user      &\tt 64

\end{longtable}

\vspace{1ex}



Simultaneous multiple architecture builds are supported, with
one restriction; the test cases may only be run on one platform at a time. 

\subsubsection{Building the ESMF Libraries}
\label{BuildESMF}

Build the library with the command:
\begin{verbatim}
  gmake BOPT=g  
\end{verbatim}
  for a debug version or
\begin{verbatim}
  gmake BOPT=O  
\end{verbatim}
  for an optimized version.

% GNU make requirement.  File in build/doc
% $Id$ 

% Text about GNU make  This file is 
% meant to be included in a user doc.

GNU Make is required to build the ESMF library.  On some
systems this will be just the command \texttt{make}.  On others
it might be installed as \texttt{gmake} or \texttt{gnumake}.
This document uses {\tt make} consistently to refer to GNU Make.

Use the {\tt \verb+--+version} option with the locally available make commands
to determine which variant corresponds to GNU Make on your system. Use the
respective command when interacting with the ESMF build system, and
where this documentation uses {\tt make}.

Notice that ESMF does not utilize Autotools (configure or autoconf) or CMake.
Instead, the selection of configuration options is done by setting environment
variables before building the framework. The relevant environment variables
all begin with prefix {\tt ESMF\_}, and are discussed in detail under section
\ref{EnvironmentVariables}.


Build options that enable you to copy the library and *.mod files to
specified directories are explained in Section~\ref{BuildOptions}. 

\subsubsection{Building the ESMF Documentation}
\label{BuildDocumentation}

\noindent To build documentation:
\begin{verbatim}
  gmake dvi           ! Makes the dvi files.
  gmake pdf           ! Makes the pdf files.
  gmake html          ! Creates the html directory.
  gmake alldoc        ! Builds all the above documents.
\end{verbatim}

\noindent To build documentation for one module:

\noindent First change directory to the where the desired module's documentation resides;  for
example, to build the {\tt TimeMgr} documentation start with:

\begin{verbatim}
cd ${ESMF_DIR}/src/Infrastructure/TimeMgr/doc
\end{verbatim}

\noindent Next issue one of the following commands:
\begin{verbatim}
  gmake dvi      ! Builds local dvi files.
  gmake pdf      ! Builds local pdf files.
  gmake html     ! Builds local html files.
  gmake alldoc   ! Builds all of the local documents.
\end{verbatim}

\noindent The output from this local documentation build is in the top level {\tt doc}
directory, as with the previous commands.






