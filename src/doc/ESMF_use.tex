% $Id: ESMF_use.tex,v 1.3 2003/05/07 20:15:12 cdeluca Exp $

\subsection{Using the ESMF}
\label{UsingLibrary}

To use ESMF from Fortran, link with the shared object library and
create links to the library modules in your build directory.  The
*.mod files are
in the top level {\tt mod} directory under the appropriate architecture.  
Alternately, most compilers have a module-include-path directive which 
may be used to point to the correct module directory.

There is a single ESMF module, called {\tt ESMF\_Mod}, that should be 
include in applications with the Fortran {\tt USE} statement.  It 
is not necessary to include any header files in Fortran.

To use ESMF from C/C++, link with the ESMF shared object library 
and include the {\tt ESMC.h} file.

\subsubsection{Shared Object Libraries and Linking}

In simplest terms a shared object is a type of UNIX software library. 
Under Linux it is also known as a shared library. 

Shared object libraries are libraries that are loaded by the first program that uses them. All programs that start afterwards automatically use the existing shared library. The library is kept in memory as long as any active program is still using it. 

Since shared object libraries are pre-linked to system libraries, using them
simplifies life for the user when a variety of system libraries are
required or when system libraries vary a great deal on a 
platform-to-platform basis.  ESMF requires linking to both Fortran90 and
C++ libraries on a set of very non-standardized platforms, and using
shared objects helps to hide some of this complexity.

Using the shared libraries provided by ESMF is handled in a manner similar to standard libraries that you have probably encountered in the past. The descriptions below show examples of how to link to the ESMF shared library \emph{'libesmf.so'} on each of the platforms currently supported by the ESMF development team. The example for each platform is using a Fortran 90 application, \emph{'myModel'}, and linking it with the ESMF shared object library, \emph{'libesmf.so'}

\noindent{Linking using the Compaq:}

\begin{verbatim}
   f90 -I.. -o myModel myModel.f ../libesmf.so
\end{verbatim}

\noindent{Linking using the SGI:} 

\begin{verbatim}
   f90 -64 -c myModel.F I.. 
   f90 -64 -o myModel myModel.o -L.. -lesmf
\end{verbatim}

\noindent{Linking using the IBM:}

\begin{verbatim}
   xlf90\_r -I.. -o myModel myModel.F -L. -L.. -lesmf -brtl -lC
\end{verbatim}

These following are the linking commands for platforms for which 
we offer limited support.

\noindent{Linking using Linux:} 
 
\begin{verbatim}
   lf95 -I.. -o myModel myModel.F ../libesmf.so
   pgf90 -I.. -o myModel myModel.F ../libesmf.so
\end{verbatim}

\noindent{Linking using the Sun:} 

\begin{verbatim}
   f90 -M.. -o myModel myModel.F ../libesmf.so
\end{verbatim}

\noindent{Linking using the Alpha:} 

\begin{verbatim}
   f90 myModel.F -I..  -L.. -L/usr/ccs/lib/cmplrs/cc -lcxx -lesmf  -o myModel
\end{verbatim}

The order in which the libraries are presented to the linker is important. Library routines must be called before they are defined. So, if a library \emph{'A'} uses functionality provided by library \emph{'B'}, then library \emph{'A'} must appear before library \emph{'B'} during linking. 






