% $Id: ESMF_use.tex,v 1.25 2006/12/13 20:18:02 cdeluca Exp $

\subsection{Using the ESMF}
\label{UsingLibrary}

To use an ESMF library to compile and link your application against you will
need to tell the compiler where the ESMF module and ESMF library files
are located. If this has already been documented by the installer of the ESMF
library, follow the directions given. However, if not, then you must add the
correct compiler and linker flags yourself.

In order to compile a Fortran application against ESMF add the directory that
contains the ESMF {\tt *.mod} file(s):

\begin{verbatim}
$ESMF_DIR/mod/mod$ESMF_BOPT/$ESMF_OS.$ESMF_COMPILER.$ESMF_ABI.$ESMF_SITE
\end{verbatim} 

For most Fortran compilers this path is identified either with a {\tt -I} or a
{\tt -M}. Linking against the ESMF library is a bit more complex because of
ESMF's dependency on the Fortran90 {\em and} C++ runtime libraries. Generally
it is not enough to sepcify the location of the ESMF library:

\begin{verbatim}
$ESMF_DIR/lib/lib$ESMF_BOPT/$ESMF_OS.$ESMF_COMPILER.$ESMF_ABI.$ESMF_SITE
\end{verbatim} 

via the {\tt -L} flag followed by {\tt -lesmf}! However, ESMF provides a 
file called {\tt esmf.mk} that is generated during the library build and
stored in the same location as the ESMF library (see path just above). This
makefile fragment defines several variables indicating compiler options, include
paths, library paths and libraries necessary to compile and link against ESMF
with Fortran or C++ compilers. Look at "esmf.mk" to pull out the necessary flags
to compile and link your own code against the installed ESMF library.

Alternatively, you may want to include "esmf.mk" from within your own build
system. All of the variables defined in "esmf.mk" have prefix "ESMF\_"
as to prevent name space conflicts with users' makefiles. Notice that "esmf.mk"
is a self-contained file and is not affected by environment variables. It is
not necessary to set any ESMF\_ environment variables to use an ESMF library
that has been installed on your system!

When building the ESMF libraries on platforms that support both
32 and 64 bit application binary interfaces, verify that the ESMF\_ABI
environment variable is set to match the compile option that was specified 
to build your application.

\subsubsection{Shared Object Libraries}

On some platforms, a shared object library is created in addition to the
standard {\tt .a} archive library.
Shared object libraries are libraries that are loaded by the first program 
that uses them. All programs that start afterwards automatically use the 
existing shared library. The library is kept in memory as long as any 
active program is still using it. 

Shared object libraries can be pre-linked to system libraries and using them
can simplify dealing with ESMF's dependency on Fortran90 and C++ runtime 
libraries. 

\subsubsection{Customized SITE Files}

In an effort to provide platform specific information for building ESMF 
and linking the libraries with your application, a SourceForge 
site, {\tt esmfcontrib}, has been created.
To locate the platform makefiles for a specific institution, check out 
the {\tt build\_config\_files} using the appropriate CVSROOT.
The URL for the {\tt esmfcontrib} SourceForge site is:

\begin{verbatim}
        http://sourceforge.net/projects/esmfcontrib/
\end{verbatim}

Additionally, you may check out all the platform makefile fragments 
for a particular institution from the {\tt esmfcontrib} site. For example, 
to check out the available makefile fragments for platforms at the
National Center for Atmospheric Research, {\tt ncar}, change directories to

\begin{verbatim}
 	$ESMF_DIR/build_config
\end{verbatim}

and use the following CVS command:

\begin{verbatim}
	cvs -z3 -d:ext:$username@cvs.sourceforge.net:/cvsroot/esmfcontrib checkout ncar
\end{verbatim}

The following directories will be checked out:

\begin{verbatim}
	AIX.default.bluesky
	Linux.lahey.longs
\end{verbatim}

To build using these makefiles you must set the environment 
variable {\tt ESMF\_SITE} to {\tt bluesky}, or {\tt longs}.

At the present time, we have files for the following institutions:

\begin{verbatim}
anl  - Argonne National Laboratory
cola - Center for Ocean-Land-Atmosphere Studies
gsfc - Goddard Space Flight Center
mit  - Massachusetts Institute of Technology
ncar - National Center for Atmospheric Research
\end{verbatim}


Users are encouraged to contribute pertinent information to the 
{\tt esmfcontrib} respository.




