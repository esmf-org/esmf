% $Id: Regrid_req.tex,v 1.7 2002/04/08 16:27:59 pwjones Exp $

%===============================================================================
% Requirements may be itemized under a main topic:
%===============================================================================
%===============================================================================
\req{General Regridding Requirements}

%-------------------------------------------------------------------------------

The following are general requirements for regridding operations and are in
addition to the applicable general ESMF requirements (see ESMF General
Requirements document).

%-------------------------------------------------------------------------------
\sreq{Creation}

Applications must be able to create a regridding and initialize
various time-independent regridding quantities.

\begin{reqlist}
{\bf Priority:} 1 \\
{\bf Source:} All codes will require this. \\
{\bf Status:} Proposed \\
{\bf Verification:} Unit test \\
{\bf Notes:} This function will, in many cases, be computing
             regridding weights and initializing various
             communication information for performing regridding.
\end{reqlist}

%-------------------------------------------------------------------------------
\sreq{Destruction}

Applications must be able to destroy regriddings to free up memory.

\begin{reqlist}
{\bf Priority:} 1 \\
{\bf Source:} POP, CICE, CAM desired \\
{\bf Status:} Proposed \\
{\bf Verification:} Unit test \\
{\bf Notes:} 
\end{reqlist}

%-------------------------------------------------------------------------------
\sreq{Query}

Applications must be able to query various properties of a regridding.

\begin{reqlist}
{\bf Priority:} 2 \\
{\bf Source:}  \\
{\bf Status:} Proposed \\
{\bf Verification:} Unit test \\
{\bf Notes:} Ideally, applications should not need to access internal
             regridding fields.  But it might be useful to access some
             aspects for error checking, optimization of data layout or
             renormalization.  Exact query functions will be determined
             after design of structure is determined.
\end{reqlist}

%-------------------------------------------------------------------------------
\sreq{Change}

Applications should be able to change individual properties of a
regridding.

\begin{reqlist}
{\bf Priority:} 3 \\
{\bf Source:}  \\
{\bf Status:} Proposed \\
{\bf Verification:} Unit test \\
{\bf Notes:} In general, this should be strongly discouraged as it may
             affect regridding properties like conservation.  The need
             for such a function will be determined by applications.
\end{reqlist}

%-------------------------------------------------------------------------------
\sreq{Reading}

Applications may be able to read regridding information from a file.

\begin{reqlist}
{\bf Priority:} 2 \\
{\bf Source:} POP, CICE, CAM desired \\
{\bf Status:} Proposed \\
{\bf Verification:} Unit test \\
{\bf Notes:} Useful if creating regriddings is time-consuming to avoid
             start-up costs. Also permits off-line computation of regridding.
\end{reqlist}

%-------------------------------------------------------------------------------
\sreq{Writing}

Applications may be able to write regridding information to a file.

\begin{reqlist}
{\bf Priority:} 2 \\
{\bf Source:} POP, CICE, CAM desired \\
{\bf Status:} Proposed \\
{\bf Verification:} Unit test \\
{\bf Notes:} Useful if creating regriddings is time-consuming - compute once
             and save for later re-use.  Also useful for any off-line
             use of same regridding info.
\end{reqlist}

%-------------------------------------------------------------------------------
\sreq{Support for ESMF Grids}

Regridding operations must be available for all supported ESMF grids,
including ungridded data.  Not all regridding operations are appropriate for
all grid types; restrictions will be noted in individual requirements.

\begin{reqlist}
{\bf Priority:} 1 \\
{\bf Source:} All codes require this. \\
{\bf Status:} Proposed \\
{\bf Verification:} System test \\
{\bf Notes:} 
\end{reqlist}

%-------------------------------------------------------------------------------
\sreq{Multiple fields}

Regridding of multiple fields (bundles of fields) with
a single call must be supported.

\begin{reqlist}
{\bf Priority:} 1  \\
{\bf Source:} CCSM requires \\
{\bf Status:} Proposed \\
{\bf Verification:} Unit test \\
{\bf Notes:} This should be supplied for efficiency, but may not
             be required to achieve functionality.
\end{reqlist}

\ssreq{Consistency of field bundles}

The regridding function must check if the input field bundle and output field
bundle are consistent with each other, particulary in number of fields, name of
fields and grids on which the fields are placed.

\begin{reqlist}
{\bf Priority:} 2 \\
{\bf Source:} POP, CICE, CAM desired \\
{\bf Status:} Proposed \\
{\bf Verification:} Unit test \\
{\bf Notes:} A rudimentary error check, but would probably rely on
             consistent naming convention for fields?
\end{reqlist}

%-------------------------------------------------------------------------------
\sreq{Multiple methods per grid pair}

It shall be possible to create more than one regridding for a given grid
pair.  In such a case, each regridding will be separate and distinct.

\begin{reqlist}
{\bf Priority:} 1 \\
{\bf Source:} POP, CICE, CAM required \\
{\bf Status:} Proposed \\
{\bf Verification:} Unit test \\
{\bf Notes:} Both non-conservative and conservative methods will be needed
             between the same two grids, but they should not be combined in the
             same regridding structure/instantiation.
\end{reqlist}

%-------------------------------------------------------------------------------
\sreq{Consistency of coordinates}

Regridding will assume source and destination grids will be
in compatible coordinate systems.  That is, no knowledge of units or
physics must be necessary to perform regridding.
This also applies to vertical grid systems.

\begin{reqlist}
{\bf Priority:} 1 \\
{\bf Source:} Required by all. \\
{\bf Status:} Proposed \\
{\bf Verification:} Code inspection  \\
{\bf Notes:} A horizontal grid pair must be either both in x-y units (m)
             with the same origin or both in lat-lon (deg) units unless
             we decide to include arbitrary projection information as part
             of a PhysGrid structure.
             Also, vertical regridding between all possible vertical grid
             systems is problematic.
\end{reqlist}

\ssreq{Consistency of coordinates check}

The regridding function must check that the grids are in fact consistent
with each other.

\begin{reqlist}
{\bf Priority:} 2 \\
{\bf Source:} Required by all. \\
{\bf Status:} Proposed \\
{\bf Verification:} Code inspection  \\
{\bf Notes:} Will require some standard nomenclature for grid attributes,
             particularly for vertical grids.
\end{reqlist}

%-------------------------------------------------------------------------------
\sreq{Interpolation adjoints}

Adjoints shall be supplied for all methods when possible.  This is generally
possible for regriddings that are independent of the field being regridded
(see following requirement) and that can be cast as a linear operator (eg
matrix multiplication).  Methods where adjoints are absolutely required
have been so noted within their own respective decriptions.

\begin{reqlist}
{\bf Priority:} 2 \\
{\bf Source:} \\
{\bf Status:} Proposed \\
{\bf Verification:} Unit test \\
{\bf Notes:} 
\end{reqlist}

%-------------------------------------------------------------------------------
\sreq{Masked regridding}

For all regridding methods, it shall be possible to restrict the
regridding to parts of a grid through the use of a mask.

\begin{reqlist}
{\bf Priority:} 1 \\
{\bf Source:} CCSM required \\
              WRF required \\
{\bf Status:} Proposed \\
{\bf Verification:} Unit test \\
{\bf Notes:} 
\end{reqlist}

\ssreq{Mask consistency}

If masks are supplied for both source and destination grids, a
method for checking consistency of those masks must be supplied.
Alternatively, a convention for resolving mask conflicts must
be determined (eg source grid is ``master'').

\begin{reqlist}
{\bf Priority:} 1 \\
{\bf Source:} CCSM required \\
              WRF required \\
{\bf Status:} Proposed \\
{\bf Verification:} Unit test \\
{\bf Notes:} 
\end{reqlist}

%-------------------------------------------------------------------------------
\sreq{Independence of field}

At least one regridding method must be completely independent of
field information.  Whenever possible, regridding should be formulated to be
independent of the field being regridded.  This requirement exists to
aid the creation of an adjoint, to enable pre-computation of regridding
weights and to enable re-use of regridding information for multiple
fields.

\begin{reqlist}
{\bf Priority:} 1 \\
{\bf Source:} Required by all. \\
{\bf Status:} Proposed \\
{\bf Verification:} Code inspection  \\
{\bf Notes:} 
\end{reqlist}

%-------------------------------------------------------------------------------
\sreq{Dependence of field}

For regridding schemes which might require field information,
the required field information must be passed as arguments.
Some higher-order regridding schemes require information on the
gradient or other moments of a field.  Other potential schemes
interpolate in the upwind direction, requiring knowledge of a
velocity field.  In such cases, this supplemental field
information must be computed by the application and passed to
the regridding function so that the regridding does not
require detailed knowledge of operators or topology on every
supported grid or field.

\begin{reqlist}
{\bf Priority:} 1 \\
{\bf Source:} Required by all. \\
{\bf Status:} Proposed \\
{\bf Verification:} Code inspection  \\
{\bf Notes:} If a way of passing user-defined functions
             were available, this requirement could be relaxed or even
             eliminated.
\end{reqlist}

%===============================================================================
\req{Regridding algorithms}
%-------------------------------------------------------------------------------

This section contains requirements on regridding algorithms themselves.

%-------------------------------------------------------------------------------
\sreq{Conservation}

At least one regridding method between two gridded fields must be
\htmlref{conservative}{glos:conservation}.  Where possible,
higher-order conservative methods should be supplied.  This requirement
applies only to ESMF grids which have an area (2-d), volume (3-d) or
linear region (1-d) associated with them such that conservation is well
defined.

\begin{reqlist}
{\bf Priority:} 1 \\
{\bf Source:} POP,CICE,CAM required \\
{\bf Status:} Proposed \\
{\bf Verification:} Unit test \\
{\bf Notes:} Methods exist for both first and second-order
             conservative schemes in 1-d and 2-d \cite{Jones1999}.
             A Monte Carlo method could be used for first-order conservative
             regridding of 3-d fields; 3-d extensions to the 2-d methods above
             exist for some special cases but are non-trivial.
             High-order conservative schemes are more expensive and
             no schemes higher than second-order have been implemented.

             MI - Some grids may have a higher-order integration method
             associated with them (overlappping functions as weights),
             potentially making conservative regridding difficult and expensive.
\end{reqlist}

\ssreq{Verification of conservation}

A method for verifying conservation must be supplied.

\begin{reqlist}
{\bf Priority:} 2 \\
{\bf Source:} \\
{\bf Status:} Proposed \\
{\bf Verification:} Unit test \\
{\bf Notes:} For error checking and testing.
\end{reqlist}

%-------------------------------------------------------------------------------
\sreq{Monotonicity}

At least one regridding method must be \htmlref{monotone}{glos:monotone}.

\begin{reqlist}
{\bf Priority:} 2-3? \\
{\bf Source:}  \\
{\bf Status:} Proposed \\
{\bf Verification:} Unit test \\
{\bf Notes:} Biogeochemical models may need this.  First-order
             conservative schemes are generally monotone by
             construction, so this could be satisfied by the
             conservation requirement for gridded data.
\end{reqlist}


%-------------------------------------------------------------------------------
\sreq{Higher-order schemes}

At least one regridding method must be higher than first
\htmlref{order}{glos:order}.  This is required for preventing
``patchwork'' patterns when regridding from coarse to fine
grids and for preventing discontinuities in gradients of
regridded fields.

\begin{reqlist}
{\bf Priority:} 1 \\
{\bf Source:}  POP, CICE, CAM required. \\
{\bf Status:} Proposed \\
{\bf Verification:} Unit test \\
{\bf Notes:} This will require either internal approximations to
             gradients (eg bilinear, bicubic, trilinear) or will require the
             user to pass gradient information (eg second-order conservative
             methods).  See requirements on field dependence.
\end{reqlist}

%-------------------------------------------------------------------------------
\sreq{Vector fields in physical space}

At least one regridding method must be available for regridding a horizontal
vector field with components aligned with physical directions
(eg zonal-meridional or x-y), where the physical direction may be
inferred by the grid type or specified by user.

\begin{reqlist}
{\bf Priority:} 1 \\
{\bf Source:}  POP, CICE, CAM required. \\
{\bf Status:} Proposed \\
{\bf Verification:} Unit test \\
{\bf Notes:} 
\end{reqlist}

%-------------------------------------------------------------------------------
\sreq{Vector fields in logical space}

At least one regridding method must be available for regridding a horizontal
vector field with components aligned along grid logical directions.
Logical directions here refer to directions parallel and perpendicular
to cell sides.

\begin{reqlist}
{\bf Priority:} 2-3 \\
{\bf Source:}  POP, CICE (CCSM) desired. \\
{\bf Status:} Proposed \\
{\bf Verification:} Unit test \\
{\bf Notes:} Currently working on whether it is even possible to do this
             in all cases, but willing to make the attempt.  If not possible,
             physical space requirement is adequate with requirement that
             component models perform appropriate rotations of vectors to
             logical directions from physical directions.
\end{reqlist}

%-------------------------------------------------------------------------------
\sreq{Regridding based on index space}

At least one regridding method must be available for regridding based
only on logical indices of grid points and thus only on DistGrid information.  
Such a function is useful for nested grid and multi-grid applications where 
no physical grid information is required for creating the regridding.

\begin{reqlist}
{\bf Priority:} 1 \\
{\bf Source:}  WRF required \\
{\bf Status:} Proposed \\
{\bf Verification:} Unit test \\
{\bf Notes:} Will need a general way to specify stencils
\end{reqlist}

\ssreq{Index space changes}

An efficient method must be supplied for rapidly changing the
regridding in cases where indices of one grid shift in relation 
to the other grid (eg as a nested grid moves in relation to its 
parent).

\begin{reqlist}
{\bf Priority:} 1 \\
{\bf Source:}  WRF required \\
{\bf Status:} Proposed \\
{\bf Verification:} Unit test \\
{\bf Notes:} 
\end{reqlist}

%-------------------------------------------------------------------------------
\sreq{Fourier transforms}

Methods shall be supplied for regridding between physical space and
Fourier space.  The adjoints shall also be supplied.  Ordering in
Fourier space will be defined by DistGrid.  This requirement applies only
to grids consistent with the Fourier transform (eg lat/lon grids,
reduced grids, spectral elements, etc.).

\begin{reqlist}
{\bf Priority:} 1 \\
{\bf Source:}  NCEP-GSM, NCEP-SSI (milestone) \\
{\bf Status:} Proposed \\
{\bf Verification:} Unit test \\
{\bf Notes:} 
\end{reqlist}

%-------------------------------------------------------------------------------
\sreq{Legendre transforms}

Methods shall be supplied for regridding between spectral space and
Fourier space.  The adjoints shall also be supplied.  This requirement
applies only to grids consistent with the Legendre transform (the
data must be located at appropriate quadrature points).

\begin{reqlist}
{\bf Priority:} 1 \\
{\bf Source:}  NCEP-GSM, NCEP-SSI (milestone) \\
{\bf Status:} Proposed \\
{\bf Verification:} Unit test \\
{\bf Notes:}
\end{reqlist}

%-------------------------------------------------------------------------------
\sreq{Other functional transforms}

Methods shall be supplied for regridding using user-supplied matrices,
particularly between functional space and physical space.
The adjoints shall also be supplied.  The grids again must be consistent
with the functional transform being applied.

\begin{reqlist}
{\bf Priority:} 1 \\
{\bf Source:}  NCEP-SSI (milestone) \\
{\bf Status:} Proposed \\
{\bf Verification:} Unit test \\
{\bf Notes:} MI - The NCEP-SSI transforms between vertical EOF space
             and vertical model levels.
\end{reqlist}

%-------------------------------------------------------------------------------
\sreq{Interpolating from gridded data to ungridded data}

All methods shall work for regridding FROM gridded data TO ungridded
data, except that no conservation properties are required.
Adjoints shall be supplied for interpolation TO ungridded data.

\begin{reqlist}
{\bf Priority:} 1 \\
{\bf Source:}  NCEP-SSI (milestone) \\
{\bf Status:} Proposed \\
{\bf Verification:} Unit test \\
{\bf Notes:} 
\end{reqlist}

%-------------------------------------------------------------------------------
\sreq{Interpolating from ungridded data to gridded data}

Methods for regridding FROM ungridded data TO gridded data may be
supplied (eg nearest-neighbor distance-weighted schemes).

\begin{reqlist}
{\bf Priority:} 3 \\
{\bf Source:}  \\
{\bf Status:} Proposed \\
{\bf Verification:} Unit test \\
{\bf Notes:} Generally, operations like this will be covered by
             data assimilation schemes, but simple methods may be useful
             for other model-data comparisons.
\end{reqlist}

%-------------------------------------------------------------------------------
\sreq{User-supplied regridding methods}

It shall be possible for users to supply their own regridding
routines.  This is especially useful for regriddings that are
strongly dependent on model fields.

\begin{reqlist}
{\bf Priority:} 3 \\
{\bf Source:}  \\
{\bf Status:} Proposed \\
{\bf Verification:} Unit test \\
{\bf Notes:} The implementation report will examine use of
             function pointers or their equivalent, but implementation
             may run into other interface issues.

\end{reqlist}

%===============================================================================
\req{Other utilities}
%-------------------------------------------------------------------------------

The following are utilities related to regridding which should be made
public.

%-------------------------------------------------------------------------------
\sreq{Exchange grid}

A method for constructing a new grid formed by the intersecting
cells of two grids shall be available.

\begin{reqlist}
{\bf Priority:}  \\
{\bf Source:}  NSIPP \\
{\bf Status:} Proposed \\
{\bf Verification:} Unit test \\
{\bf Notes:} 
\end{reqlist}

