% $Id: Time_desc.tex,v 1.5 2003/08/14 20:31:43 cdeluca Exp $
\label{sec:Time}

A Time represents a specific point in time.  In order to accommodate
the range of time scales in Earth system applications, Times in
the ESMF can be specified in many different ways, from years to 
nanoseconds.  The Time interface is designed so that you select one or 
more options from a list of time units in order to specify a 
Time. The options for specifying a Time are shown in 
Figure\ref~{fig:TimeOpts}.  

There are Time methods defined for setting and getting a
Time, incrementing and decrementing a Time by a TimeInterval,
taking the difference between two Times, and comparing Times.
Special quantities such as the middle of the month and the 
day of the year associated with a particular Time can be retrieved. 
There is a method for returning the Time value as a string in 
the ISO 8601 format YYYY-MM-DDThh:mm:ss.

A Time that is specified in hours, minutes, seconds, or subsecond intervals 
does not need to be associated with a standard calendar; a Time whose
specification includes larger units must be.  The ESMF representation
of a calendar, the Calendar class, is described in Section~\ref{sec:Calendar}.
The {\tt ESMF_TimeSet} method can be called to associate a Time with a 
Calendar.  If an operation is requested in which a Calendar is necessary 
and one has not been set, the ESMF method will return an error.

In the ESMF the TimeInterval class used to represent time periods.
This class is frequently used in combination with the Time class.
The Clock class, for example, advances model time by 
 
Times are used by other classes in the ESMF timekeeping system,
 such as Clocks (Section~\ref{sec:Clock}) and Alarms 
(Section~\ref{sec:Alarm}).




