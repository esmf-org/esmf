% $Id: Calendar_desc.tex,v 1.15 2008/09/17 05:23:25 eschwab Exp $

\label{sec:Calendar}
The Calendar class represents the standard calendars used in 
geophysical modeling:  Gregorian, Julian, Julian Day, Modified Julian Day, 
no-leap, 360-day, and no-calendar.  It also supports a user-customized 
calendar.  Brief descriptions are provided for each calendar below.  For more 
information on standard calendars, see ~\cite{Seidelman} and ~\cite{Meyer1}.
