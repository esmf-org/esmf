% $Id: Grid_desc.tex,v 1.5 2003/11/13 23:42:17 jwolfe Exp $
%
% Earth System Modeling Framework
% Copyright 2002-2003, University Corporation for Atmospheric Research, 
% Massachusetts Institute of Technology, Geophysical Fluid Dynamics 
% Laboratory, University of Michigan, National Centers for Environmental 
% Prediction, Los Alamos National Laboratory, Argonne National Laboratory, 
% NASA Goddard Space Flight Center.
% Licensed under the GPL.

% <Describe class function and relation to other classes.>

The ESMF Grid class represents all aspects of the computational domain and its
decomposition in a parallel-processing environment, and must provide access to
any necessary grid information to the rest of the ESMF.  The ESMF Grid class
is aggregated into the Field class and/or the Gridded Component class
and provides information needed in data communication methods like halo and
redistribution.  It has methods to internally generate a variety of
simple grids or read in more complicated grids provided by a user.  The
ESMF Grid class supports multi-component coupling by providing a common
structure necessary for Regridding.

A single Grid can have more than one related subGrid.  Each subGrid corresponds to
a different representation of the Grid.  For example, a staggered Grid could
have separate subGrids representing cell centers and cell faces.  A vertical
grid may also be represented as a subGrid.

More on methods....   ordering?

\subsubsection{Grid Classes}
The Grid class aggregates two internal classes: the DistGrid (Distributed
Grid) class and the PhysGrid (Physical Grid) class.  The separation into two
classes allows the code to differentiate between functions which define the
local decomposition of data and the local representation of the grid.  The Grid
class itself maintains general information about the global grid (e.g. the
grid type, staggering, and coordinate system).  The Grid class is relatively
thin and otherwise presents a unified interface for DistGrid and PhysGrid
functions.  Each Grid contains at least one DistGrid and one PhysGrid:
\begin{itemize}
\item {\bf DistGrid} The DistGrid class maintains the relationship of how
a DELayout maps onto a Grid representation and how that Grid is distributed.
DistGrids can represent the same Grid but have different mappings (staggerings)
and can be aggregated by the same Grid object. The DistGrid class maintains the
mapping between the global Grid and the local data distribution;  it has
methods to aid in the collection and communication of global data.
\item {\bf PhysGrid} The PhysGrid class maintains a local physical
representation of a Grid, including all necessary data and masks.  PhysGrids
can represent subGrids, like vertical Grids, of a single Grid and be
aggregated by the same Grid object. 
\end{itemize}
