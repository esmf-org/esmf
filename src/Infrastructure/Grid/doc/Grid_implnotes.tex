

%\subsection{Design and Implementation Notes}


\subsubsection{Grid Classes}
The Grid class contains two internal private classes: the DistGrid (Distributed
Grid) class and the PhysGrid (Physical Grid) class.  The separation into two
classes allows the code to differentiate between functions which define the
DE-local decomposition of data and the DE-local representation of the grid.
The Grid class itself maintains general information about the global grid (e.g.
the grid type, staggering, and coordinate system).  The Grid class is relatively
thin and otherwise presents a unified interface for DistGrid and PhysGrid
functions.  Each Grid contains at least one subGrid, represented by a unique
DistGrid and PhysGrid pair:
\begin{itemize}
\item {\bf DistGrid} The DistGrid class maintains the relationship of how
a DELayout maps onto a Grid representation and how that Grid is distributed.
DistGrids can represent the same Grid but have different mappings (staggerings)
and can be contained by the same Grid object. The DistGrid class represents the
mapping between the global Grid and the DE-local data distribution;  it has
methods to aid in the collection and communication of global data.
\item {\bf PhysGrid} The PhysGrid class maintains the DE-local decomposed
physical representation of a Grid, including all necessary coordinate data and
masks.  Separate PhysGrids are created for each relative location associated
with the Grid's staggering (e.g. a Grid with Arakawa D staggering will have
PhysGrids representing the cell centers and specified cell faces), as well as
for any vertical subGrids.
\end{itemize}

Some methods which have a Grid interface are actually implemented
at the underlying DistGrid or PhysGrid level; they will be inherited
by the Grid class.  This allows the API to present functions at the 
level which is most consistent to the application without restricting
the actual implementation.

