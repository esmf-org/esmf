% $Id: Grid_implnotes.tex,v 1.5 2004/08/16 22:53:57 jwolfe Exp $

%\subsection{Design and Implementation Notes}

Overall Grid design strategy...

\subsubsection{Grid Classes}
The Grid class contains two internal private classes: the DistGrid (Distributed
Grid) class and the PhysGrid (Physical Grid) class.  The separation into two
classes allows the code to differentiate between functions which define the
local decomposition of data and the local representation of the grid.  The Grid
class itself maintains general information about the global grid (e.g. the
grid type, staggering, and coordinate system).  The Grid class is relatively
thin and otherwise presents a unified interface for DistGrid and PhysGrid
functions.  Each Grid contains at least one DistGrid and one PhysGrid:
\begin{itemize}
\item {\bf DistGrid} The DistGrid class maintains the relationship of how
a DELayout maps onto a Grid representation and how that Grid is distributed.
DistGrids can represent the same Grid but have different mappings (staggerings)
and can be contained by the same Grid object. The DistGrid class represents the
mapping between the global Grid and the local data distribution;  it has
methods to aid in the collection and communication of global data.
\item {\bf PhysGrid} The PhysGrid class maintains a local physical
representation of a Grid, including all necessary data and masks.  PhysGrids
can represent subGrids, like vertical Grids, of a single Grid and be
contained by the same Grid object.
\end{itemize}

Some methods which have a Grid interface are actually implemented
at the underlying DistGrid or PhysGrid level; they will be inherited
by the Grid class.  This allows the API to present functions at the 
level which is most consistent to the application without restricting
where inside the ESMF the actual implementation is done.

