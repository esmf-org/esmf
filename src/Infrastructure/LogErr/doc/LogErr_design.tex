% $Id: LogErr_design.tex,v 1.1 2003/03/28 21:33:08 shep_smith Exp $
%
% Earth System Modeling Framework
% Copyright 2002-2003, University Corporation for Atmospheric Research, 
% Massachusetts Institute of Technology, Geophysical Fluid Dynamics 
% Laboratory, University of Michigan, National Centers for Environmental 
% Prediction, Los Alamos National Laboratory, Argonne National Laboratory, 
% NASA Goddard Space Flight Center.
% Licensed under the GPL.

%\subsection{Design/Implementation}

There are two aspects to the design of the Log class.  The first is the actual class
itself, and the second is Fortran interface to the Log class. The design of
the Log class is pretty straightforward. 
Although the class contains a large number of methods, the most important methods
are ESMF\_LogInit, for initializing data; ESMF\_LogOpenFile and ESMF\_CloseFile,
for opening and closing output files; and ESMF\_LogInfo, ESMF\_LogErr, and ESMF\_LWarn, for
writing miscellaneous information, error, and warning information. In addition, there are
a variety of other methods that allow the user to control other behavior such whether
or not to stop execution on encountering errors or warnings, whether output
should be flushed from buffers to files automatically, or whether output should be verbose
or not.

The Log class was implemented in C/C++, but uses the Fortran I/O libraries when
the class methods are called from the Fortran wrappers. We forced the C/C++ methods 
to use the Fortran I/O library by creating 
utility functions, written in Fortran, but callable from Log's C++ methods,
which call the standard Fortran write, open and close functions.  We chose this 
approach because we still wanted to allow users to write to the Log files using the
standard Fortran write statements if they chose to.
Since you could conceivably be writing to the same file using 
both ESMF\_LogInfo, for example,  and the standard Fortran  write(),
both approaches have to use the same I/O 
libraries to avoid any collisions.  Since we can't force the Fortran write() to use the
C I/O libraries, our only alternative was to force our routines to use Fortran I/O
libraries.  Note: if you call the Log methods from C/C++ code, you bypass the wrappers
and all I/O is done with the C I/O libraries.

The design of the Fortran interface is relatively straightforward, but there are some
inter-language operability issues here as well.
Because the ESMF\_LogInfo routine needed to
take a variable number of arguments, the Fortran wrapper needed to be implemented in C.
To keep things as uniform as possible, all of the other wrapper functions were implemented
in C as well, except for ESMF\_LogInit(). We wanted to overload this function, and since
you can't overload functions in C, we needed to implement the code in another language.
Fortran was the obvious choice.




