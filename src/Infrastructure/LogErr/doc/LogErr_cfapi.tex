%                **** IMPORTANT NOTICE *****
% This LaTeX file has been automatically produced by ProTeX v. 1.1
% Any changes made to this file will likely be lost next time
% this file is regenerated from its source. Send questions 
% to Arlindo da Silva, dasilva@gsfc.nasa.gov
 
\setlength{\parskip}{0pt}
\setlength{\parindent}{0pt}
\setlength{\baselineskip}{11pt}
 
%--------------------- SHORT-HAND MACROS ----------------------
\def\bv{\begin{verbatim}}
\def\ev{\end{verbatim}}
\def\be{\begin{equation}}
\def\ee{\end{equation}}
\def\bea{\begin{eqnarray}}
\def\eea{\end{eqnarray}}
\def\bi{\begin{itemize}}
\def\ei{\end{itemize}}
\def\bn{\begin{enumerate}}
\def\en{\end{enumerate}}
\def\bd{\begin{description}}
\def\ed{\end{description}}
\def\({\left (}
\def\){\right )}
\def\[{\left [}
\def\]{\right ]}
\def\<{\left  \langle}
\def\>{\right \rangle}
\def\cI{{\cal I}}
\def\diag{\mathop{\rm diag}}
\def\tr{\mathop{\rm tr}}
%-------------------------------------------------------------

\markboth{Left}{Source File: ESMC\_LogErrInterface.C,  Date: Fri Mar 28 13:11:41 EST 2003
}

 
%/////////////////////////////////////////////////////////////
\subsubsection [ESMF\_LogCloseFile] {ESMF\_LogCloseFile - closes a file from Fortran code}


  
   !INTERFACE 
   subroutine ESMF\_LogCloseFile(aLog)
  
\bigskip{\em ARGUMENTS:}
\begin{verbatim}     typdef(ESMF\_Log) :: aLog
   \end{verbatim}
{\sf DESCRIPTION:\\ }


   Calls the method ESMC\_LogCloseFileForWrite to close aLog's 
   log file.
   
%/////////////////////////////////////////////////////////////
 
\mbox{}\hrulefill\ 
 

  \subsubsection [ESMF\_LogOpenFile] {ESMF\_LogOpenFile - opens a log file}


\bigskip{\sf INTERFACE:}
\begin{verbatim}   subroutine ESMF_LogOpenFile(aLog, numFile, name)\end{verbatim}{\em ARGUMENTS:}
\begin{verbatim}    typdef(ESMF\_Log) :: aLog
    integer :: numFile           !! set to either ESMF_SINGLE_FILE
                                 !! or ESMF_MULTIPLE_FILE 
    characlter(len=*) :: name    !! name of file\end{verbatim}
{\sf DESCRIPTION:\\ }


   This routine finds the first space in the array name and inserts a
   a null character. It then calls ESMC\_LogOpenFileForWrote an ESMC\_Log method
   for opening files.
   
%/////////////////////////////////////////////////////////////
 
\mbox{}\hrulefill\ 
 
\subsubsection [ESMF\_LogInfo] {ESMF\_LogInfo - writes miscellaneous information to }


               a log file
  
\bigskip{\sf INTERFACE:}
\begin{verbatim}     
    ESMF\_LogInfo(aLog, fmt, ...)\end{verbatim}{\em ARGUMENTS:}
\begin{verbatim}    typedef(ESMF\_LogInfo) :: aLog   ! log object
    character(len=*) :: fmt         !c-style character descriptior\end{verbatim}
{\sf DESCRIPTION:\\ }


    This routine allows the user to write miscellaneous information the
    Log file. It uses a printf style character descriptor, e.g. 
    ESMF\_LogInfo(aLog,"Hi there, %s ", shep), where shep here would be
    a character string. The routine takes a variable number of arguments,
    so that any number of data items can be written to the Log file.
    Currently, only character, strings, integers, and reals are supported.
    However, field widths, precisions, and flags are ignored.
   
%/////////////////////////////////////////////////////////////
 
\mbox{}\hrulefill\ 
 

  \subsubsection [ESMF\_LogWarnMsg] {ESMF\_LogWarnMsg - writes a warning message to the log file}


  
\bigskip{\sf INTERFACE:}
\begin{verbatim}    subroutine ESMF\_LogWarnMsg(aLog, errCode, line,file,dir,msg)\end{verbatim}{\em ARGUMENTS:}
\begin{verbatim}      typdef(ESMF\_Log)::aLog
      integer :: errCode         !integer value for error code         
      character(len=*) :: msg    !msg written to log file
      integer :: line            !line number of warning; argument
                                 !supplied by macro
      character(len=*) :: file   !file where warning occurred in;
                                 !argument supplied by macro
      character(len=*) :: dir    !directory where warning occurred in;
                                 !argument supplied by macro\end{verbatim}
{\sf DESCRIPTION:\\ }


      This routine writes a warning message to the log file.  This warning
      message consists of the erroCode, a description of the warning, the 
      line number, file, and directory of the error, and a message. A 
      preprocessor macro adds the predefined preprocessor symbolic
      constants \_\_LINE\_\_, \_\_FILE\_\_, and \_\_DIR\_\_.
      The macro operates on
      the file from which these routines are called.  Note,
      the value of \_\_DIR\_\_ 
      must be suppliled by the user (usually done in
      the makefile.).  By default, execution continues after encountering
      a warning, but by calling the routine ESMF\_LogWarnHalt(), the user
      can halt on warnings.
   
%/////////////////////////////////////////////////////////////
 
\mbox{}\hrulefill\ 
 
\subsubsection [ESMF\_LogWarn] {ESMF\_LogWarn - writes a warnng message to log file}


  
\bigskip{\sf INTERFACE:}
\begin{verbatim}      ESMF\_LogWarn(aLog, errCode)\end{verbatim}{\em ARGUMENTS:}
\begin{verbatim}     typdef(ESMF\_Log) :: aLog
     integer :: errCode\end{verbatim}
{\sf DESCRIPTION:\\ }


     This routine is identical to ESMF\_LogWarnMsg, except a msg is
     not written to the log file.
   
%/////////////////////////////////////////////////////////////
 
\mbox{}\hrulefill\ 
 
\subsubsection [ESMF\_LogFlush] {ESMF\_LogFlush - flushes output}


  
\bigskip{\sf INTERFACE:}
\begin{verbatim}      ESMF\_logflush(aLog)\end{verbatim}{\em ARGUMENTS:}
\begin{verbatim}      typdef(ESMF\_Log) :: aLog\end{verbatim}
{\sf DESCRIPTION:\\ }


      This routine calls the Log method ESMC\_LogFlush() which sets a flag
      that causes all output from the buffers.
   
%/////////////////////////////////////////////////////////////
 
\mbox{}\hrulefill\ 
 
\subsubsection [ESMF\_LogNotFlush] {ESMF\_LogNotFlush - prevents output from being flushed}


  
\bigskip{\sf INTERFACE:}
\begin{verbatim}      ESMF\_LogNotFlush(aLog)\end{verbatim}{\em ARGUMENTS:}
\begin{verbatim}      typdef(ESMF\_Log) :: aLog\end{verbatim}
{\sf DESCRIPTION:\\ }


      This routine calls the Log method ESMC\_LogNotFlush() which sets a flag
      that turns off flushing. By default, this flag is set.
   
%/////////////////////////////////////////////////////////////
 
\mbox{}\hrulefill\ 
 
\subsubsection [ESMF\_LogVerbose] {ESMF\_LogVerbose - causes output to be written to the Log}


  
\bigskip{\sf INTERFACE:}
\begin{verbatim}     ESMF\_LogVerbose(aLog)\end{verbatim}{\em ARGUMENTS:}
\begin{verbatim}      typdef(ESMF\_Log) :: aLog\end{verbatim}
{\sf DESCRIPTION:\\ }


      This routine sets a flag that causes all output associated with
      the aLog ESMC\_Log handle to be written. 
%/////////////////////////////////////////////////////////////
 
\mbox{}\hrulefill\ 
 
\subsubsection [ESMF\_LogNotVerbose] {ESMF\_LogNotVerbose - causes output not to be written to the Log}


                         
\bigskip{\sf INTERFACE:}
\begin{verbatim}     ESMF\_LogVerbose(aLog)\end{verbatim}{\em ARGUMENTS:}
\begin{verbatim}      typdef(ESMF\_Log) :: aLog\end{verbatim}
{\sf DESCRIPTION:\\ }


      This routine sets a flag that forces all output associated with
      the aLog ESMC\_Log handle from being written. 
%/////////////////////////////////////////////////////////////
 
\mbox{}\hrulefill\ 
 

  \subsubsection [LogWrite] {LogWrite - Fortran style method to write to log file.}


  
\bigskip{\sf INTERFACE:}
\begin{verbatim}      LogWrite(aLog)\end{verbatim}{\em ARGUMENTS:}
\begin{verbatim}     typdef(ESMF\_Log) :: aLog\end{verbatim}
{\sf DESCRIPTION:\\ }


      This function called from with a Fortran write statement, e.g.
      write(LogWrite(aLog),*)"Hi".  The LogWrite function appends some
      header information (time,date etc.) to what ever is printed out
      from the write, e.g. Hi. 
%/////////////////////////////////////////////////////////////
 
\mbox{}\hrulefill\ 
 

  \subsubsection [ESMF\_LogErrMsg] {ESMF\_LogErrMsg - writes a err message to the log file}


  
\bigskip{\sf INTERFACE:}
\begin{verbatim}    subroutine ESMF\_LogErrMsg(aLog, errCode, line,file,dir,msg)\end{verbatim}{\em ARGUMENTS:}
\begin{verbatim}      typdef(ESMF\_Log)::aLog
      integer :: errCode         !integer value for error code
      character(len=*) :: msg    !msg written to log file
      integer :: line            !line number of warning; argument
                                 !supplied by macro
      character(len=*) :: file   !file where warning occurred in;
                                 !argument supplied by macro
      character(len=*) :: dir    !directory where warning occurred in;
                                 !argument supplied by macro\end{verbatim}
{\sf DESCRIPTION:\\ }


      This routine writes a warning message to the log file.  This warning
      message consists of the erroCode, a description of the warning, the
      line number, file, and directory of the error, and a message. A
      preprocessor macro adds the predefined preprocessor symbolic
      constants \_\_LINE\_\_, \_\_FILE\_\_, and \_\_DIR\_\_.
      The macro operates on
      the file from which these routines are called.  Note,
      the value of \_\_DIR\_\_
      must be suppliled by the user (usually done in
      the makefile).  By default, execution continues after encountering
      a warning, but by calling the routine ESMF\_LogWarnHalt(), the user
      can halt on warnings.
   
%/////////////////////////////////////////////////////////////
 
\mbox{}\hrulefill\ 
 
\subsubsection [ESMF\_LogErr] {ESMF\_LogErr - writes a warnng message to log file}


  
\bigskip{\sf INTERFACE:}
\begin{verbatim}      ESMF\_LogErr(aLog, errCode)\end{verbatim}{\em ARGUMENTS:}
\begin{verbatim}     typdef(ESMF\_Log) :: aLog
     integer :: errCode\end{verbatim}
{\sf DESCRIPTION:\\ }


     This routine is identical to ESMF\_LogErrMsg, except a msg is
     not written to the log file.
   
%/////////////////////////////////////////////////////////////
 
\mbox{}\hrulefill\ 
 

  \subsubsection [ESMF\_LogHaltOnErr] {ESMF\_LogHaltOnErr - program halts on encountering an error}


  
\bigskip{\sf INTERFACE:}
\begin{verbatim}      subroutine ESMF_LogHaltOnErr(aLog)
   !ARGUMENTS;
      typdef(ESMF\_Log) :: aLog\end{verbatim}
{\sf DESCRIPTION:\\ }


      This routine calls a Log method that sets a flag to stop execution on
      reaching an error. This is the default behavior of the Log class. 
%/////////////////////////////////////////////////////////////
 
\mbox{}\hrulefill\ 
 

  \subsubsection [ESMF\_LogNotHaltOnErr] {ESMF\_LogNotHaltOnErr - program does not halt}


                                     on encountering an error
  
\bigskip{\sf INTERFACE:}
\begin{verbatim}      subroutine ESMF\_LogNotHaltOnErr(aLog)\end{verbatim}{\em ARGUMENTS:}
\begin{verbatim}      typdef(ESMF\_Log) :: aLog\end{verbatim}
{\sf DESCRIPTION:\\ }


      This routine calls a Log method that sets a flag to prevent the program
      from stopping reaching an error.  
%/////////////////////////////////////////////////////////////
 
\mbox{}\hrulefill\ 
 

  \subsubsection [ESMF\_LogHaltOnWarn] {ESMF\_LogHaltOnWarn - program halts on encountering a warning}


  
\bigskip{\sf INTERFACE:}
\begin{verbatim}      subroutine ESMF\_LogHaltOnWarn(aLog)\end{verbatim}{\em ARGUMENTS:}
\begin{verbatim}      typdef(ESMF\_Log) :: aLog\end{verbatim}
{\sf DESCRIPTION:\\ }


      This routine calls a Log method that sets a flag to stop execution on
      reaching a warning. 
%/////////////////////////////////////////////////////////////
 
\mbox{}\hrulefill\ 
 

  \subsubsection [ESMF\_LogNotHaltOnWarn] {ESMF\_LogNotHaltOnWarn - program does not halt}


                                     on encountering a warning
  
   !INTERFACE;
      subroutine ESMF\_LogNotHaltOnWarn(aLog)             
\bigskip{\em ARGUMENTS:}
\begin{verbatim}      typdef(ESMF\_Log) :: aLog                                        
                                \end{verbatim}
{\sf DESCRIPTION:\\ }

 
      This routine calls a Log method that sets a flag to prevent the program
      from stopping reaching an error.                            
%...............................................................
