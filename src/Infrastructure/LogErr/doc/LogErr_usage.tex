% $Id: LogErr_usage.tex,v 1.8 2004/06/21 16:09:00 cpboulder Exp $



%\subsection{Use and Examples}


A default log is created on {\tt ESMF\_Initialize()} and placed in the

root ESMF directory.  ESMF handles the initialization and finalization of

the default log so the user can immediately start using the default log.

A single default log is opened in the XXXX directory.  If one is not 

present, a new one is created.  If multiple logs are desired, they must be

explicitly created or opened using {\tt ESMF\_LogOpen()}.


If a user wants to use a new or different log, the user must call

{\tt ESMF\_LogOpen()} and pass in a log object and file name to open a Log 

file.
  

By default, the log file is not truncated at the start of a new run; it just

gets appended to each time.  Future functionality could include an option to

either truncate or append to the log file. 


In all cases where a log is opened, a unit number is assigned to a specific

log.  A log will be assigned the lowest available unit number starting with

11.  If a unit number is occupied, the next higher unit number will be 

checked using the "inquire" method.  The process repeats until a free unit
 
number is found or when the unit number reaches {\tt ESMF\_LOGUPPER()} in
 
which case an error is returned.  As a result, the user should always check

for free numbers using "inquire" to prevent potential unit number conflicts.


The user can then set or get options on how the Log should be used 

with the {\tt ESMF\_LogSet()} and {\tt ESMF\_LogGet()} methods.  These are 

not fully implemented at this time. 


Depending on how the options are set, {\tt ESMF\_LogWrite()} either writes

user messages directly to a Log file or writes to a buffer that can be flushed

when full or by using the {\tt ESMF\_LogFlush()} method.  In the current

implementation the Log flushes after every write.  


For every {\tt ESMF\_LogWrite()}, a time and date stamp is prepended to the

log entry.  The time is given in microsecond precision.


When calling {\tt ESMF\_LogWrite()}, the user can supply an optional line,

file and method.  These arguments can be passed in explicitly or with the help

of cpp macros.  In the latter case, a define for an {\tt ESMF\_FILE} must be 

placed at the beginning of the code and a define for {\tt ESMF\_METHOD} must

be placed at the beginning of each method.  The user can then use the

{\tt ESMF\_CONTEXT} cpp macro in place of line, file and method to insert the 

parameters into the method.  The user does not have to specify line number as

it is a value supplied by cpp.


When done writing messages, the Log is closed by calling 

{\tt ESMF\_LogClose()} which will release the assigned unit number.


Some of the assumptions implicit in using the Log utility are:


\begin{enumerate}


\item Prior to closing, {\tt ESMF\_LogFlush} is always executed.

\item Prior to halting, {\tt ESMF\_LogFlush} is always executed.

\item If the number of elements in a Log exceeds {\tt maxElements}, 

a message is written to file and the buffer flushes automatically.


\end{enumerate}





