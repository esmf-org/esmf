% $Id: LogErr_usage.tex,v 1.18 2005/01/10 18:09:51 cpboulder Exp $

%\subsection{Use and Examples}

A default log is created at {\tt ESMF\_Initialize()}.  ESMF handles the 
initialization and finalization of the default log so the user can immediately
start using it.  A single default log is opened in the directory that 
initializes the default log.  If a log is not present, a new one is created.
If multiple log objects are desired, they must be explicitly created or opened 
using {\tt ESMF\_LogOpen()}.

If a user wants to use a new or different log, the user must call
{\tt ESMF\_LogOpen()} and pass in a Log object and filename to open a Log file.

Additionally,  the user can specify single or multi logs by setting the
{\tt defaultlogtype} property in the {\tt ESMF\_LogInitialize()} or 
{\tt ESMF\_Open()} method to {\tt ESMF\_LOG\_SINGLE} or {\tt ESMF\_LOG\_MULTI}.
This is useful as the PET numbers are automatically added to the log entries.
A single log will put all entries, irregardless of PET number, into a single
log while a multi log will create multiple logs with the PET number prepended
to the filename and all entries will be written to their corresponding log 
by their PET number.
 
By default, the log file is not truncated at the start of a new run; it just
gets appended each time.  Future functionality would include an option to
either truncate or append to the log file. 

In all cases where a Log is opened, a unit number is assigned to a specific
log.  A log is assigned the lowest available unit number starting with
11.  If a unit number is occupied, the next higher unit number is 
checked using the "inquire" method.  The process repeats until a free unit 
number is found or when the unit number reaches {\tt ESMF\_LOGUPPER} in 
which case an error is returned.  As a result, the user should always check
for free numbers using "inquire" to prevent potential unit number conflicts.
In the near future we anticipate supporting an option in which a desired
unit number can be passed in.

The user can then set or get options on how the log should be used 
with the {\tt ESMF\_LogSet()} and {\tt ESMF\_LogGet()} methods.  These are 
partially implemented at this time. 

Depending on how the options are set, {\tt ESMF\_LogWrite()} either writes user
messages directly to a log file or writes to a buffer that can be flushed when 
full or by using the {\tt ESMF\_LogFlush()} method.  The default is to flush 
after every ten entries because {\tt maxElements} is initialized to ten 
(which means the buffer reaches its full state after every ten writes and then
flushes).

For every {\tt ESMF\_LogWrite()}, a time and date stamp is prepended to the
log entry.  The time is given in microsecond precision.  The user call other 
methods to write to the log.  In every case, all methods eventually make a call
implicitly to {\tt ESMF\_LogWrite()} even though the user may never explicitly
call it.

When calling {\tt ESMF\_LogWrite()}, the user can supply an optional line,
file and method.  These arguments can be passed in explicitly or with the help
of cpp macros.  In the latter case, a define for an {\tt ESMF\_FILE} must be 
placed at the beginning of the code and a define for {\tt ESMF\_METHOD} must
be placed at the beginning of each method.  The user can then use the
{\tt ESMF\_CONTEXT} cpp macro in place of line, file and method to insert the 
parameters into the method.  The user does not have to specify line number as
it is a value supplied by cpp.

An example of log output is given below running with {\tt defaultlogtype} 
property set to {\tt ESMF\_LOG\_SINGLE} using the default log:

(Log file {\tt ESMF\_LogFile}
20041105 163418.472210 INFO      PET0     Running with ESMF Version 2.0.2   
20041105 163419.186153 ERROR     PET1     ESMF\_Field.F90             812  
ESMF\_FieldGet No Grid or Bad Grid attached to Field

The results are all put into the log file {\tt ESMF\_LogFile}.

The next example shows same messages running with {\tt defaultlogtype} 
property set to {\tt ESMF\_LOG\_MULTI} using the default log:

\begin {verbatim}
(Log file {\tt PET0.ESMF\_LogFile}
20041105 163418.472210 INFO      PET0     Running with ESMF Version 2.0.2   
(Log file {\tt PET1.ESMF\_LogFile}
20041105 163419.186153 ERROR     PET1     ESMF\_Field.F90             812  
ESMF\_FieldGet No Grid or Bad Grid attached to Field
\end {verbatim}

Note in this example that a separate file is created for each PET when using
{\tt ESMF\_LOG\_MULTI}.

The first entry shows date and time stamp.  The time is given in microsecond 
precision.  The next item shown is the type of message (INFO in this case).  
Next, the PET number is added.  Lastly, the context is written.

The second entry shows something slightly different.  In this case, we have
an ERROR.  The method name (ESMF\_Field.F90) is automatically provided from 
the cpp macros as well as the line number (812).  Then the context of the 
message is written.
 
When done writing messages, the default Log is closed by calling 
{\tt ESMF\_LogFinalize()}  or {\tt ESMF\_LogClose()} for user created Logs.  
Both methods will release the assigned unit number.




