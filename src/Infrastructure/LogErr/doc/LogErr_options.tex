% $Id: LogErr_options.tex,v 1.26 2011/07/01 17:56:11 rokuingh Exp $

% Earth System Modeling Framework
% Copyright 2002-2011, University Corporation for Atmospheric Research, 
% Massachusetts Institute of Technology, Geophysical Fluid Dynamics 
% Laboratory, University of Michigan, National Centers for Environmental 
% Prediction, Los Alamos National Laboratory, Argonne National Laboratory, 
% NASA Goddard Space Flight Center.
% Licensed under the University of Illinois-NCSA License.

\subsubsection{ESMF\_LOGERR\_PASSTHRU}
\label{const:logerrpassthru}

A named constant that is used with the
{\tt ESMF\_LogFoundError()} routine to pass a return code through
to a Log message.

\subsubsection{ESMF\_LOGKIND}
\label{const:logkindflag}

{\sf DESCRIPTION:\\}
Specifies a single log file, multiple log files (one per PET), or no log files.

The type of this flag is:

{\tt type(ESMF\_LogKind\_Flag)}

The valid values are:
\begin{description}
   \item [ESMF\_LOGKIND\_SINGLE] 
         Use a single log file, combining messages from all of the PETs.  Not supported on some platforms.
   \item [ESMF\_LOGKIND\_MULTI]
         Use multiple log files --- one per PET.  (Default.)
   \item [ESMF\_LOGKIND\_NONE]
         Do not issue messages to a log file.
\end{description}

\subsubsection{ESMF\_LOGMSG}
\label{const:logmsgflag}

{\sf DESCRIPTION:\\}
\begin{sloppypar}
Specifies what message level --- e.g., informational, warning, 
error --- will be written to log files.  This may be used both when
writing the message with {\tt ESMF\_LogWrite()}, and when setting the
{\tt logmsgList} option with {\tt ESMF\_LogSet()}.
\end{sloppypar}

The type of this flag is:

{\tt type(ESMF\_LogMsg\_Flag)}

The valid values are:
\begin{description}
   \item [ESMF\_LOGMSG\_INFO] 
         Write or allow informational messages.
   \item [ESMF\_LOGMSG\_WARNING]
         Write or allow warning messages.
   \item [ESMF\_LOGMSG\_ERROR]
         Write or allow error messages.
   \item [ESMF\_LOGMSG\_TRACE]
         Write or allow trace messages, when tracing is turned on.
\end{description}




