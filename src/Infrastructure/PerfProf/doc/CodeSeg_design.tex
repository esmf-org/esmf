% $Id: CodeSeg_design.tex,v 1.1 2002/11/14 18:25:45 cdeluca Exp $
%
% Earth System Modeling Framework
% Copyright 2002-2003, University Corporation for Atmospheric Research, 
% Massachusetts Institute of Technology, Geophysical Fluid Dynamics 
% Laboratory, University of Michigan, National Centers for Environmental 
% Prediction, Los Alamos National Laboratory, Argonne National Laboratory, 
% NASA Goddard Space Flight Center.
% Licensed under the GPL.

%\subsection{Design}

The HWMonitor class uses Hardware monitoring libraries (PCL or PAPI)
to do performance profiling of code segments. Floating point operations
per second, data cache utilization and miss rates, floating point unit
utilization, and cycle counts are the desired statistics to be reported.
The HWMonitor class is meant to be created by the CodeSeg
class. There are three important methods for HWMonitor: Init, BeginMonitoring,
EndMonitoring, and Report. The Init method does the initialization needed
to figure out which hardware monitoring counters need to be used and verify
that they are available. It also prepares arrays for the counters that
will be activated. BeginMonitoring turns performance monitoring on, for
a given code segment. EndMonitoring turns performance monitoring off, for
a given code segment. The Report method then gives a report on the statistics
for the code segment timed.


