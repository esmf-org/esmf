% $Id: DistGrid_implnotes.tex,v 1.1 2003/10/14 18:10:36 nscollins Exp $

\subsection{Implementation Notes}
\begin{enumerate}

\item {\bf Local verses Global Data}

The primary purpose of {\tt ESMF\_DistGrid} is to encapsulate information
about the local decomposition(s) ({\tt DE}) of the Grid on this PE.  
This includes such information as the total 
number of local cells, if logically rectangular the numbers of cells along 
each dimension, and the relative location of this {\tt DE} compared to the
overall {\tt ESMF\_Grid}.  The minimum information required would be
to compute and store data only for the local {\tt DE}.

However, at create time
{\tt ESMF\_DistGrid} computes information not only about the local
decomposition, but also less detailed information
about the other decompositions for the entire Grid.
While this duplicates some data, it avoids communication 
when a {\tt DE} requires information to enable it to send data to 
or receive data from other {\tt DE}s, 

\item {\bf Boundary Cells}

As part of the create-time computation {\tt ESMF\_DistGrid} computes sizes 
and lengths for the local DE grid cells, and also does a secondary computation 
of sizes and lengths taking into account a layer of boundary cells around 
each {\tt DE}.  These boundary cells are distinct from the {\tt halo} cells
which are specified on a per-Field basis and are visible to the user code.

The boundary cells inside {\tt ESMF\_DistGrid} are only used internally
to the Framework, for example during regridding to avoid unnecessary
inter-DE communication and to handle exterior boundaries in a consistent 
manner.

\end{enumerate}












