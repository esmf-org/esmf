% $Id$

% Earth System Modeling Framework
% Copyright (c) 2002-2025, University Corporation for Atmospheric Research, 
% Massachusetts Institute of Technology, Geophysical Fluid Dynamics 
% Laboratory, University of Michigan, National Centers for Environmental 
% Prediction, Los Alamos National Laboratory, Argonne National Laboratory, 
% NASA Goddard Space Flight Center.
% Licensed under the University of Illinois-NCSA License.

\subsubsection{ESMF\_VMEPOCH}
\label{const:vmepoch_flag}

{\sf DESCRIPTION:\\}
Specifies the kind of VM Epoch being entered.

The type of this flag is:

{\tt type(ESMF\_VMEpoch\_Flag)}

The valid values are:
\begin{description}
  \item [ESMF\_VMEPOCH\_NONE] 
    An epoch wihout special behavior.
  \item [ESMF\_VMEPOCH\_BUFFER]
    This option must only be used for parts of the code with distinct sending
    and receiving PETs, i.e. where no PETs are both sender and receiver.
    All non-blocking messages are being buffered. A single message is sent
    between unique pairs of src-dst PETs. This can significantly improve
    performance for cases with a large imbalance in the number of sending
    versus receiving PETs. The extra buffering also improves the overall
    asynchronous behavior between the sending and receiving side.
\end{description}
