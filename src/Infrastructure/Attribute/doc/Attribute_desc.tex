% $Id: Attribute_desc.tex,v 1.35 2010/08/18 16:53:56 murphysj Exp $
%
% Earth System Modeling Framework
% Copyright 2002-2010, University Corporation for Atmospheric Research,
% Massachusetts Institute of Technology, Geophysical Fluid Dynamics
% Laboratory, University of Michigan, National Centers for Environmental
% Prediction, Los Alamos National Laboratory, Argonne National Laboratory,
% NASA Goddard Space Flight Center.
% Licensed under the University of Illinois-NCSA License.

The Attribute class is used to hold the metadata for other ESMF objects.  This class can be used to build Attribute hierarchies which connect the Attributes of different ESMF classes.  The class is also capable of allowing the representation of standard Attribute packages for a more unified description of an object.  All Attributes are consistent across the entire virtual machine of the object to which they are attached.  

\subsubsection{Attribute Representation in ESMF}

Attributes are meant to be used as a tool for the user to help internally document their project.   Several ESMF objects can have Attributes associated with them, these objects are the following:

\begin{itemize}
\item Array
\item ArrayBundle
\item CplComp
\item GridComp
\item DistGrid
\item FieldBundle
\item Field
\item Grid
\item State
\end{itemize}

Each Attribute contains a name-value pair in which the value can be any of several numeric, character, and logical types.  See table \ref{table:attTypes} for the available Attribute value types.  All Attributes also contain character strings specifying the convention, purpose, and object type of the Attribute for identification purposes - each of which is initialized as an empty string until specified otherwise.  Each Attribute can be uniquely identified by its name, convention, purpose within any one ESMF object.  

All Attributes contain three vectors of pointers to other Attributes, which are empty until specified otherwise.  These vectors of Attribute pointers hold the Attributes, Attribute packages, and Attribute links.  This feature is what allows the Attribute class to self assemble complex structures for representing and organizing the metadata of an ESMF object hierarchy.

\subsubsection{Attribute Hierarchies}

Of the ESMF objects with Attributes, only some can link their Attributes together in an Attribute hierarchy.  These objects are:

\begin{itemize}
\item CplComp
\item GridComp
\item State
\item Field
\item FieldBundle
\item Array
\item ArrayBundle
\end{itemize}

The most common use for this capability is for linking the Attributes of a Field to the FieldBundle which holds it, which is then linked to the State that is used to transport all of the data for a Component.  All of these links, with the exception of the link between the Component and the State, are automatically handled by ESMF.  In addition, the State will automatically set the {\it import} and {\it export} boolean valued Attributes that are part of the ESMF supplied standard Attribute package for Field when that Field is added to the State. 

Attribute hierarchies are linked in a "shallow" manner, meaning that the Attributes belonging to an external object are not copied, they are merely referenced by a pointer.  This is important to ensure that the Attribute hierarchy has a one-to-one correspondence with the object hierarchy.  


\subsubsection{Table of Available Attributes}

The following is an alphabetical list of all the attributes impletmented in ESMF, their definitions, and which packages they are contained within.