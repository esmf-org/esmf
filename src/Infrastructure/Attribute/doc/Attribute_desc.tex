% $Id: Attribute_desc.tex,v 1.45 2011/05/10 23:27:36 rokuingh Exp $
%
% Earth System Modeling Framework
% Copyright 2002-2011, University Corporation for Atmospheric Research,
% Massachusetts Institute of Technology, Geophysical Fluid Dynamics
% Laboratory, University of Michigan, National Centers for Environmental
% Prediction, Los Alamos National Laboratory, Argonne National Laboratory,
% NASA Goddard Space Flight Center.
% Licensed under the University of Illinois-NCSA License.


The ESMF Attribute class is a metadata utility that supports emerging standards 
in a flexible way.  Metadata, which is data about data, is broken down into 
name-value pairs by the Attribute class.  Attributes can be attached at any 
level of the ESMF object hierarchy, and they can be grouped into packages to 
help organize model metadata.  

Most of the ESMF deep objects can hold metadata in the form of an Attribute 
hierarchy.  The objects that can hold Attributes of various types are listed in 
the following sections.  The Attribute class is useful for documenting data 
provenance and encourages models to be more self describing.  Attributes can 
also be used for some aspects of model execution and coupling.

Standardized metadata is organized in ESMF by Attribute packages.  These 
packages are used to aggregate, store, and output model metadata.  They can be 
nested inside each other to make larger organized package, distributed across 
processors and updated at runtime, and expanded to suite specific needs.  These 
Attribute packages are designed around accepted metadata standards such as: 
climate and forecast (CF), ISO standards, and METAFOR Common Information Model 
(CIM).

\subsubsection{The ESMF approach to Attributes}

ESMF's Approach to Attributes can be summarized as follows:

\begin{itemize}
  \item Implement community standards where they exist
  \item Associate Attributes with the ESMF object they describe. Currently, the following ESMF objects can have Attributes:
  \begin{itemize}
     \item Array
     \item ArrayBundle
     \item CplComp
     \item GridComp
     \item DistGrid
     \item Field
     \item FieldBundle
     \item Grid
     \item State
     \end{itemize}
  \item Establish pre-defined Attribute packages (see Section \ref{sec:AttPacks}) to make Attribute creation easier for the user.
  \item Allow for user-defined Custom Attribute packages (see Section \ref{sec:CustomAttPacks}).
  \item Enable the nesting of Attribute packages (see Section \ref{sec:AttPackNesting}) including Custom packages.
  \item Enable complex Attribute heirarchies (see Section \ref{sec:AttHier}.
  \item Export Attributes in more than one format (see Section \ref{sec:AttributeExports}).
  \item Ensure that all Attributes are consistent across the entire virtual machine of the object to which they are attached.
\end{itemize}

\subsubsection{Attribute hierarchies}
\label{sec:AttHier}

Of the ESMF objects with Attributes, only some can link their Attributes together in an Attribute hierarchy.  These objects are:

\begin{itemize}
\item CplComp
\item GridComp
\item State
\item Field
\item FieldBundle
\item ArrayBundle
\end{itemize}

The most common use for this capability is for linking the Attributes of a Field to the FieldBundle which holds it, which is then linked to the State that is used to transport all of the data for a Component.  All of these links, with the exception of the link between the Component and the State, are automatically handled by ESMF. Additionally, the State will automatically set the {\tt VariableIntent} Attribute that is part of the ESMF supplied standard Attribute package for Field when that Field is added to the State.  {\tt VariableIntent} will be set to either {\tt Export} or {\tt Import}.

Attribute hierarchies are linked in a ``shallow'' manner, meaning that the Attributes belonging to an external object are not copied, they are merely referenced by a pointer.  This is important to ensure that the Attribute hierarchy has a one-to-one correspondence with the object hierarchy.  
