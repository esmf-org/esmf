% $Id: Attribute_implnotes.tex,v 1.4 2008/09/09 20:03:04 rokuingh Exp $
%
% Earth System Modeling Framework
% Copyright 2002-2008, University Corporation for Atmospheric Research,
% Massachusetts Institute of Technology, Geophysical Fluid Dynamics
% Laboratory, University of Michigan, National Centers for Environmental
% Prediction, Los Alamos National Laboratory, Argonne National Laboratory,
% NASA Goddard Space Flight Center.
% Licensed under the University of Illinois-NCSA License.

This section covers the current development on Attribute consistency in a distributed environment and Attribute memory deallocation.  The {\tt ESMF\_Attribute} class currently has limited ability to communicate the current state of an Attribute hierarchy, which is only available in the initialize phase of a model component.  Upcoming design and development involves a routine that will allow the state of the Attribute hierarchy to updated during the run phase of a model component.  Issues and procedures dealing with Attribute memory deallocation are also discussed.

\subsubsection{Attributes in a Distributed Environment} 
\label{sec:Att:Dist}

The  {\tt ESMF\_Attribute} class is consistent with the other ESMF objects in the context of reconciling across exclusive sets of PETs.  The {\tt ESMF\_StateReconcile()} call is used to create a consistent view of ESMF objects on exclusive PETs in the initialization phase of a model run.  All Attributes that are attached to an ESMF object contained in the State, i.e. an object that is being reconciled, are also reconciled.  This is done by making a deep copy of the {\it entire} Attribute hierarchy below the point at which the reconcile is taking place.  This means that, at the conclusion of {\tt ESMF\_StateReconcile()} there are many partial duplicate copies of the Attribute hierarchy.

These partial copies of the Attribute hierarchy do not present a problem with respect to resource management, as they are all considered proxy objects, which are cleaned up by the State when {\tt ESMF\_StateDestroy()} is called.  These partial copies will not conflict with each other, either, as they are all independent and exclusively managed by their respective proxy objects.  The caveat is that these partial copies are permanent, after they are created they cannot be changed, or updated.

The need to update these partial Attribute hierarchies is an area of current development and design.  Another caveat with the current design is that Attributes which are attached directly to an {\tt ESMF\_State} are not included when the State is reconciled.  This is also an area of current development.  The complete construction of a consistent view of the Attribute hierarchy in the initialization phase of a model run, and the ability to update the Attribute hierarchy throughout the run phase are two main goals for the continued development of the {\tt ESMF\_Attribute} class.

\subsubsection{Attribute Memory Deallocation}

The {\tt ESMF\_Attribute} class presents a somewhat different paradigm with respect to memory deallocation than other ESMF objects.  The {\tt ESMF\_AttributeRemove()} call can be issued to remove any single Attribute from an ESMF object or an Attribute package on an ESMF object.  This call is not enabled to remove Attributes in groups, such as an entire Attribute package with one call, at this time.  However, the user is not required to remove all Attributes that are used in a model run.  The entire Attribute hierarchy will be removed automatically by the ESMF, provided the ESMF objects which contain them are properly destroyed.  