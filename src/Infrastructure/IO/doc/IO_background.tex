% $Id: IO_background.tex,v 1.24 2010/09/24 05:59:18 samsoncheung Exp $

%\subsection{Background}

The ESMF IO provides an unified interface for input and output of
high level ESMF objects such as Fields.  The system is expected to
automatically detect file formats at runtime, and to output data in 
a variety of formats. In the current release, the ESMF IO capability
is integrated with third-party software such as, 
\htmladdnormallink{Parallel IO (PIO)}{http://code.google.com/p/parallelio/}
to read and write Fortran array data in binary/NetCDF format and 
\htmladdnormallink{Xerces}{http:xerces.apache.org/xerces-c/} 
Library to read and write Attribute data in XML format.  Other file IO
functionalities, such as writing of error and log messages and input of
configuration parameters from an ASCII file, are not covered in this
document.


%\subsection{I/O architecture}
%
%The future development of ESMF IO would include an ESMF\_IO object that
%combines other ESMF objects, such as Components, States, FieldBundles,
%ArrayBundles, Fields, Arrays, Grids, as well as Array and Attributes to 
%perform robust and unified input/output system.
%


%\subsection{Data models}
%
%Earth system models use a variety of discrete grids to maintain information 
%about fields in continuous space, as well as observations. The primary ESMF 
%codes employ finite-difference and finite-volume grids, spectral grids, 
%unstructured land-surface grids, and ungridded observational networks.
%
%Fields within a model component are frequently defined on the same
%physical grid and are decomposed in memory in an identical fashion;
%that is, they share a distributed grid. They form a {\em bundle of
%fields} defined on the same distributed grid. The gridded data are
%supported by three ESMF elements: {\em PhysGrid} element 
%for physical grids, {\em DistGrid} element for distributed grids, and 
%{\em Fields} class for fields (\cite{ESMF-PhysGrid-Req},
%\cite{ESMF-DistGrid-Req}, \cite{ESMF-Field-Req}). 
%
%ESMF I/O will support input/output of data defined on all ESMF
%supported grids and location streams (\cite{ESMF-PhysGrid-Req},
%\cite{ESMF-DistGrid-Req}). For
%the purpose of this document, we will consider data belonging to three
%broad categories:
%
%\begin{description}
%
%\item[\bf Structured Gridded Data.] A {\em structured grid} is one on 
%which the relationship between gridpoints can be derived from their
%indices, without the need for an explicit map.  A simple example is fields
%defined on a rectangular lat/lon grid.
%
%\item[\bf Unstructured Gridded Data.] For the more general 
%{\em unstructured grid} the relationship between gridpoints cannot be
%derived from their indices, and the specification of an explicit map
%is necessary.  An example is a {\em catchment grid} used by some
%land-surface models.
%
%\item[\bf Observational Data on location streams.] As defined in 
%the {\em Physical Grid Requirements}, a location stream contains 
%a list of locations which 
%describe the measurements. Each observation is 
%associated with a spatial point or region. A neighbor relationship is not 
%defined for observations. 
%\end{description}
%
%As we have already mentioned, logically rectangular grids are naturally 
%represented by multi-dimensional arrays. The two latter data models can be 
%represented as one-dimensional arrays of structures with each structure 
%containing information about location, field values associated with this 
%location, and a list of neighbors, if relevant. 


\subsection{ESMF Attribute}

The metadata IO is handled via ESMF Attribute object. The third
party software Xerces Library is integrated with ESMF to provide
the ability to read and write Attribute data in XML file format.
To enable this capability, the environment variable ESMF\_XERCES should be
set. Details can be founds in ESMF User Guide, Third Party Libraries
in "Building and Installing ESMF.


\subsection{Data I/O}

ESMF provides parallel IO capability by integrating a third-party IO
software, Parallel I/O (PIO). This is a software interface layer, was 
first developed by the team under CCSM Software Engineering Group at NCAR
\cite{CESM_Site}. In the current release, the read and write of the 
contents in the following ESMF objects are performed by PIO:

\begin{description}
\item[-] ESMF\_Array (ref)
\item[-] ESMF\_ArrayBundle (ref)
\item[-] ESMF\_Field (ref)
\item[-] ESMF\_FieldBundle (ref)
\end{description}

Two formats are supported, namely, binary format and NetCDF format.  
It depends on the which environment variables are tuned on when ESMF is built,
the system is expected to automatically detect file formats at runtime.

Details of the environment variables can be founds in ESMF User Guide, 
Third Party Libraries in "Building and Installing ESMF.



\subsection{Data formats}

Several standard formats are currently used in Earth Science modeling
for input/output of data:

\begin{description}
\item[\bf NetCDF] Network Common Data Form (NetCDF) is an interface for 
array-oriented data access. The NetCDF library provides an
implementation of the interface. It also defines a 
machine-independent format for representing scientific data. Together,
the interface, library, and format support the creation, access, and
sharing of scientific data. The NetCDF software was developed at the
Unidata Program Center in Boulder, Colorado. See \cite{NetCDF3_UsersGuide_C}.
In geoscience, NetCDF can be naturally used for represenation of fields 
defined on logically rectangular grids. NetCDF use in geosciences is 
specified by CF conventions mentioned above \cite{NetCDF_CF_v1_beta3}. 

To the extent that data on unstructured grids (or even observations) can be 
represented as one-dimensional arrays, NetCDF can also be used to store these 
data. However, it does not provide a high-level abstraction for this type of 
data. 

\item[\bf IEEE Binary Streams]
A natural way for a machine to represent data is to use a native
binary data representation. There are two choices of ordering of bytes
(so-called {\it Big Endian} and {\it Little Endian}), and a lot of
ambiguity in representing floating point data. The latter, however, is
specified, if IEEE Floating Point Standard 754 is satisfied
(\cite{IEEE-Floating-Point}, \cite{Kahan-IEEE-754}). It is desirable
to be able to use efficient native representation, and optionally
provide ESMF metadata on a companion file using for example XML
\cite{XML-W3C}.

\end{description}

%Modern data management approaches could potentially provide significant 
%advantages in manipulating data and have to be carefully studied.
%For example, ESMWF has created and employed relational-database based 
%Observational Data Base (ODB) software \cite{ODB}.  However, such complex 
%data management systems are beyond the scope of the basic ESMF I/O. 




