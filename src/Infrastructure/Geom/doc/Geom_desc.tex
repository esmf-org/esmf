% $Id$
%
% Earth System Modeling Framework
% Copyright 2002-2022, University Corporation for Atmospheric Research, 
% Massachusetts Institute of Technology, Geophysical Fluid Dynamics 
% Laboratory, University of Michigan, National Centers for Environmental 
% Prediction, Los Alamos National Laboratory, Argonne National Laboratory, 
% NASA Goddard Space Flight Center.
% Licensed under the University of Illinois-NCSA License.

The ESMF Geom class is used as a holder for other ESMF geometry objects (e.g. an ESMF Grid). This allows a generic
representation of a geometry to be passed around (e.g. through a coupled system) without it's specific type being known.
Some operations (e.g. creating a Field) are supported on a Geom object and more will be added over time as needed. However, if
an unsupported operation is needed, then the specific geometry object can always be pulled out and operated on that way.

In addition to the geometry object a Geom can also contain information describing a location on a geometry, for example, in the case of
Grid a geometry object can contain a stagger location. Having this information allows the creation of Fields and other capabilities to
be performed in the most generic way on just a Geom object. The user doesn't need to specify this location information during the
creation of a Geom object. If they don't, then default values for this information are provided which match those which would
be used when when creating a Field with the specific geometry (e.g. stagger location ESMF\_STAGGERLOC\_CENTER for a Grid).
