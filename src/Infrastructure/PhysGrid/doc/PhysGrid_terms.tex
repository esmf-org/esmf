% $Id: PhysGrid_terms.tex,v 1.8 2002/04/24 20:43:30 cnh Exp $

\begin{description}

\item [Cell] \label{glos:Cell} A physical location that is specified by both
its extent (vertices) and a nominal central location.  Cells are basic to the
discrete description of continuous fields in a finite volume formulation.

\item [Global physical grid] \label{glos:GlobGrid} A grid containing physical
information about the entire, undecomposed domain.  No DistGrid need be
associated with a global physical grid.

\item [Grid] An ESMF Grid. This is an ESMF General Design Element
(Dsgn\_GENn.m), discussed in its own Requirements and  Design documents. It
contains all physical grid (via the PhysGrid facilty) and memory organization
and layout information (via the DistGrid facility) required to manipulate
consistently defined Gridded Fields, as well as to create and operate with Grid
Transforms.

\item [Location stream] \label{glos:LocStream} A structure containing a list of
locations with no assumed relationship between these locations, save that the
types of information in each element of the list is similar.

\item [Logically rectangular grid] \label{glos:RecGrid} A grid in which
physically adjacent points have sequential indices, and in which the extent of
each index is independent of the other indices.

\item [Metrics] \label{glos:Metrics} Terms relating changes in nondimensional
index space to physical distances.

\item [Physical location] \label{glos:PhysLoc} The location in physical space 
to which data pertain.  A physical location may be a point, a two-dimensional 
area or a three dimensional volume.

\end{description}

\subsection{ESMF physical grid notation}

The symbol $G$ is used to refer to a physical grid. A physical grid, $G$, 
consists of a series of zero or more distinct sets of mappings between index space
and physical coordinate space. A set of mappings, $g$, consists of members
that map between an index space of $n$-dimensional indices
$\underline{i}_{g}$ and a location space of 
$m$-dimensional locations $\underline{p}_{g}$, so that
\begin{equation}
G = g_{1}, g_{2} \ldots g_{n}
\end{equation}
\noindent where
\begin{equation}
g_{n}: \underline{i} \mapsto \underline{p},~~~
\begin{cases}
  \underline{i} \subset i_{g} & \text{and}, \\
  \underline{p} \subset p_{g} & \text{}
\end{cases}
\end{equation}
The mappings $\mapsto$ may be either numerical or functional.
The values of $\underline{i}$ are elements from the valid index space
$i_{g}$ for the mapping set $g$, while the elements $\underline{p}$ are elements
from the valid physical grid coordinate space $p_{g}$ for $g$.
In a grid point physical grid $\underline{i}$ could be
an index with $n$ dimensions, while $\underline{p}$ could
be a coordinate of $m$ dimensions. The sizes of 
$n$ and $m$ will be depend on the type of physical 
grid and on the mapping operator $g$. For example, an ungridded 
grid could be represented with a one-dimensional index space
,for which $n=1$,
which maps to a three-dimensional physical space, for which $m=3$.
For spectral codes a physical grid can be used both
to hold physical coordinate grids onto which inverse transforms
convert and to hold information on which wave numbers map
to which index space coordinates in spectral space.
If a grid is not modified then the mapping for a particular
$\underline{i}$ will always return the same $\underline{p}$.
However, there may be multiple $\underline{i}$ values that
can return the same $\underline{p}$ value.
