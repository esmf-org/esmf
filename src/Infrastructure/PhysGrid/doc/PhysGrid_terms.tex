% $Id: PhysGrid_terms.tex,v 1.4 2002/04/02 01:28:10 rhallberg Exp $

\begin{description}

\item [Cell] \label{glos:Cell} A physical location that is specified by both
its extent (vertices) and a nominal central location.  Cells are basic to the
discrete description of continuous fields in a finite volume formulation.

\item [Global physical grid] \label{glos:GlobGrid} A grid containing physical
information about the entire, undecomposed domain.  No DistGrid need be
associated with a global physical grid.

\item [Grid] An ESMF Grid. This is an ESMF General Design Element
(Dsgn\_GENn.m), discussed in its own Requirements and  Design documents. It
contains all physical grid (via the PhysGrid facilty) and memory organization
and layout information (via the DistGrid facility) required to manipulate
consistently defined Gridded Fields, as well as to create and operate with Grid
Transforms.

\item [Location stream] \label{glos:LocStream} A structure containing a list of
locations with no assumed relationship between these locations, save that the
types of information in each element of the list is similar.

\item [Logically rectangular grid] \label{glos:RecGrid} A grid in which
physically adjacent points have sequential indices, and in which the extent of
each index is independent of the other indices.

\item [Metrics] \label{glos:Metrics} Terms relating changes in nondimensional
index space to physical distances.

\item [Physical location] \label{glos:PhysLoc} The location in 2-dimensional
(horizontal) or 3-dimensional physical space to which data pertain.  A
physical location may be a point or region in either the horizontal or the
vertical.

\end{description}



