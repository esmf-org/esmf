% $Id: PhysGrid_req.tex,v 1.2 2002/03/07 02:49:02 cnh Exp $
%===============================================================================
\req{Physical Locations}
%-------------------------------------------------------------------------------

A mechanism shall be provided for describing physical locations in space in 1
(vertical), 2 (horizontal) or 3 dimensions, including both specification of
points and of ranges. 


\sreq{Horizontal Locations}

\ssreq{Horizontal Coordinates}

Physical domains may use Cartesian or spherical coordinate systems in the
horizontal directions.  Units for these coordinates are meters and degrees of
latitude and longitude.

\ssreq{Horizontal Locations may be points}

   Horizontal locations may be specified as a pair of real values in the order
(X,Y) or (longitude, latitude).


\ssreq{Horizontal Locations may be polygonal regions}

  Horizontal Locations may be specified to be regions by providing the number
of vertices and the list of the vertex points.  Vertex points must be specified
either clockwise or counterclockwise around the region.  The vertex points
may be redundant.
 
\ssreq{Horizontal regions may have central points}

  Both a central point and a region may be specified in describing a horizontal
location.  The points may provide a convenient nominal location, even when
a value actually pertains to a region.

\ssreq{Horizontal Locations may have radii of influence}

  A horizontal location may be specified by adding a nominal radius of
influence to the central point.  This may be the radius of a Gaussian
distribution of influence.


\ssreq{Paths between region vertices may be specified} 

The faces of polygonal regions in sperical coordinates may be linear in
latitude-longitude space or may follow great-circle routes between vertices, or
may follow user-defined curves.  There must be an exact definition of control
volumes for calculating horizontal areas and ensuring that the surface is tiled
by horizontal regions. The former is traditionally used in spherical
coordinates and is particularly simple for analytically calculating the areas
of regions, but may be ill defined near the poles.  The grid associated with a
component model may require some other specification for optimal numerical
performance.

\sreq{Vertical Locations}

\ssreq{Vertical Coordinates}

Physical domains may use a variety of vertical coordinates, including pressure,
height, density, isotherms, sigma, other terrain-following, or any other
vertically monotonic quantity.  In addition, a user-interpretable vertical
proxy (such as a satellite measurement channel) may be used.  Units of this
coordinate must be self-consistent.  (See the CF convention for a full
discussion of options for vertical coordinates.)

\ssreq{Vertical Locations may be points}

\ssreq{Vertical Locations may be regions}

Vertical locations may be specified by providing the values of the top and
bottom bounding points.  Such regions have the same extent regardless of the
order in which the bounding points are specified.

\ssreq{Vertical regions have central points}

  Both a central point and a region may be specified in describing a vertical
location.  The points may provide a convenient nominal location, even when
a value actually pertains to a region.

%===============================================================================
\req{Physical Grids}
%-------------------------------------------------------------------------------

Physical Grids provide the locations of each of the cells/points associated with
the range of indices in a distributed grid.  Each PhysGrid is associated with a
single Dist grid.  Physical grids may be purely horizontal, purely vertical, or
both.  Structured grids assume that adjacent locations in index space share
boundaries in a predicable way.

\sreq{Reading grids}

Grids can be read from standard files.

\sreq{Writing grids}

Grids are output to standard files.

\sreq{PhysGrids may be internally generated}

For an arbitrary number of points in the global domain of the associated
DistGrid, it shall be possible to specify an algorithm for internally
determining the PhysGrid.

\sreq{Cell specification}

PhysGrids shall specify both the locations of cell vertices, and the locations
of cell centers.

\sreq{Interpolation}

A PhysGrid may be interpolated to generate an equivalent PhysGrid on a coarser
or finer global DistGrid.  Methods should be provided to accomplish such
interpolation.

\sreq{Horizontal coordinate independent of vertical}
Horizontal grid locations can be assumed independent of the vertical coordinate.

\sreq{Vertical coordinate potentially dependent on horizontal}

Vertical grid locations may be functions of the horizontal coordinate, or may be
independent of it.  This is necessary to support, for example, partial cells in
Z-coordinate ocean models.

\sreq{Horizontal physical grids}

\ssreq{PhysGrids map projections}

PhysGrids may be generated from a number of standard map projections,
including traditional and Mercator grids on a sphere, cubed-sphere, or tripolar
grids.

\ssreq{Supported topologies} Supported horizontal grid topologies will include
logically rectangular grids that are reentrant in 0, 1, or 2 directions,
northern and southern tripolar (Murray 1996), sphere, icosahedral, and
unstructured grids.  Unstructured arrays of logically rectangular grids
(cubed-sphere and arbitrary nesting) will also be supported.

\ssreq{PhysGrid topology consistency checking}

A mechanism shall be provided to verify that the locations of the points in
a PhysGrid are consistent with the topology of the underlying DistGrid.  An
exception shall be generated in case of inconsistency.

\ssreq{Areas tile sphere}
Grid areas may be calculated using algorithms that guarantee that the grid
exactly tiles the sphere (or a portion of it).

\ssreq{Available subgrids}

For locally quadrilateral horizontal grids, information shall be available for
each of the 4 related subgrids.  That is if a t-cell is centered at a tracer
point,  cells centered on the east face, north face, and northeast corner of
the t-cell will also be provided in the case of a NorthEast underlying
distributed grid.

\sreq{Horizontal Functional Representations}

A spectral horizontal description may be used.  More generally, the horizontal
structure of information may be given by specifying functional decompositions.

\ssreq{Horizontal Fourier grids} 

Cartesian Fourier grids will be supported.  Associated with this grid are the
wavenumbers (in units of $m^{-1}$) of each of the elements on the grid.

\ssreq{Horizontal spherical harmonics grids} 

Spherical harmonics grids will be supported.  Associated with this grid are the
wavenumbers (nondimensional m,n) of each of the elements on the grid.  At a
minimum, rhomboidal and triangular trunctations will be supported.

\ssreq{Extensible horizontal functional representations}

The PhysGrid design should not preclude the user from using alternative
functional horizontal representations, such as spectral elements.

\begin{reqlist}
{\bf Priority:} <Priority 1-3> \\
{\bf Source:} <JMC Code (required, desired)> \\
{\bf Status:} <Proposed, Approved-1, Approved-2, Rejected, Implemented, Verified> \\
{\bf Verification:} <e.g., Code inspection, Unit test, System test> \\
{\bf Notes:} <Background, comments on design, implementation, etc.> 
\end{reqlist}
%===============================================================================
\req{Grid Metrics}
%-------------------------------------------------------------------------------

Grid metrics are all of the lengths (or partial derivatives of distances with
index number) and related quantities required to do a variety of calculations. 
All metrics are a function of the grid and must be static with time.

\sreq{Calculation of metrics}
All metrics may be calculated from grid locations.

\sreq{Reading metrics}
All metrics may be read from a standard grid file.

\sreq{MKS metric units}
Metrics have units of m or $m^2$, or other appropriate MKS units.

\sreq{Available Metrics}
Available metric information includes an extensive list of grid lengths, cell
areas, and the angle between logical and physical north.  (This list will be
developed as a part of the requirement.)

\sreq{Methods for calculating metrics}
Metrics may be calculated by either standard or user-provided algorithms. 


% 
% %===============================================================================
% % Requirements may be itemized under a main topic:
% %===============================================================================
% %===============================================================================
% \req{<Requirement>}
% %-------------------------------------------------------------------------------

% <Description>

% \sreq{<Subrequirement>}

% <Description>

% \begin{reqlist}
% {\bf Priority:} <Priority 1-3> \\
% {\bf Source:} <JMC Code (required, desired)> \\
% {\bf Status:} <Proposed, Approved-1, Approved-2, Rejected, Implemented, Verified> \\
% {\bf Verification:} <e.g., Code inspection, Unit test, System test> \\
% {\bf Notes:} <Background, comments on design, implementation, etc.> 
% \end{reqlist}

% %===============================================================================
% % Or requirements may be itemized under a subtopic:
% %===============================================================================
% %===============================================================================
% \req{<Requirement>}
% %-------------------------------------------------------------------------------

% <Description>

% \sreq{<Subrequirement>}

% <Description>

% \ssreq{Subsubrequirement} 

% <Description>

% \begin{reqlist}
% {\bf Priority:} <Priority 1-3> \\
% {\bf Source:} <JMC Code (required, desired)> \\
% {\bf Status:} <Proposed, Approved-1, Approved-2, Rejected, Implemented, Verified> \\
% {\bf Verification:} <e.g., Code inspection, Unit test, System test> \\
% {\bf Notes:} <Background, comments on design, implementation, etc.> 
% \end{reqlist}









