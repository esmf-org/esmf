% $Id: PhysGrid_background.tex,v 1.3 2002/04/02 01:18:23 rhallberg Exp $

\section{Background}


  Earth system models use a variety of different discrete grids to represent
continuous physical space. The ESMF Physical Grid (PhysGrid) element is 
responsible for maintaining information about physical coordinate based 
discretizations of simulation domains. Components may include and use 
additional grid information internally, however, the only physical grid 
information that framework operates on will come through PhysGrids. 

  The information that the framework needs to represent in PhysGrid is  quite
extensive. Coupled components need to be able to provide PhysGrid  information
that is sufficiently detailed for regridding operators (see Regrid) which
transfer field data between the components.  The primary ESMF codes employ
finite-difference and finite-volume grids, spectral grids, unstructured
land-surface grids and ungridded observational networks.  This requires
PhysGrid to support an extensive range of metric terms (e.g. grid spacings,
areas and volumes) and grid masks.  The ungridded observational networks are
supported through the PhysGrid Location Stream facilities.

\subsection{Location}

  PhysGrid is a part of the "Fields and Grids" portion of the infrastructure.
PhysGrid interacts closely with the index spaces in the Distributed Grid
facility (DistGrid); the creation of a DistGrid precedes the creation of a
PhysGrid.  Most Fields exist at locations described via PhysGrid.  Regrid
typically uses PhysGrid information.

\subsection{Scope}

PhysGrid does not cover domain decomposition or specification of grid
topology; these are covered by DistGrid.









