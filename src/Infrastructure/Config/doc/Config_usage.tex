% $Id: Config_usage.tex,v 1.9 2008/01/14 19:24:32 murphysj Exp $

 \subsubsection{Resource Files}

   A {\em Resource File} is a text file consisting of list of {\em label}- 
   {\em value} pairs. The list can be of variable length (records).
   Each {\em label}) should be followed by some data, the {\em value}. 
   A simple resource file looks like this:

 \begin{verbatim}
 # Lines starting with # are comments which are
 # ignored during processing.
 my_file_names:         jan87.dat jan88.dat jan89.dat
 radius_of_the_earth:   6.37E6  # these are comments too
 constants:             3.1415   25
 my_favourite_colors:   green blue 022 # text & number are OK
 \end{verbatim}

    In this example, {\tt my\_file\_names:} and {\tt constants:}
    are labels, while {\tt jan87.dat, jan88.dat} and {\tt jan89.dat} are
    data associated with label {\tt my\_file\_names:}.
    Resource files can also contain simple tables of the form:

 \begin{verbatim}
 my_table_name::
  1000     3000     263.0   
   925     3000     263.0
   850     3000     263.0
   700     3000     269.0
   500     3000     287.0
   400     3000     295.8
   300     3000     295.8    
 ::
 \end{verbatim}

 Resource files are intended for random access (except between ::'s in a 
 table definition). This means that order in which a particular {\em label}- 
 {\em value} pair is retreived is not dependent upon the original order 
 of the pairs. The only exception to this, however, is when the same {\em label} appears 
 multiple times within the Resource file.




