% $Id: Route_options.tex,v 1.1 2005/02/28 20:10:26 nscollins Exp $

\subsubsection{ESMF\_RouteOptions}

\label{opt:routeopt}
{\sf DESCRIPTION:\\}
Specifies control options when executing the communication
represented by a Route object.  Normally these do not need to
be set by the user, but can be if the best communication strategy
is known in advance.   The synchronous and asychronous options 
are mutually exclusive; and the four packing options are also
mutually exclusive.  Setting the Route options is "sticky"; it
maintains the last value set until explicitly changed.

Valid values are:
\begin{description}
    \item [ESMF_ROUTE_OPTION_ASYNC]
	Execute the Route asychronously.
    \item [ESMF_ROUTE_OPTION_SYNC]
	Execute the Route sychronously.
    \item [ESMF_ROUTE_OPTION_PACK_PET]
        Pack all data from or to another PET into a single buffer
        when sending or receiving.
    \item [ESMF_ROUTE_OPTION_PACK_XP]
        Pack all data from each non-contiguous exchange packet 
        into a single buffer when sending or receiving.
    \item [ESMF_ROUTE_OPTION_PACK_NOPACK]
        Do no buffering; send each contiguous run of data as a distinct
        communications operation.
    \item [ESMF_ROUTE_OPTION_PACK_VECTOR]
        Use the MPI type vector interfaces to send non-contiguous data
        which has regular strides when sending or receiving.
    \item [ESMF_ROUTE_OPTION_DEFAULT]
	Use the system default for communication, which is the
        combination of {\tt ESMF\_ROUTE\_OPTION\_SYNC} and
        {\tt ESMF\_ROUTE\_OPTION\_PACK\_XP}.
\end{description}







