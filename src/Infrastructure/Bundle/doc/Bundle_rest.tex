% $Id: Bundle_rest.tex,v 1.9 2004/09/23 21:44:59 nscollins Exp $

\label{sec:bundlerest}

\begin{enumerate}
\item{\bf No mathematical operators.}
The Bundle class does not support differential or other
mathematical operators.  We do not anticipate providing this 
functionality in the near future.

\item{\bf Limited validation and print options.}
We are planning to increase the number of validity checks available
for Bundles as soon as possible.  We also will
be working on print options.

\item{\bf Limited communication support.}
Only a subset of the communication routines are currently supported
for Bundles.  For those routines not implemented yet the user can
loop over the Fields in the Bundle and call the Field level 
communication routines instead.

\item{\bf Packed data not supported.}
One of the options that we are currently working on for Bundles is
packing.  Packing means that the data from all the
Fields that comprise the Bundle are copied into a single Array and
manipulated collectively.  This operation can be done without 
destroying the original Field data.  Packing is being designed to 
facilitate optimized regridding, data communication, and IO operations.  
It will be possible to collectively manipulate all the Fields within 
a Bundle at once, rather than operating on each Field separately.  
This will reduce the latency overhead of the communication.  

\item{\bf Interleaving Fields within a Bundle.}
Data locality is important for performance on some computing
platforms.  An interleave option will allow the user to create
a packed Bundle in which Fields are either concatenated in memory
or in which Field elements are interleaved.

\end{enumerate}




