% $Id: Bundle_rest.tex,v 1.8 2004/06/15 23:23:53 cdeluca Exp $

\label{sec:bundlerest}

\begin{enumerate}
\item{\bf No mathematical operators.}
The Bundle class does not support differential or other
mathematical operators.  We do not anticipate providing this 
functionality in the near future.

\item{\bf Limited validation and options.}
We are planning to increase the number of validity checks available
for Bundles as soon as possible, foremost the ability to check 
that the Fields it contains are on the same Grid.  We also will
be working on print options.

\item{\bf Limited collective operations.}
The Bundle class does not automatically support collective field 
operations.  In order to perform the same operation on Fields within
a Bundle, you must currently loop over a query for each Field and
call the method on each Field in turn.  We expect to have 
interfaces for non-optimized collective operations such as 
regridding, data communication, and IO shortly.

One of the options that we are currently working on for Bundles is
packing.  Packing means that the data from all the
Fields that comprise the Bundle are copied into a single Array and
manipulated collectively.  This operation can be done without 
destroying the original Field data.  Packing is being designed to 
facilitate optimized regridding, data communication, and IO operations.  
It will be possible to collectively manipulate all the Fields within 
a Bundle at once, rather than operating on each Field separately.  
This will reduce the latency overhead of the communication.  

\item{\bf Interleaving Fields within a Bundle.}
Data locality is important for performance on some computing
platforms.  An interleave option will allow the user to create
a packed Bundle in which Fields are either concatenated in memory
or in which Field elements are interleaved.

\end{enumerate}




