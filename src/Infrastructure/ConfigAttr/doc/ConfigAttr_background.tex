% $Id: ConfigAttr_background.tex,v 1.2 2002/06/08 02:37:06 cnh Exp $

\section{Background}
As part of their state Earth system models need to maintain ad-hoc sets of
parameters. These parameters can be related to physical terms (for example
mixing coefficients, length scales or time scales) or can be related
to computational aspects (for example directory names, output and input
locations). Some of these parameters will be explicitly set by user
input at runtime, others may be automatically determmined from system
defaults or from other utilities. The parameters may be single valued
or may be multi-dimensional. Parameters can be of various types float,
integer, logical, string.

The configuration attributes element of ESMF will enable
these parameters or {\bf attributes} to be recorded and provided services
for setting attributes from human readbale text files and for saving attributes
to persistent storage in appropriate formats.

\subsection{Location}

The configuration attributes element of ESMF will be part of the
Infrastructure/Utilities area. Other upper layer tools may use the
configuration attributes internally. User-level component code will
also use the configuration attributes element.

\subsection{Scope}

Compile time and runtime setting of parameters will initially be through
text based specification. Extensive automatic systems for choosing parameters or GUI
based systems for manipulating parameters are not required within the 
initial ESMF development scope. Such systems are however highly desireable
and it is envisaged that such systems would one-day interface with ESMF applications
in part through the configuration attributes services.
Another area that is outside the present scope of the inial ESMF project is automated,
structured archival of attribute information along with other numerical
experiment state. However, again this would be a useful concept to layer on
top of ESMF.

\subsection{Related material}
 Property sheets used in many component based programming environments, for example
the property sheet notion  systems in J2EE \cite[], have parallels with the concepts that need to 
be supported here. All the ESMF applications already contain services of this nature with widely 
varying degress of spohistication \cite[MOM_manual, MITgcm_manual, CCSM_manual]. The widely used dot 
files and dot subdirectories used for all manner of application configuration on UNIX platforms also 
perform similar roles \cite[pinerc,sshrc] to the functions envisaged here, as does the Windows 
registry concept \cite[Windows_registry].


