% $Id: Grid_desc.tex,v 1.12 2007/05/18 18:32:04 cdeluca Exp $
%
% Earth System Modeling Framework
% Copyright 2002-2008, University Corporation for Atmospheric Research, 
% Massachusetts Institute of Technology, Geophysical Fluid Dynamics 
% Laboratory, University of Michigan, National Centers for Environmental 
% Prediction, Los Alamos National Laboratory, Argonne National Laboratory, 
% NASA Goddard Space Flight Center.
% Licensed under the University of Illinois-NCSA License.

The ESMF Grid class is used to describe the geometry and discretization
of a structured physical grid.  It also contains the description of the
grid's underlying topology and the decomposition of the physical grid
across the available computational resources.  The most frequent 
use of the Grid class is to describe physical grids in user
code so that sufficient information is available to perform ESMF
methods such as regridding. 

\begin{center}
\begin{tabular}{|p{6in}|}
\hline
\vspace{.01in}
{\bf Key Features} \\[.01in]
Representation of structured grids, including uniform and
rectilinear grids (e.g. lat-lon grids), curvilinear grids (e.g. displaced pole grids), 
and grids formed by connected logically rectangular regions (e.g. cubed
sphere grids).\\
Support for 2D, 3D, and higher dimension grids.\\ 
Distribution of grids across computational resources for parallel
operations - users set which grid dimensions are distributed.\\
Grids can be created already distributed, so that no single
resource needs global information in the process.\\
Options to define periodicity implicitly via standard shapes
(e.g. sphere, rectangle) or specify it explicitly.\\ 
Options for users to define grid coordinates themselves or call
prefabricated coordinate generation routines for standard grids.\\
Options for one-shot or incremental construction of grids.\\
Options for using a set of pre-defined stagger locations or for setting
custom stagger locations.\\[.03in] \hline
\end{tabular}
\end{center}

\subsubsection{Supported Grids}

The ESMF Grid class is based on the concepts described in {\it A Standard Descriptor 
of Grids Used in Earth System Models} \cite{BalajiGridSpec}.  In this document
Balaji introduces the mosaic concept as a means of describing a wide variety of
Earth system model grids.  A mosaic is composed of logically rectangular
grid tiles connected at their edges.  Mosaic grids includes simple,
single tile grids as a special case.

Grid tiles can have uniform, rectilinear, or curvilinear
coordinates.  The coordinates of {\bf uniform} grids are equally spaced along their
axes, and can be fully specified by the coordinates of the two opposing points
that define the grid's physical span.  The coordinates of {\bf rectilinear} grids
are unequally spaced along their axes, and can be fully specified by giving
the spacing of grid points along each axis.  The coordinates of {\bf curvilinear 
grids} must be specified by giving the explicit set of coordinates for each
grid point.  Curvilinear grids are often uniform or rectangular grids that 
have been warped; for example, to place a pole over a land mass so that it
does not affect the computations performed on an ocean model grid.  Figure
\ref{fig:LogRectGrids} shows examples of each type of grid.

All of these logically rectangular grid types can be combined through edge
connections to form a mosaic.  Cubed sphere and yin-yang grids, shown in 
Figure {not done yet} are examples of mosaic grids.
 
\begin{center}
\begin{figure}
\scalebox{0.9}{\includegraphics{LogRectGrids}}
\caption{Types of logically rectangular grid tiles.  Red circles show the
values needed to specify grid coordinates for each type.}
\label{fig:LogRectGrids}
\end{figure}
\end{center}

\subsubsection{Grid Representation in ESMF}

The ESMF Grid class is a representation of a mosaic grid.  Each ESMF
Grid is constructed of one or more logically rectangular {\bf Tiles}.
A Tile will usually have some physical significance (e.g. the region
of the world covered by one face of a cubed sphere grid).

The piece of a Tile that resides on one DE is called a {\bf LocalTile}.
For example, the six faces of a cubed sphere grid are each Tiles, and
each Tile can be divided across DEs into many LocalTiles. 

Every ESMF Grid contains a DistGrid object, which defines the index space,
topology, distribution, and connectivities for the grid.  It enables
the user to define the complex edge relationships of tripole and other
grids.  The DistGrid can be created explicitly and passed into a Grid
creation routine, or it can be created implicitly if the user takes
a Grid creation shortcut.

\subsubsection{Shortcuts for Shape Specification and Periodicity}
\label{sec:ShapeShortcut}
ESMF has shortcuts for creation of standard Grid {\bf shapes}.  
In many cases, these enable the user to bypass the step of creating 
a DistGrid before creating the Grid.  
{\bf Sphere} and {\bf Box} are the simplest shapes.
A Sphere is periodic in the
longitudinal dimension and has defined poles.  A Box is a cartesion
space up to seven dimensions.  The periodicity of a Box can be
defined on a per-dimension basis by the user.  

If the user's grid shape is too complex for an ESMF shortcut routine,
a DistGrid can be created to specify the shape in detail. 
This DistGrid is then passed into a Grid create call.

\subsubsection{Shortcuts for Specification of Grid Distribution} 

The shortcuts for creating standard Grid shapes have options
for data distribution.  The main distribution options are 
{\bf regular} and {\bf arbitrary}.  {\bf define}

If the user's grid distribution is too complex for an ESMF shortcut routine,
a DistGrid can be created to specify the grid distribution
in detail.  This DistGrid is then passed into a Grid create call.

\subsubsection{Coordinate Specification}

Grid Tile coordinates are specified
according to the information needed for the coordinate type.
Uniform grids are specified with coordinate corner
points, rectilinear grids with coordinate locations for each dimension,
and curvilinear grids with coordinates for each grid point.  Hybrid
grids - for example, a curvilinear horizontal grid with regular spacing
in the vertical dimension - can be specified as well.  Each of these
coordinate types can be set for each of the standard grid shapes
described in section \ref{sec:ShapeShortcut}.  

The table shows how examples of common single Tile grids fall 
into this shape and coordinate type taxonomy.  Note that any
of the grids in the table can have a regular or arbitrary distribution.

\medskip
\begin{tabular}{|p{.9in}|p{1.7in}|p{1.7in}|p{1.7in}|}
\hline
 & {\bf Uniform} & {\bf Rectilinear} & {\bf Curvilinear} \\ 
\hline
{\bf Sphere} & Global uniform lat-lon grid & Gaussian grid & Displaced pole grid \\
\hline
{\bf Box} & Regional uniform lat-lon grid & Gaussian grid section & Mercator grid \\
\hline
\end{tabular}

\subsubsection{Shortcuts for Coordinate Specification and Grid Generation}

There are two ways of specifying coordinates in ESMF.  The
first way is for the user to {\bf set} the coordinates.  The second 
way is to take a shortcut and have the framework {\bf generate}
the coordinates.  

The only ESMF generation routine now available is for Box-shaped
uniform grids.

See Section~\ref{sec:usage:coordstore} for more description and examples of
setting Tile coordinates.

\subsubsection{Stages in Grid Creation} 

In ESMF there are two stages to Grid creation.
\begin{enumerate}
\item Creation of the Grid shape.  At the completion of this
stage, the Grid has a specific topology and distribution, but
empty coordinate arrays.  The Grid can be used as the basis for
allocating a Field.
\item Specifying the Grid coordinates and any other information
required for regridding (this can vary depending on the 
particular regridding method).  At the completion of this
stage, the Grid can be used in a regridding method.
\end{enumerate}
There are shortcut methods in ESMF for both stages.

Each stage has at least one ESMF call, and depending on 
whether shortcuts are used or not, may have more.
This gives the user the opportunity to trade off the 
ease of using a prefabricated grid with the flexibility 
of specifying details about their grid.  Incremental
construction can also be useful when distributing a grid object
before setting coordinates or when multiple components
contribute to the definition of a grid. 

\medskip
\subsubsection{Options for Grid Creation}

The Grid class uses create, set or generate, and 
commit calls during grid creation.



\medskip
\begin{tabular}{|p{2.6in}|p{1in}|p{1in}|p{1.4in}|}
\hline
\multicolumn{4}{|l|}{{\bf Options for Grid Creation}} \\
\hline
Command sequence & Missing & Function & ESMF\_GRIDSTATUS\_ \\ 
\hline
ESMF\_GridCreateEmpty() 
& Topology information and coordinates, other options undefined
& A Grid object shell is allocated but space for 
internal contents is not
& NOT\_READY \\
\hline
ESMF\_GridCreate()\newline
{\bf or} \newline
ESMF\_GridCreateEmpty()\newline
ESMF\_GridSet()\newline
ESMF\_GridSet()\newline
...\newline
ESMF\_GridCommit(\newline
\hspace{.1in} ESMF\_GRIDSTATUS\_SHAPE\_READY)\newline
{\bf or shortcut method, e.g.} \newline
ESMF\_GridCreateShapeSphere()\newline
ESMF\_GridCreateShapeBox()
& Space for coordinates is allocated but coordinate
values are not set
& Can be used as the basis for Field allocation
& SHAPE\_READY\\
\hline
ESMF\_GridCreate()\newline
ESMF\_GridSetCoordFromArray()\newline
ESMF\_GridCommit(\newline
\hspace{.1in} ESMF\_GRIDSTATUS\_REGRID\_READY)\newline
{\bf or} \newline 
ESMF\_GridCreate()\newline
ESMF\_GridGetLocalTile(localTile)\newline
... user fills in localTile coordinate values
ESMF\_GridCommit(\newline
\hspace{.1in} ESMF\_GRIDSTATUS\_REGRID\_READY)\newline
{\bf or shortcut method, e.g.} \newline
ESMF\_GridCreateShapeSphere()\newline
ESMF\_GridGenCoordSphere()\newline
ESMF\_GridCommit(\newline
ESMF\_GRIDSTATUS\_REGRID\_READY)
& Nothing; Grid is complete
& Can be used in regrid methods
& REGRID\_READY\\
\hline
\end{tabular}

\subsubsection{Staggering}

 A useful finite difference technique is to place different physical
quantities at different locations within a grid cell. This {\bf staggering}
of the physical variables on the mesh is introduced so that the difference
of a field is naturally defined at the location of another variable. When creating an 
ESMF Grid the user specifies which stagger locations the grid should 
contain (default is only the cell center).  Staggers can be set for up to the 
maximum grid dimension.  Later, when creating an  ESMF\_Field on the Grid the user
will specify upon which stagger location the Field and their data will reside.

There are predefined staggers and predefined stagger locations for ease of use,
however, the user can also specify their own if they need something different 
(see Section~\ref{ref:stagger} for a more in depth discussion of staggers).
 
For examples and a full description of the stagger interface 
please see Section~\ref{sec:usage:staggerloc}. 

To add:

{\bf Treatment of global and global data}{br}
 


