% $Id: Grid_desc.tex,v 1.20 2007/05/23 21:37:06 oehmke Exp $
%
% Earth System Modeling Framework
% Copyright 2002-2008, University Corporation for Atmospheric Research, 
% Massachusetts Institute of Technology, Geophysical Fluid Dynamics 
% Laboratory, University of Michigan, National Centers for Environmental 
% Prediction, Los Alamos National Laboratory, Argonne National Laboratory, 
% NASA Goddard Space Flight Center.
% Licensed under the University of Illinois-NCSA License.

The ESMF Grid class is used to describe the geometry and discretization
of a structured physical grid.  It also contains the description of the
grid's underlying topology and the decomposition of the physical grid
across the available computational resources.  The most frequent 
use of the Grid class is to describe physical grids in user
code so that sufficient information is available to perform ESMF
methods such as regridding.  

\begin{center}
\begin{tabular}{|p{6in}|}
\hline
\vspace{.01in}
{\bf Key Features} \\[.01in]
Representation of structured grids, including uniform and
rectilinear grids (e.g. lat-lon grids), curvilinear grids (e.g. displaced pole grids), 
and grids formed by connected logically rectangular regions (e.g. cubed
sphere grids).\\
Support for 2D, 3D, and higher dimension grids.\\ 
Distribution of grids across computational resources for parallel
operations - users set which grid dimensions are distributed.\\
Grids can be created already distributed, so that no single
resource needs global information in the process.\\
Options to define periodicity and other connectivities implicitly via
shape shortcuts or specify them explicitly.\\ 
Options for users to define grid coordinates themselves or call
prefabricated coordinate generation routines for standard grids.\\
Options for incremental construction of grids.\\
Options for using a set of pre-defined stagger locations or for setting
custom stagger locations.\\ [.03in] \hline
\end{tabular}
\end{center}

\subsubsection{Supported Grids}

The ESMF Grid class is based on the concepts described in {\it A Standard Descriptor 
of Grids Used in Earth System Models} [Balaji 2006].  In this document
Balaji introduces the mosaic concept as a means of describing a wide variety of
Earth system model grids.  A mosaic is composed of logically rectangular
grid tiles connected at their edges.  Mosaic grids includes simple,
single tile grids as a special case.

Grid tiles can have uniform, rectilinear, or curvilinear
coordinates.  The coordinates of {\bf uniform} grids are equally spaced along their
axes, and can be fully specified by the coordinates of the two opposing points
that define the grid's physical span.  The coordinates of {\bf rectilinear} grids
are unequally spaced along their axes, and can be fully specified by giving
the spacing of grid points along each axis.  The coordinates of {\bf curvilinear 
grids} must be specified by giving the explicit set of coordinates for each
grid point.  Curvilinear grids are often uniform or rectangular grids that 
have been warped; for example, to place a pole over a land mass so that it
does not affect the computations performed on an ocean model grid.  Figure
\ref{fig:LogRectGrids} shows examples of each type of grid.

All of these logically rectangular grid types can be combined through edge
connections to form a mosaic.  Cubed sphere and yin-yang grids, shown in 
Figure {not done yet} are examples of mosaic grids.
 
\begin{figure}
\scalebox{0.9}{\includegraphics{LogRectGrids}}
\caption{Types of logically rectangular grid tiles.  Red circles show the
values needed to specify grid coordinates for each type.}
\label{fig:LogRectGrids}
\end{figure}

\subsubsection{Grid Representation in ESMF}

The ESMF Grid class is a representation of a mosaic grid.  Each ESMF
Grid is constructed of one or more logically rectangular {\bf Tiles}.
A Tile will usually have some physical significance (e.g. the region
of the world covered by one face of a cubed sphere grid).

The piece of a Tile that resides on one DE is called a {\bf LocalTile}.
For example, the six faces of a cubed sphere grid are each Tiles, and
each Tile can be divided across DEs into many LocalTiles. 

Every ESMF Grid contains a DistGrid object, which defines the index space,
topology, distribution, and connectivities for the grid.  It enables
the user to define the complex edge relationships of tripole and other
grids.  The DistGrid can be created explicitly and passed into a Grid
creation routine, or it can be created implicitly if the user takes
a Grid creation shortcut.

\subsubsection{Shortcuts for Shape Specification and Periodicity}
\label{sec:ShapeShortcut}
ESMF has shortcuts for creation of standard Grid {\bf shapes} up
to 3D.  In many cases, these enable the user to bypass the step of creating 
a DistGrid before creating the Grid.  The basic call is 
ESMF\_GridCreateShape().  With this call, the user can specify for
each dimension whether it is periodic, is not periodic, is a pole, or
is a bipole.

The table below shows how standard shapes can be created
using this call.  Connectivities are specified using the
ESMF\_GridConn parameter.  Note that the dimensions
the connections are on is arbitrary. For example, putting the
settings for specifying a sphere into {\tt connDim2} and
{\tt connDim3\} instead would still result in a sphere, but
oriented along a different axis.

\medskip
\begin{tabular}{|l|c|c||c|c||}
\hline
& {\bf connDim1(1)} & {\bf connDim1(2)}  & {\bf connDim2(1)} & {\bf connDim2(2)}  \\
\hline
{\bf Rectangle}  & NONE & NONE & NONE & NONE \\
{\bf Bipole Sphere} & POLE & POLE & PERIODIC & PERIODIC \\
{\bf Tripole Sphere} & POLE & BIPOLE & PERIODIC & PERIODIC \\
{\bf Cylinder} & NONE & NONE & PERIODIC & PERIODIC \\
{\bf Torus}  & PERIODIC & PERIODIC & PERIODIC & PERIODIC \\
\hline
\hline
\end{tabular}

If the user's grid shape is too complex for an ESMF shortcut routine,
or involves more than three dimensions, a DistGrid can be created
to specify the shape in detail.  This DistGrid is then passed
into a Grid create call.

\subsubsection{Shortcuts for Specification of Grid Distribution}\label{sec:desc:dist}

The shortcuts for creating standard Grid shapes have options
for data distribution (also referred to as decomposition).  The
user specifies which dimensions 
are to be distributed through a coordinate dependency (coordDep)
argument.

The main distribution options are block, regular and arbitrary.
A {\bf block} distribution is one in which the same number of
contiguous grid points are assigned to each DE in the
distributed dimension.  A {\bf regular} distribution is one in which
unequal numbers of contiguous gridpoints are assigned to each
DE in the distributed dimension.  An {\bf arbitrary} distribution is
one in which any gridpoint can be assigned to any DE.  Any ofthese
distribution options can be applied to any of the grid shapes (i.e.,
rectangle) or types (i.e., rectilinear).

Figure \ref{fig:GridDecomps} illustrates options for distribution.
\begin{figure}
\scalebox{0.9}{\includegraphics{GridDecomps}}
\caption{Examples of block and regular decompositions for
a 6x6 matrix {\bf a}, and an arbitrary decomposition for a 6x3 matrix {\bf b}.}
\label{fig:GridDecomps}
\end{figure}

A distribution can also be specified using the DistGrid.  This
DistGrid is then passed into a Grid create call.

\subsubsection{Coordinate Specification}

Grid Tile coordinates are specified
according to the information needed for the coordinate type.
Uniform grids are specified with coordinate corner
points, rectilinear grids with coordinate locations for each dimension,
and curvilinear grids with coordinates for each grid point.  Hybrid
grids - for example, a curvilinear horizontal grid with regular spacing
in the vertical dimension - can be specified as well.  Each of these
coordinate types can be set for each of the standard grid shapes
described in section \ref{sec:ShapeShortcut}.  

The table shows how examples of common single Tile grids fall 
into this shape and coordinate type taxonomy.  Note that any
of the grids in the table can have a regular or arbitrary distribution.

\medskip
\begin{tabular}{|p{.9in}|p{1.7in}|p{1.7in}|p{1.7in}|}
\hline
 & {\bf Uniform} & {\bf Rectilinear} & {\bf Curvilinear} \\ 
\hline
{\bf Sphere} & Global uniform lat-lon grid & Gaussian grid & Displaced pole grid \\
\hline
{\bf Box} & Regional uniform lat-lon grid & Gaussian grid section & Mercator grid \\
\hline
\end{tabular}

\subsubsection{Shortcuts for Coordinate Specification and Grid Generation}

There are two ways of specifying coordinates in ESMF.  The
first way is for the user to {\bf set} the coordinates.  The second 
way is to take a shortcut and have the framework {\bf generate}
the coordinates.  

The only ESMF generation routine now available is for Box-shaped
uniform grids.

See Section~\ref{sec:usage:coordstore} for more description and examples of
setting Tile coordinates.

\subsubsection{Staggering}

A useful finite difference technique is to place different physical
quantities at different locations within a grid cell. This {\bf staggering}
of the physical variables on the mesh is introduced so that the difference
of a field is naturally defined at the location of another variable. 

The ESMF Grid class supports a wide variety of stagger locations (including
those sufficient to specify any of the standard Arakawa staggers). The 
Grid supports staggers located at cell centers, corners, edges, 
faces, and higher dimensional elements. The user may specify
common stagger locations using predefined values, or may construct
custom ones for cases which the predefines don't cover.

As a default the ESMF Grid class provides symmetric staggering, so
that cell centers are enclosed by cell perimeter (e.g. corner) 
stagger locations. This means the coordinate arrays for
some stagger locations may be padded to one element larger in some
dimensions than the cell center arrays to allow the enclosure. 
However, to achieve other types of staggering, the user may alter 
or eliminate this padding by using the appropriate options when adding
a stagger location to a Grid. 
 
For examples and a full description of the stagger interface 
please see Section~\ref{sec:usage:staggerloc}. 

\subsubsection{Stages in Grid Creation} 

In ESMF there are two stages to Grid creation.
\begin{enumerate}
\item Creation of the Grid shape.  At the completion of this
stage, the Grid has a specific topology and distribution, but
empty coordinate arrays.  The Grid can be used as the basis for
allocating a Field.
\item Specifying the Grid coordinates and any other information
required for regridding (this can vary depending on the 
particular regridding method).  At the completion of this
stage, the Grid can be used in a regridding method.
\end{enumerate}
There are shortcut methods in ESMF for both stages.

Each stage has at least one ESMF call, and depending on 
whether shortcuts are used or not, may have more.
This gives the user the opportunity to trade off the 
ease of using a prefabricated grid with the flexibility 
of specifying details about the grid.  Incremental
construction can also be useful when distributing a grid object
before setting coordinates or when multiple components
contribute to the definition of a grid. 

\subsubsection{Options for Grid Creation}

The Grid class uses create, set or generate, and 
commit calls during grid creation.  The following chart
shows typical sequences.

There are two simple rules for when commit calls are required:
\begin{enumerate}
\item An ESMF\_GridCommit() call with an ESMF\_GRIDSTATUS\_SHAPE\_READY
argument is required when ESMF\_GridCreateEmpty() is used to initiate
Grid creation.
\item An ESMF\_GridCommit() call with an ESMF\_GRIDSTATUS\_REGRID\_READY
argument is required before using any ESMF Grid in a regridding call.
\end{enumerate}

\medskip
\begin{tabular}{|p{2.6in}|p{1in}|p{1in}|p{1.4in}|}
\hline
\multicolumn{4}{|l|}{{\bf Options for Grid Creation}} \\
\hline
Command sequence & Missing & Function & ESMF\_GRIDSTATUS\_ \\ 
\hline
ESMF\_GridCreateEmpty() 
& Topology information and coordinates, other options undefined
& A Grid object shell is allocated but space for 
internal contents is not
& NOT\_READY \\
\hline
ESMF\_GridCreate()\newline
{\bf or} \newline
ESMF\_GridCreateEmpty()\newline
ESMF\_GridSet()\newline
ESMF\_GridSet()\newline
...\newline
ESMF\_GridCommit(\newline
\hspace{.1in} ESMF\_GRIDSTATUS\_SHAPE\_READY)\newline
{\bf or shortcut method, e.g.} \newline
ESMF\_GridCreateShape()\newline
& Space for coordinates is allocated but coordinate
values are not set
& Can be used as the basis for Field allocation
& SHAPE\_READY\\
\hline
ESMF\_GridCreate()\newline
ESMF\_GridSetCoordFromArray()\newline
ESMF\_GridCommit(\newline
\hspace{.1in} ESMF\_GRIDSTATUS\_REGRID\_READY)\newline
{\bf or} \newline 
ESMF\_GridCreate()\newline
ESMF\_GridGetLocalTile(localTile)\newline
... user fills in localTile coordinate values
ESMF\_GridCommit(\newline
\hspace{.1in} ESMF\_GRIDSTATUS\_REGRID\_READY)\newline
{\bf or shortcut method, e.g.} \newline
ESMF\_GridCreateShapeSphere()\newline
ESMF\_GridGenCoordUni()\newline
ESMF\_GridCommit(\newline
ESMF\_GRIDSTATUS\_REGRID\_READY)
& Nothing; Grid is complete
& Can be used in regrid methods
& REGRID\_READY\\
\hline
\end{tabular}



 


