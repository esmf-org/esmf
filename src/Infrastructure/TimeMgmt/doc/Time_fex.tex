% $Id: Time_fex.tex,v 1.1 2002/11/14 23:58:45 jwolfe Exp $

In the example below we demonstrate basic usage of the {\tt Time} class.  For
a detailed description of the {\tt Time} class interface, see 
Appendix A.

In the example below we initialize a {\tt Time} and increment it by a
number of days and seconds.  Since no memory is allocated from the heap
when a {\tt Time} is initialized, there is no need to deallocate a 
{\tt Time} object.

\begin{verbatim}

  use ESMF_TimeMgmtMod

  type(ESMF_Time) :: startTime, endTime

!-------------------------------------------------------------------------------    
! Initialize a time interval to 90 days and 1800 seconds. 
!-------------------------------------------------------------------------------

  startTime = ESMF_TimeInit(90, 1800) 

!-------------------------------------------------------------------------------    
! Increment the time by 10 days and 1200 seconds.
!-------------------------------------------------------------------------------
       
  endTime = ESMF_TimeIncrement(startTime, 10, 1200)

\end{verbatim}









