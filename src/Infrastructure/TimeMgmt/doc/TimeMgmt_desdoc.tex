% $Id: TimeMgmt_desdoc.tex,v 1.1 2002/11/14 23:51:21 jwolfe Exp $

\documentclass[]{article}

\usepackage{epsf}
\usepackage{html}
\usepackage[T1]{fontenc}

\textwidth 6.5in
\textheight 8.5in
\addtolength{\oddsidemargin}{-.75in}

\begin{document}

\bodytext{BGCOLOR=white LINK=#083194 VLINK=#21004A}

\begin{titlepage}

\begin{center}
{\Large Earth System Modeling Framework } \\
\vspace{.25in}
{\Large {\bf Time Management Library Design}} \\
\vspace{.25in}
{\large {\it Authors}}
\vspace{.5in}
\end{center}

\begin{latexonly}
\vspace{5.5in}
\begin{tabular}{p{5in}p{.9in}}
\hrulefill \\
\noindent {\bf NASA High Performance Computing and Communications Program} \\
\noindent Earth and Space Sciences Project \\
\noindent CAN 00-OES-01 \\
\noindent http://www.esmf.ucar.edu \\
\end{tabular}
\end{latexonly}

\end{titlepage}

\tableofcontents

\newpage
%\section{Synopsis}
% $Id: TimeMgmt_syn.tex,v 1.1 2002/11/14 23:55:26 jwolfe Exp $
\section{Synopsis}

The Earth System Modeling Framework (ESMF) Time Management Library provides utility functions for time and date calculations, and higher-level functions that control model time stepping and alarms.  










The interface has been designed with use by Earth system models specifically in mind. The use of encapsulated classes for date, time and time of day  
allows for a wide range of values and precision so that applications from weather forecasting to paleoclimate 
simulation are supported.  

The design of this Time Management Library is based on the time management 
utilities in the \htmladdnormallink{Flexible Modeling System}
{http://www.gfdl.gov/~fms} (FMS) from the NOAA Geophysical 
Fluid Dynamics Laboratory.

It will be built on infrastructure derived from the \htmladdnormallink
{Portable Extensible Toolkit for Scientific Computation}
{http://www-fp.mcs.anl.gov/petsc/} (PETSc)

%\section{Algorithmic Description}
%% $Id: comp_alg.tex,v 1.5.2.3 2010/02/01 20:48:49 svasquez Exp $
%
% Earth System Modeling Framework
% Copyright 2002-2010, University Corporation for Atmospheric Research, 
% Massachusetts Institute of Technology, Geophysical Fluid Dynamics 
% Laboratory, University of Michigan, National Centers for Environmental 
% Prediction, Los Alamos National Laboratory, Argonne National Laboratory, 
% NASA Goddard Space Flight Center.
% Licensed under the University of Illinois-NCSA License.

%\section{Algorithmic Description}

<Description of the continuous and discrete scientific algorithms used
in the software.  May reference rather than describe algorithms explicitly.>





%\section{Requirements}
%\input{TimeMgmt_req}

%\section{Architecture}
%% $Id: comp_arch.tex,v 1.2 2002/07/25 17:15:39 eschwab Exp $

%\section{Architecture}

<Describe layering strategy and interaction of major components,
provide examples of high-level interfaces.>


\section{Time Class}

%\subsection{Description}
% $Id: Time_desc.tex,v 1.5 2003/08/14 20:31:43 cdeluca Exp $
\label{sec:Time}

A Time represents a specific point in time.  In order to accommodate
the range of time scales in Earth system applications, Times in
the ESMF can be specified in many different ways, from years to 
nanoseconds.  The Time interface is designed so that you select one or 
more options from a list of time units in order to specify a 
Time. The options for specifying a Time are shown in 
Figure\ref~{fig:TimeOpts}.  

There are Time methods defined for setting and getting a
Time, incrementing and decrementing a Time by a TimeInterval,
taking the difference between two Times, and comparing Times.
Special quantities such as the middle of the month and the 
day of the year associated with a particular Time can be retrieved. 
There is a method for returning the Time value as a string in 
the ISO 8601 format YYYY-MM-DDThh:mm:ss.

A Time that is specified in hours, minutes, seconds, or subsecond intervals 
does not need to be associated with a standard calendar; a Time whose
specification includes larger units must be.  The ESMF representation
of a calendar, the Calendar class, is described in Section~\ref{sec:Calendar}.
The {\tt ESMF_TimeSet} method can be called to associate a Time with a 
Calendar.  If an operation is requested in which a Calendar is necessary 
and one has not been set, the ESMF method will return an error.

In the ESMF the TimeInterval class used to represent time periods.
This class is frequently used in combination with the Time class.
The Clock class, for example, advances model time by 
 
Times are used by other classes in the ESMF timekeeping system,
 such as Clocks (Section~\ref{sec:Clock}) and Alarms 
(Section~\ref{sec:Alarm}).






\subsection{Design}

\subsubsection{Class Definition}

The attributes of the time class expressed as a Fortran 90 derived type are:

\noindent type time\_t \\
\indent private \\
\indent integer :: days \\
\indent integer :: seconds \\
\noindent end type time\_t

\subsubsection{Design Strategy}

The precision and range requirements of time intervals and dates determine their representation in 
terms of native machine types (i.e, floating point numbers and integers). The time and date classes
are defined to represent time intervals and dates, respectively, in order to insulate the user interface 
from the underlying representation.  This not only makes it easier to pass arguments (e.g., passing a
date argument versus passing all the integers that represent the components of a date), it also means 
that the interface doesn't have to change if the underlying representation were changed to support 
new precision or range requirements.

When an error is encountered the functions in this library will call a private error handling routine 
which issues a message explaining the error and then calls exit. The user may be able to customize the
error handling.

\section{Date Class}

%\subsection{Description}
% $Id: Date_desc.tex,v 1.1 2002/11/14 23:47:36 jwolfe Exp $

The {\tt Date} class provides a set of functions for manipulating dates.
These include setting and retrieving dates, incrementing and decrementing 
dates by a specified time interval, taking the difference of two dates,
determining whether one date is later than another, and computing the
day of year of a given date.
   
The {\tt Date} class contains attributes representing year, month and day 
quantities and a time of day.  It also contains a calendar which 
stores, for a given year, such quantities as the number of days per 
month and per year.  Gregorian and no-leap year calendars are currently 
supported.  

The algorithm to convert from Gregorian to Julian days is from 
Henry F. Fliegel and Thomas C. Van Flandern, in Communications of 
the ACM (CACM, volume 11, number 10, October 1968, p.657).  Julian 
day refers to the number of days since a reference day.  For the 
algorithm used, this reference day is November 24, -4713 in the Gregorian 
calendar.  The algorithm is valid through all future dates, assuming 
standard corrections are applied (at 4 years, 100 years,
and 400 years).




\subsection{Design}

\subsubsection{Class Definition}

The attributes of the date class expressed as a Fortran 90 derived type are:

\noindent type date\_t \\
\indent private \\
\indent integer :: year \\
\indent integer :: month \\
\indent integer :: day \\
\indent integer :: sec \\
\noindent end type date\_t

\subsubsection{Design Strategy}

As with the time class, the date class hides the representation of its internal attributes from the 
user, thereby increasing the class's extensibility and portability.

\section{Review Status}

\noindent{\bf Requirements and Design Review} \\

\begin{tabular}{r p{1.3in} p{2in}}
{\bf Review Date:} & March 1, 2001 \\ \\
{\bf Reviewers:}   & Byron Boville        & NCAR/CGD \\
                   & Dave Williamson      & NCAR/CGD \\
                   & Phil Rasch           & NCAR/CGD \\
                   & Cecelia DeLuca       & NCAR/SCD \\
                   & Jim Rosinski         & NCAR/CGD
\end{tabular}

%\section{Glossary}
% $Id: TimeMgmt_glos.tex,v 1.1 2002/11/14 23:51:54 jwolfe Exp $
\section{Glossary}

\begin{description}

\item [date] \label{glos:date} A date is used to specify an instant of time at the Greenwich meridian.  It consists
of year, month, day of month and time of day components.

\item [time] \label{glos:time} A time is used to specify an interval of time. 
              
\item [day of year] \label{glos:dayofyear} The day number in the calendar year. January 1 is day 1 of the year. 
Day of year expressed in a floating point format is used to express the day number plus the time of day 
at Greenwich. For example, assuming a Gregorian calendar: 

\begin{tabular}{ll}
{\bf date}              & {\bf day of year} \\
\hline 
10 January 2000, 6Z     & 10.25 \\
31 December 2000, 18Z   & 366.75 
\end{tabular}

\item [no-leap calendar] \label{glos:noleap} Every year uses the same months and days per month as in a non-leap 
year of a Gregorian calendar.

\end{description}













%\section{Bibliography}
%\bibliography{comp} 
%\bibliographystyle{plain}
%\addcontentsline{toc}{section}{Bibliography}

\section*{Appendix:  Fortran Interface}
\addcontentsline{toc}{section}{Appendix:  Fortran Interface}

%\section{ESMF_Time Interface}
\input{ESMF_TimeMod}

%\section{ESMF_Date Interface}
%\input{ESMF_Date}

%\section{ESMF_TimeMgr Interface}
%\input{ESMF_TimeMgr}

%\section{ESMF_Alarm Interface}
%\input{ESMF_Alarm}

%\section*{Appendix:  C Interface}
%\addcontentsline{toc}{section}{Appendix:  C Interface}

%\section{MF_Time Interface}
%\input{MF_Time}

%\section{MF_Date Interface}
%\input{MF_Date}

%\section{MF_TimeMgr Interface}
%\input{MF_TimeMgr}

%\section{MF_Alarm Interface}
%\input{MF_Alarm}

\end{document}
A







