% $Id: TimeMgmt_glos.tex,v 1.1 2002/11/14 23:51:54 jwolfe Exp $
\section{Glossary}

\begin{description}

\item [date] \label{glos:date} A date is used to specify an instant of time at the Greenwich meridian.  It consists
of year, month, day of month and time of day components.

\item [time] \label{glos:time} A time is used to specify an interval of time. 
              
\item [day of year] \label{glos:dayofyear} The day number in the calendar year. January 1 is day 1 of the year. 
Day of year expressed in a floating point format is used to express the day number plus the time of day 
at Greenwich. For example, assuming a Gregorian calendar: 

\begin{tabular}{ll}
{\bf date}              & {\bf day of year} \\
\hline 
10 January 2000, 6Z     & 10.25 \\
31 December 2000, 18Z   & 366.75 
\end{tabular}

\item [no-leap calendar] \label{glos:noleap} Every year uses the same months and days per month as in a non-leap 
year of a Gregorian calendar.

\end{description}











